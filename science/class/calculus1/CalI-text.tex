%original : April 17,  24, 1996
%original : May 1, 8, 1996
%original : (積分)May 22, 29, 1996
%original : June 5, 6, 1996
%%revised : April 16, 17, 23, 24, 30, 1997
%%revised : May 14, 1997

\documentstyle[12pt]{jarticle}
%A4 Size Setting
\topmargin = -0.5cm
\oddsidemargin = 0cm \evensidemargin = 0cm
\textheight = 23cm \textwidth = 15cm % default 16cm
%A4 size Setting End

\title{CALCULUS I}
      
\author{Hiroshi SUZUKI\thanks{E-mail:hsuzuki@icu.ac.jp}\\ 
        Department of Mathematics \\ 
        International Christian University}

\renewcommand{\thepage}{%
	\arabic{section}--\arabic{page}}
\newcommand{\mysection}[1]{%
	\section{#1}\setcounter{page}{1}}

\newtheorem{thm}{定理}[section]
\newtheorem{prop}[thm]{命題}
\newtheorem{lemma}[thm]{補題}
\newtheorem{cor}[thm]{系}
\newtheorem{exercise}{練習問題}[section]
\newtheorem{example}{例}[section]
\newtheorem{problem}{問題}[section]
\newtheorem{defin}{定義}[section]
\newenvironment{definition}{\begin{defin} \rm}{\end{defin}}
\newenvironment{ex}{\begin{exercise} \rm}{\end{exercise}}
\newenvironment{eg}{\begin{example} \rm}{\end{example}}
\newenvironment{prob}{\begin{problem} \rm}{\end{problem}}
\newcommand{\remarks}{\vspace{2ex}\noindent{\bf Remarks.\quad}}
\newcommand{\note}{\vspace{2ex}\noindent{\gt 注\quad}}
\newcommand{\proof}{{\gt 証明\quad}}
\newcommand{\qed}{\hfill\hbox{\rule{6pt}{6pt}}}
\newcommand{\bZ}{\mbox{\boldmath $Z$}}
\newcommand{\bR}{\mbox{\boldmath $R$}}
\newcommand{\bC}{\mbox{\boldmath $C$}}
\newcommand{\bQ}{\mbox{\boldmath $Q$}}

\begin{document}
\setcounter{page}{0}
\maketitle
\newpage
\mysection{関数の極限と連続}
\begin{definition}
点 $x = a$ の近くで定義されている関数 $y = f(x)$ において、次の3つの条件が成り立つとき、$y = f(x)$ は、$x = a$  で、連続である (continuous) という。
\begin{itemize}
\item[$(1)$] $f(a)$ が、定義されている。
\item[$(2)$] ${\displaystyle\lim_{x\rightarrow a}f(x)}$  が、存在する。
\item[$(3)$] ${\displaystyle\lim_{x\rightarrow a}f(x) = f(a)}$。
\end{itemize}
また、ある区間の各点で、連続な関数を、連続関数という。
\end{definition}

\begin{eg}
$x = 0$ において、次の4つの関数を考える。
$$\begin{array}{ll}
(a)\;f(x) = \left\{\begin{array}{ll}1/x & x\neq 0\\ 0 & x = 0 \end{array}\right. &
(b)\;g(x) = \left\{\begin{array}{ll}\sin(1/x)& x\neq 0\\ 0 & x = 0 \end{array}\right.\\
(c)\;h(x) = \left\{\begin{array}{ll}x\sin(1/x) & x\neq 0\\ 0 & x = 0 \end{array}\right. & (d)\;l(x) = \left\{\begin{array}{ll}x & x\mbox{ 無理数}\\ 0 & x\mbox{ 有理数} \end{array}\right.
\end{array}$$
$(a)$、$(b)$ は、$x = 0$ において、連続ではない。$(c)$、$(d)$ は、連続である。しかし、$(d)$ は、$x = 0$ 以外の各点で不連続である。
\end{eg}

\begin{eg}
定数関数 $c$、多項式、$\sin x$、$\cos x$、$e^x$ は、各点で連続、また、連続関数の和、差、定数倍、積も、連続関数。商も、分母が零にならない範囲で連続関数である。
\end{eg}

\begin{prop}   \label{prop:inter}
閉区間 $[a,b]$ 上で連続な関数 $f(x)$ において、$C$ を、$f(a)$ と、$f(b)$ の間の値とすると、$f(c) = C$ となる点 $c$ が、区間 $[a,b]$ 内にある。
\end{prop}

\begin{eg}
$f(x) = 4x^5 - 10x^4 - 20x^3 + 4x^2 + 16x - 15$ とする。$f(-2) = -15$、$f(-1) = 15$、$f(0) = -15$、$f(1) = 15$、$f(2) = -15$、$f(3) = 15$。従って、5つの区間 $[-2,-1]$、$[-1,0]$、$[0,1]$、$[1,2]$、$[2,3]$ の内部で、$f(x) = 0$ となる点、すなわち根を少なくとも一つずつ持つ。$f(x)$ は、5次多項式だから、高々5個の実根を持つ。すなわち、この5個以外には、根を持たず、これらの区間に丁度一つずつあることも解ります。(なぜでしょう。)
\end{eg}

\begin{prop}  \label{prop:maxmin}
閉区間 $[a,b]$ 上で連続な関数 $f(x)$ は、$[a,b]$ 上の最大・最小をとる。
\end{prop}

\note 上の命題で、閉区間でない場合は、必ずしも、最大・最小を持つとは限らない。

例えば、$f(x) = -x^2 + 2x + 1$、$(0 < x < 2)$ とすると、$1 < f(x) \leq 2 = f(1)$ だから、区間 $(0,2)$ で最大値は取るが、最小値は取らない。

\newpage
\mysection{微分係数と導関数}
\begin{definition}
関数 $f(x)$ が、点 $x = a$ 及びその近くで定義されていて、かつ
$$\lim_{x\rightarrow a}\frac{f(x)-f(a)}{x-a}$$
が存在するとき、この値を $f(x)$ の点 $a$ における微分係数と言い、$f'(a)$ と書く。関数 $f(x)$ が、各点 $a$ で微分可能であるとき、$a$ に $f'(a)$ を対応させる関数を $f(x)$ の導関数と言い、$f'(x)$、$df/dx$、$Df$ で表す。関数 $f(x)$ から、その導関数 $f'(x)$ を求めることを、微分するという。
\end{definition}

\begin{eg} 
$f, g$  を微分可能な関数、$c$ を定数とすると以下が成り立つ。
$$(f + g)' = f' + g'\quad (cf)' = cf'\quad (fg)' = f'g + fg'\quad \left(\frac{f}{g}\right) = \frac{f'g - fg'}{g^2}$$
\end{eg}

\begin{prop}
関数 $f(x)$ が、点 $a$ で微分可能ならば、点 $a$ で、連続である。
\end{prop}
\proof
関数 $a(x)$ を次のように定義する。
$$a(x) = \left\{\begin{array}{ll}
\displaystyle{\frac{f(x)-f(a)}{x-a}} & x\neq a\\ f'(a) & x = a
\end{array}\right.$$
すると、$a(x)$ は、定義から、点 $a$ で連続。従って、
$$f(x) = a(x)(x-a) + f(a)$$
も、点 $a$ で連続。
\qed

\medskip
\note 上の命題において、逆は必ずしも成り立たない。すなわち、連続でも、微分可能とは言えない。例えば、$f(x) = |x|$、$g(x) = x\sin (1/x)$ は、ともに $x = 0$ で連続だが、微分可能ではない。Weierstrass (1872) は、ある区間の各点で微分可能でない連続関数の実例を作って、当時の数学会に衝撃を与えたとの事である。これは、Weierstrass 関数といわれる。フラクタル図形も至る所微分不可能な曲線である。

\begin{eg}
\begin{enumerate}
\item (多項式の微分)$f(x) = x^n$ の、$x = a$ における、微分係数と導関数。
\begin{eqnarray*}
\lefteqn{\lim_{x\rightarrow a}\frac{x^n - a^n}{x-a}}\\
& = & \lim_{x\rightarrow a}\frac{(x-a)(x^{n-1} + ax^{n-2} + \cdots + a^{n-2}x + a^{n-1})}{x-a}\\
& = & \lim_{x\rightarrow a} (x^{n-1} + ax^{n-2} + \cdots + a^{n-2}x + a^{n-1}) \\
& = & na^{n-1}
\end{eqnarray*}
従って、$f(x) = x^n$  の導関数は、$f'(x) = nx^{n-1}$。

\item (三角関数の微分)$f(x) = \sin x$ の、$x = a$ における、微分係数と導関数。
\begin{eqnarray*}
\lefteqn{\lim_{x\rightarrow a}\frac{\sin x - \sin a}{x-a}}\\
\lefteqn{x = a+h \mbox{ と置くと、}}\\
& = & \lim_{h\rightarrow 0}\frac{\sin(a+h) - \sin a}{h}\\
& = & \lim_{h\rightarrow 0}\frac{\cos(a + h/2)\sin(h/2)}{h/2}\\
& = & \lim_{h\rightarrow 0}{\cos(a + h/2)}\cdot 
\lim_{h\rightarrow 0}\frac{\sin(h/2)}{h/2}\\
& = & \cos a
\end{eqnarray*}
従って、$f(x) = \sin x$ の導関数は、$f'(x) = \cos x$。
ここで、以下の公式を用いている。
$$\lim_{\theta\rightarrow 0}\frac{\sin\theta}{\theta} = 1$$
これは、単位円の扇形と、それを挟む、三角形の面積を用いた次の不等式から得られる。
$$\frac12\sin\theta \leq \pi\cdot\frac{\theta}{2\pi} \leq \frac12\tan\theta, \quad 1\leq \frac{\theta}{\sin \theta} \leq \frac{1}{\cos\theta}$$

\item (指数関数の微分)$(e^x)' = e^x$。ここで、
$$e = \lim_{n\to \infty}\left(1 + \frac1n\right)^n.$$
とする。$e = 2.71828182845904523\ldots$ は、無理数のなかでも、特に、超越数と呼ばれ、どんな有理数係数の多項式の根にもなっていないことが知られている。
$$\lim_{x\to a}\frac{e^x - e^a}{x-a} = \lim_{h\to 0}\frac{e^{a+h} - e^a}{h} = e^a\lim_{h\to 0}\frac{e^h - 1}{h}$$
従って、$\lim_{h\to 0}\frac{e^h - 1}{h} = 1$ を示せばよい。

$e^h = 1 + 1/t$ とおく。$h = \log(1 + 1/t)$ だから、
$$\frac{e^h - 1}{h} = \frac{1/t}{\log(1+1/t)} = \frac{1}{\log(1+1/t)^t}$$
従って、結局、$h\to 0$ のとき、$e^h \to 0$ のとき、従って、$t\to \infty$ で、$(1 + 1/t)^t \to 0$ を言えばよい。実は、これは上の、自然対数 $e$ の定義から得られる。
\end{enumerate}
\end{eg}

\note
\begin{itemize}
\item $f(x) = a^x$ $(a > 0)$ としたとき、$f(x)$ の導関数は、どうなるでしょうか。この様な指数関数は、「正確には」どのように定義したらよいでしょうか。これらの定義には、実数の連続性が深くかかわった、次の事が重要です。「有界な単調数列は収束する。」この様な、数学的に厳密に議論することにはここでは十分立ち入れません。参考書にあげた、高木貞治著「解析概論」などに挑戦して下さい。
\item $(\log x)' = 1/x$ を定義に従って、示してみましょう。
\item 最初に述べた、$e$ の定義、によって、ある数が定まることを確かめて下さい。ここでも、
「有界な単調数列は収束する。」が重要です。
\item これらの、実数の連続性が深くかかわった事については、数学通論 I (BCMM I) で学びます。
\end{itemize}



\newpage
\mysection{微分法}
$A$. 合成関数の微分
\begin{prop}
$g(x)$ は、点 $a$ で微分可能、$f(x)$ は、点 $g(a)$ で微分可能とする。このとき、
$$\frac{d}{dx}f(g(a)) = f'(g(a))g'(a)$$
\end{prop}
\proof
$F(x) = f(g(x))$ とおく。すると、
\begin{eqnarray*}
F'(a) & = & \lim_{x\rightarrow a}\frac{f(g(x)) - f(g(a))}{x-a}\\
        & = & \lim_{x\rightarrow a}\frac{f(g(x)) - f(g(a))}{g(x) - g(a)}\lim_{x\rightarrow a}\frac{g(x) - g(a)}{x-a}\\
        & = & f'(g(a))g'(a)
\end{eqnarray*}
$g(x)$ は、点 $x = a$ で微分可能だから、連続、すなわち、$x$ が、$a$ に近づくとき、$g(x)$  は、$g(a)$ に近づく。
\qed

\begin{eg}
$f(x) = e^{-x^2}$ とする。このとき、$f'(x) = -2x\cdot e^{-x^2}$。$f''(x) = (-2 + 4x^2)\cdot e^{-x^2}$。$f(x)$ を $n$ 回微分したものを $f(x)$ の $n$ 階導関数と言い $f^{(n)}(x)$ と書く。実は、$n$ 次多項式 $H_n(x)$ で、$f^{(n)} = (-1)^nH_n(x)e^{-x^2}$ となるものがある。この、$H_n(x)$ を $n$ 次エルミート(Hermite) 多項式という。$H_0(x) = 1$、$H_1(x) = 2x$、$H_2(x) = 4x^2 -2$。
\end{eg}

\begin{ex}
$H_n(x)$ を 上の例で定義された $n$ 次エルミート(Hermite) 多項式とするとき以下を示せ。
\begin{enumerate}
\item $H_n(x)$ は実際 $n$ 次多項式である。
\item $H_{n+1}(x) = 2xH_n(x) - 2nH_{n-1}(x)$。
\item $H'_n(x) = 2nH_{n-1}(x)$。
\item $y = H_n(x)$ とおくと、$y$ は、次の微分方程式を満たす。
$$y'' - 2xy' + 2ny = 0.$$
\end{enumerate}
\end{ex}

\medskip
$B$. 逆関数の微分

写像 $f: X \to Y$ は、 $f(X) = Y$ すなわち、$Y$ のどんな元(要素)$y$ に対しても、$f(x) = y$ となる $x\in X$ が存在するとき、全射であるという。また、
$$x_1\neq x_2 \to f(x_1)\neq f(x_2) \Leftrightarrow f(x_1) = f(x_2) \to x_1 = x_2$$
が成り立つとき単射であるという。全射かつ単射であるとき、全単射又は、1対1対応であるという。このときは、$y\in Y$ に対して、$f(x) = y$ となる $x\in X$ がただ一つに定まるので、 $y$ にこの $x$ を対応させる写像を $f$ の逆写像という。定義から、$g(f(x)) = x$、$f(g(y)) = y$ が成り立つ。逆に、$f(x)$ および、$g(y)$ が、ある区間で互いに逆写像 $g(f(x)) = x, \;f(g(y)) = y$ という条件を満たしていると、$f(x)$、$g(y)$ は、全単射になっている。


\begin{eg}
\begin{enumerate}
\item $y = f(x) = 2x -1$ は、$\bR$ から、$\bR$ への全単射である。これの逆写像を $x = g(y)$ で表すと、$g(y) = (1/2)(y+1)$ となる。逆写像がある、すなわち、全単射であるとは、簡単に言うと、各 $y$ に対して、$f(x) = y$ となる、$x$ が、いつでもただ一つに定まるということである。
\item $y = f(x) = x^2$ は、$\bR$ から、$\bR$ への写像(数の集合から数の集合への写像を関数とも言う)ではあるが、これは、全射でも単射でもない。まず、値域を 負でない数全体、$\bR^+ = \{y\geq 0\mid y\in \bR\}$ とすると、$f:\bR \to \bR^+$ は、全射である。しかし、$f(-1) = f(1)$ からも分かるように 単射ではない。これから全単射を得るためには、この、定義域を制限して、例えば、$f:\bR^+\to\bR^+$ に制限すれば、これは、全単射である。この逆写像は、$g(y) = \sqrt{y}$。値域を 正でない数全体 $\bR^-$ に制限したときは、逆写像 $h(y)$ は、$-\sqrt{y}$ であることに注意。
\end{enumerate}
\end{eg}

\begin{prop}
$f(s)$ および、$g(t)$ が、ある区間で互いに逆関数 $(g(f(s)) = s, \;f(g(t)) = t)$ ならば、
$$\frac{dg}{dt} = \frac{1}{(\frac{df}{ds})}.$$
\end{prop}
\proof
$f(g(t)) = t$ を合成関数の微分を用いて、両辺微分すると、$f'(g(t))g'(t) = 1$。ただし、$f'(g(t))$ は、$s$ の関数としての微分である。
\qed

\begin{eg}
\begin{enumerate}
\item $\sin x$ の逆関数の微分。
$$y = f(x) = \sin^{-1} x = \arcsin x, \; (-\pi/2 \leq y \leq \pi/2).$$
$x = \sin y$ だから、
$$\frac{dy}{dx} = \frac{1}{\cos y} = \frac{1}{\sqrt{1-x^2}}.$$
ここで、$y$ の範囲を $-\pi/2 \leq y \leq \pi/2$ の様に設定したため、$\cos y\geq 0$ であることに注意。
\item $\cos x$ の逆関数の微分。
$$y = f(x) = \cos^{-1} x = \arccos x, \; (0 \leq y \leq \pi).$$
$x = \cos y$ だから、
$$\frac{dy}{dx} = \frac{-1}{\sin y} = \frac{-1}{\sqrt{1-x^2}}.$$
\item $\tan x$ の逆関数の微分。
$$y = f(x) = \tan^{-1} x = \arctan x, \; (-\pi/2 < y < \pi/2).$$
$x = \tan y$ だから、
$$\frac{dy}{dx} = \frac{1}{\sec^2 y} = \frac{1}{1+x^2}.$$
\item $e^x$ の逆関数 $y = \log x$ の微分。
$x = e^x$ だから、
$$\frac{dy}{dx} = \frac{1}{e^y} = \frac{1}{x}.$$
\end{enumerate}
\end{eg}

\medskip
$C$. 対数微分法

積、商が多く含まれる関数、指数関数には、まず、$\log$ を取ってから、微分をする対数微分法が有効である。$y = f(x)$ のとき、$\log y = \log f(x)$ を微分すると、
$$\frac{d}{dx}\log f(x) = \frac{f'(x)}{f(x)}.$$
であった。

\begin{eg}
\begin{enumerate}
\item ${\displaystyle y = \frac{(2x+1)^4}{(x^2-1)^3(4x+3)^2}}$ とすると、
$$\log y = 4\log(2x+1) - 3\log(x^2-1) - 2\log(4x+3)$$
となるから、これより、次を得る。
$$\frac{y'}{y} = \frac{4\cdot2}{2x+1} - \frac{3\cdot2x}{x^2 -1} - \frac{2\cdot4}{4x+3}.$$
\item $y = (1+2x)^{1/x}$ とすると、$\log y = \frac{1}{x}\log(1+2x)$。これより、
$$\frac{y'}{y} = -\frac{1}{x^2}\log(1+2x) + \frac1x\cdot\frac{2}{2x+1}.$$
\item $u = u(x), v = v(x), w = w(x)$ とすると、$(uvw)' = u'vw + uv'w + uvw'$ だが、対数微分法を使うことにより、$u(x)v(x)w(x) \neq 0$ の範囲で、
$$\frac{(uvw)'}{uvw} = \frac{u'}{u} + \frac{v'}{v} + \frac{w'}{w}$$
であることが分かる。この様な表示が有効なときもよくある。
\end{enumerate}
\end{eg}

\medskip
$D$. ライプニッツ(Leibniz)の公式
$$(f\cdot g)^{(n)} = \sum_{k=0}^n {}_nC_k f^{(n-k)}\cdot g^{(k)}.$$
ただし、$f^{(0)} = f$。${}_nC_k$ は、${n \choose k}$ と表すことが多い。

\begin{eg}
\begin{enumerate}
\item ${\displaystyle (\sin x)^{(k)} = \sin\bigl(x+\frac{k\pi}{2}\bigr)}$ に注意する。すると、
\begin{eqnarray*}
(e^x\sin x)^{(n)} & =  & \sum_{k=0}^n {n\choose k}e^x(\sin x)^{(k)}\\
& = & \sum_{k=0}^n {n\choose k}e^x\sin\bigl(x+\frac{k\pi}{2}\bigr)
\end{eqnarray*}
\item $f(x) = \tan^{-1}x$ とする。このとき、逆関数の微分で考えたように、$f'(x) = 1/(1+x^2)$ であった。これより、$1 = (1+x^2)f'(x)$ を得る。これに、ライプニッツの公式を適応すると、$1+x^2$ は、3回微分すると $0$ となることから、$k\geq 1$ のとき、
\begin{eqnarray*}
0 & = & \sum_{i = 0}^k {k\choose i} (1+x^2)^{(i)}(f'(x))^{(k-i)}\\
   & = & \sum_{i = 0}^2 {k\choose i} (1+x^2)^{(i)}(f^{(k-i+1)}(x))\\
   & = & (1+x^2)f^{(k+1)}(x) + k2xf^{(k)}(x) + \frac{k(k-1)}2\cdot 2f^{(k-1)}(x)\\
   & = &  (1+x^2)f^{(k+1)}(x) + 2kxf^{(k)}(x) + {k(k-1)}f^{(k-1)}(x)
\end{eqnarray*}
を得る。例えば、$x = 0$ とおくと、$k\geq 0$ で、
$$f^{(k+1)}(0) + k(k-1)f^{(k-1)}(0) = 0$$
を得るから、$f^{(0)}(0) = f(0) = 0$、$f^{(1)}(0) = 1$ に注意すると、
$$f^{(0)}(0) = f^{(2)}(0) = f^{(4)}(0) = \cdots = f^{(2i)}(0) = 0, \ldots$$
また、奇数の時は、
$$f^{(1)}(0) = 1, f^{(3)}(0) = -2, f^{(5)}(0) = 3! , \ldots, f^{(2i-1)}(0) = (-1)^{i-1}i!, \ldots$$
を得る。この様な使い方もある。
\end{enumerate}
\end{eg}

\newpage
\mysection{平均値の定理}
\begin{prop}(Roll の定理)\quad \label{prop:roll}
関数 $f(x)$ が、閉区間 $[a,b]$ で、連続、開区間 $(a,b)$ で、微分可能であり、$f(a) = f(b)$ ならば、開区間 $(a,b)$ の点 $c$ で、$f'(c) = 0$ を満たすものがある。
\end{prop}
\proof
$f(x) = f(a) = f(b)$ が常に成り立てば、$f'(x) = 0$ だから、この場合は、良い。
$f(x) > f(a) = f(b)$ となる点、$x$ があるとする。$f(x)$ は、連続だから、命題 \ref{prop:maxmin} によって、区間内のある点、$c$ で、最大値 $f(c)$ をとる。このとき、
$$f'(c) = \lim_{x\rightarrow c}\frac{f(x) - f(c)}{x-c} $$
かつ、
$$\frac{f(x) - f(c)}{x-c} = \left\{\begin{array}{ll}\mbox{負又は0} & x< c \\
\mbox{正又は0} & x>c \end{array} \right.$$
従って、$f'(c) = 0$。

$f(x) < f(a) = f(b)$ となる点、$x$ がある場合も同様。
\qed

\begin{thm} (平均値の定理)\quad  \label{thm:mvt}
関数 $f(x)$ が、閉区間 $[a,b]$ で、連続、開区間 $(a,b)$ で、微分可能ならば、開区間 $(a,b)$ の点 $c$ で、次を満たすものがある。
$$f'(c) = \frac{f(b) - f(a)}{b-a}.$$
\end{thm}
\proof
$F(x) = f(x) - {\displaystyle \frac{f(b) - f(a)}{b-a}}x$ とすると、
$$F(a) = f(a) - \frac{f(b) - f(a)}{b-a}a = \frac{bf(a) - af(b)}{b-a} = 
f(b) - \frac{f(b) - f(a)}{b-a}b = F(b)$$
従って、命題 \ref{prop:roll} によって、$F'(c) = 0$  となる、点 $c$ がある。これは、
$$f'(c) = \frac{f(b) - f(a)}{b-a}$$
を意味する。
\qed

\newpage\mysection{関数とグラフ}
\begin{definition}
点 $x$ が、点 $a$ に十分近いときは、常に、
$f(a) > f(x)$ が成り立つとき、$f(x)$ は、$x = a$ で、極大になるといい、$f(a)$ をその極大値、点 $a$ を、極大点という。同様にして、点 $x$ が、点 $a$ に十分近いときは、常に、
$f(a) < f(x)$ が成り立つとき、$f(x)$ は、$x = a$ で、極小になるといい、$f(a)$ をその極小値、点 $a$ を、極小点という。極大値と極小値を合わせて極値という。
\end{definition}

\begin{prop} \label{prop:1stderiv-test}
$f(x)$ を、閉区間 $[a,b]$ 上連続、開区間 $(a,b)$  上微分可能とする。このとき次が成立する。
\begin{itemize}
\item[$(1)$] $(a,b)$ 内の点 $c$ で極値を持てば、$f'(c) = 0$。
\item[$(2)$] $(a,b)$ 上で、常に $f'(x)>0$ ならば、$f(x)$  は、$[a,b]$ 上で真に増加。
\item[$(3)$] $(a,b)$ 上で、常に $f'(x)<0$ ならば、$f(x)$  は、$[a,b]$ 上で真に減少。
\item[$(4)$] $(a,b)$ 上で、常に $f'(x)=0$ ならば、$f(x)$  は、$[a,b]$ 上で定数関数。
\end{itemize}
\end{prop}
\proof
平均値の定理 \ref{thm:mvt} によって、
$$\frac{f(x) - f(c)}{x-c} = f'(d) \;\mbox{ となる点 } d \mbox{ が、$x$ と $c$ の間にある。}$$
\begin{itemize}
\item[$(1)$] $f(c) > f(x)$ の時は、上の値は、$x < c$ のとき、正、$x>c$ のとき、負だから、$x$ が、$c$ に近づくと、つまり、$f'(c)$ は、$0$ にならざるを得ない。
$f(c) < f(x)$ のときも同様。
\item[$(2)$] $f(x) - f(c)$ は、$x<c$ のとき、負、$x>c$ のとき、正だから、真に増加。
\item[$(3)$] 同様。
\item[$(4)$] 明らか。\qed
\end{itemize}

$f'(x)$ の増加、減少は、$f''(x)$ によって分かることを考えれば、次を得る。
\begin{prop}
$f(x)$ を、閉区間 $[a,b]$ 上連続、開区間 $(a,b)$  上2回微分可能とする。このとき次が成立する。
\begin{itemize}
\item[$(1)$] $f'(c) = 0$、$f''(c) < 0$ ならば、関数 $f(x)$ は、$c$ で極大値 $f(c)$ を持つ。
\item[$(2)$] $f'(c) = 0$、$f''(c) > 0$ ならば、関数 $f(x)$ は、$c$ で極小値 $f(c)$ を持つ。
\end{itemize}
\end{prop}

\begin{definition}
関数 $f(x)$ が、点 $c$ において、接線を持ち、$c$ のごく近くで、$f(x)$ のグラフが、接線の上にあれば、$f(x)$ は、点 $c$ において、下に凸、接線の下にあれば、上に凸という。$f(x)$ のグラフが、接線の上から下、又は、下から上に移るとき、この点を、変曲点という。
\end{definition}

上の定義の条件を式で書いてみる。点 $c$ での接線の方程式は、$y = f(c) + f'(c) (x-c)$ だから、$f(x)\geq f(c) + f'(c)(x-c)$ となる。$a, b$ を $a < c < b$  である、$c$ のごく近くの点とすると、
$$\frac{f(c) - f(a)}{c-a} \leq f'(c) \leq \frac{f(b) - f(c)}{b-c}$$
だから、$f'(x)$ が 微分可能とすると、$c$ の近くで、$f'(x)$ が増加していることが分かる。逆に、$f'(x)$ が $c$ の近くで、増加していれば、上の式から、$c$ のごく近くで、$f(x)$ のグラフが、接線の上にあることが分かるから、関数 $f(x)$ が、点 $c$ で、下に凸であるための必要十分条件は、$f(x)$ が、$c$ の近くで、増加していることであることが分かる。(ただし、$f''(c)>0$ を意味しているわけではない。)このことより、次が得られる。

\begin{prop}
$f(x)$ が、開区間 $(a,b)$ において、2階導関数 $f''(x)$ を持てば、次が成立。
\begin{itemize}
\item[$(1)$]  $f''(x)>0$ ならば、$f(x)$ は、開区間 $(a,b)$ で、下に凸。
\item[$(2)$]  $f''(x)<0$ ならば、$f(x)$ は、開区間 $(a,b)$ で、上に凸。
\item[$(3)$]  $f''(c) = 0$ かつ、$f''(x)$ の符号(正であるか、負であるか)が点 $c$ で変われば、$x = c$ は、変曲点。
\end{itemize}
\end{prop}

\begin{eg}
区間 $[0,2\pi]$ 上で、関数 $f(x) = \sin x(1 + \cos x)$ 考える。
\begin{eqnarray*}
f'(x) & = & \cos x(1+ \cos x) + \sin x(-\sin x)\\
        & = & \cos x + \cos^2 x - \sin^2 x\\
        & = & 2\cos^2 x + \cos x - 1\\
        & = & (2\cos x -1)(\cos x+ 1)
\end{eqnarray*}
従って、 $f'(x) = 0$ ならば、$\cos x = -1$ または、$1/2$。与えられた区間の中では、$x = \pi/3, \pi, 5\pi/3$。また、
$$f''(x) = -2\sin x(\cos x +1) - \sin x(2\cos x -1) = -\sin x(4\cos x +1)$$
より、$f''(x) = 0$ となるのは、$x = 0, \pi, 2\pi$ および、$\cos x$ が、$-1/4$ となるところ。$x_1 = \arccos (-1/4)$ とすると、そのような点は、$x_1$ と、$2\pi - x_1$。
$$\begin{array}{c|ccccccccccccc}
x & 0 &  & \pi/3 & & x_1 & & \pi & & 2\pi-x_1 & & 5\pi/3 & & 2\pi\\
\hline 
f(x) & & \nearrow & \mbox{凸} & \searrow &  \searrow & \searrow & & \searrow & \searrow & \searrow & \mbox{凹} & \nearrow & \\
f'(x) & + & + & 0 & - & - & - & 0 & - & - & - & 0 & + & + \\
f''(x) & 0 & - & - & - & 0 & + & 0 & - & 0 & + & + & + & 0 
\end{array}$$
\end{eg}

\begin{eg}
$p>1$ $\frac{1}{p} + \frac{1}{q} = 1$ のとき、任意の $\alpha$, $\beta$ に対して、
$$\frac{1}{p}|\alpha|^p + \frac{1}{q}|\beta|^q \geq |\alpha||\beta|.$$
\proof
$\beta$ を固定して、
$$f(x) = \frac{1}{p}x^p + \frac{1}{q}|\beta|^q - |\beta|x\quad(x\geq 0)$$
とおく。$f(x)\geq 0$ であることを示せばよい。$f'(x) = x^{p-1} - |\beta|$ だから、$x\geq 0$  において、次のようになる。
$$\begin{array}{c|cccc}
x & 0 &  & |\beta|^{1/(p-1)} & \\
\hline 
f'(x) & & - & 0 & + \\
f(x) & & \searrow & \mbox{極小} & \nearrow 
\end{array}$$
したがって、$f(x)\geq f(|\beta|^{1/(p-1)})$ となる。ところが、
$$f(|\beta|^{1/(p-1)}) = \frac{1}{p}|\beta|^{p/(p-1)} + \frac{1}{q}|\beta|^{q} - |\beta|^{p/(p-1)} = 0.$$
これより、$f(x)\geq 0$ を得る。
\qed
\end{eg}

\newpage\mysection{微分の応用、テイラー展開}
\begin{thm} (Cauchy の平均値の定理)\label{thm:cauchy-mvt}
閉区間 $[a,b]$ において、$f(x)$、$g(x)$ は、連続、開区間 $(a,b)$ において、微分可能とする。もし、$g(a)\neq g(b)$  かつ、$f'(t)$、$g'(t)$  が、同時に $0$ にならなければ、$(a,b)$ のある点、$c$ において、 次が成り立つ。
$$\frac{f(b) - f(a)}{g(b) - g(a)} = \frac{f'(c)}{g'(c)}$$
\end{thm}
\proof
$F(x) = (g(b) - g(a))f(x) - (f(b) - f(a))g(x)$  とおくと、$F(a) = F(b)$ だから、Roll の定理が、使える。
\qed

\note
上記の定理により、次の、右辺の極限があれば、左辺の極限も存在し、等しくなる。
$$\lim_{x\rightarrow a}\frac{f(x) - f(a)}{g(x) - g(a)} = 
\lim_{c\rightarrow a}\frac{f'(c)}{g'(c)}$$
特に、$f(a) = g(a) = 0$ のときなどに利用でき、
$$\lim_{x\rightarrow a}\frac{f(x)}{g(x)} = 
\lim_{x\rightarrow a}\frac{f'(x)}{g'(x)}$$
が得られる。
証明を、少し工夫すると、$\lim_{x\rightarrow a} f(x) = \pm\infty$、
$\lim_{x\rightarrow a} g(x) = \pm\infty$ の場合も、同様のことが言える。
この様に、極限が、$0/0$ 又は、$\pm\infty/\pm\infty$ となるものを{\gt 不定形}という。

\smallskip
以上をまとめると次のようになる。

\begin{thm} [L'Hospital の定理]
\begin{itemize}
\item[$(1)$] 関数 $f(x)$、$g(x)$ が、$(a,b)$ で微分可能で $g'(x) \neq 0$ とし、
$$\lim_{x\to a}f(x) = 0,\;\lim_{x\to a}g(x) = 0$$
とする。このとき、$\lim_{x\to a}\frac{f'(x)}{g'(x)}$ が存在すれば($\pm\infty$ も含めて)、
$$\lim_{x\to a}\frac{f(x)}{g(x)} = \lim_{x\to a}\frac{f'(x)}{g'(x)}.$$
\item[$(2)$] 関数 $f(x)$、$g(x)$ が、$(a,b)$ で微分可能で $g'(x)$ は、定符号とし、
$$\lim_{x\to a}f(x) = \pm\infty,\;\lim_{x\to a}g(x) = \pm\infty$$
とする。このとき、$\lim_{x\to a}\frac{f'(x)}{g'(x)}$ が存在すれば($\pm\infty$ も含めて)、
$$\lim_{x\to a}\frac{f(x)}{g(x)} = \lim_{x\to a}\frac{f'(x)}{g'(x)}.$$
\end{itemize}
\end{thm}

\begin{eg}
\begin{enumerate}
\item ${\displaystyle \lim_{x\rightarrow 0}\frac{x - \sin x}{x^3} = 
\lim_{x\rightarrow 0}\frac{1-\cos x}{3x^2} = 
\lim_{x\rightarrow 0}\frac{\sin x}{6x} = 
\lim_{x\rightarrow 0}\frac{\cos x}{6} = \frac16}$
\item ${\displaystyle \lim_{x\to\infty}x^ne^{-x} = \lim_{x\to \infty}\frac{x^n}{e^{x}} = \lim_{x\to\infty}\frac{nx^{n-1}}{e^x} = \cdots = \lim_{x\to\infty}\frac{n!}{e^x} = 0}$
\item ${\displaystyle \lim_{x\to +0}x\log x = \lim_{x\to +0}\frac{\log x}{1/x} = \lim_{x\to +0}\frac{1/x}{-1/x^2} = \lim_{x\to +0}(-x) = 0}$.
\item ${\displaystyle \lim_{x\to +0}x^x = \lim_{x\to +0}e^{x\log x} = e^0 = 1}$.
\end{enumerate}
\end{eg}

平均値の定理は、次のようにも記述できる。
$$f(x) = f(a) + f'(c)(x-a)$$
ただし、$c$ は、$x$ と、$a$ の間の点。$f(x)$  が、高階導関数を持つときは、この表示を一般化出来る。

\begin{prop}
関数 $f(x)$ が、$[a,b]$ 上で連続、$(a,b)$ 上で $n$ 回微分可能ならば、区間
$(a,b)$ 内の点 $c$ で、次の条件を満たすものが存在する。
$$f(b) = f(a) + \frac{f'(a)}{1!}(b-a) + \frac{f''(a)}{2!}(b-a)^2 + \cdots + \frac{f^{(n-1)}(a)}{(n-1)!}(b-a)^{n-1} + R_n$$
$$R_n = \frac{f^{(n)}(c)}{n!}(b-a)^n$$
特に、$|f^{(n)}(x)|\leq M$ が、区間内で成り立てば、
$$|R_n| \leq \frac{M}{n!}|b-a|^n$$
このとき、
$$f(x) = f(a) + \frac{f'(a)}{1!}(x-a) + \frac{f''(a)}{2!}(x-a)^2 + \cdots + \frac{f^{(n-1)}(a)}{(n-1)!}(x-a)^{n-1} + \frac{f^{(n)}(c)}{n!}(x-a)^n$$
の様に表すことを、$f(x)$ の $x = a$ を中心とする Taylor 展開という。特に、$a = 0$ のときは、Maclaughrin 展開という。$c = a + \theta(x-a)$ $0<\theta < 1$ と表すこともある。
\end{prop}
\proof
$F(x)$ を次のようにして定義する。
\begin{eqnarray*}
F(x) & =  & -f(b) + f(x) + f'(x)(b-x) + \frac{1}{2!}f''(x)(b-x)^2 + \cdots + \\
  & & \frac{1}{(n-1)!}f^{(n-1)}(x)(b-x)^{n-1} + K(b-x)^n\\
K & = &  \frac{1}{(b-a)^n}\bigl(f(b) - \bigl(f(a) + f'(a)(b-a) + \frac{1}{2!}f''(x)(b-a)^2 + \cdots +\\
& & \frac{1}{(n-1)!}f^{(n-1)}(a)(b-a)^{n-1}\bigr)\bigr)
\end{eqnarray*}
とおくと、$F(a) = F(b) = 0$ より、Roll の定理が使える。
\qed

\begin{eg}
\begin{enumerate}
\item $f(x) = \sin x$ のマクローリン展開
$$f(0) = 0, \;f'(0) = \cos 0 = 1,\; f''(0) = -\sin 0 = 0,\;f'''(0) = -\cos 0 = -1$$
だから、以下のようになっている。
$$f^{(n)}(0) = \left\{\begin{array}{ll}0 & n = 2m\\ (-1)^m & n = 2m+1 \end{array}\right. \mbox{ 次のようにも書ける} f^{(n)}(x) = \sin\Bigl(x + \frac{n\pi}{2}\Bigr).$$
従って、
$$\sin x = x - \frac{x^3}{3!} + \frac{x^5}{5!} - \frac{x^7}{7!} + \cdots + (-1)^m\frac{x^{2m+1}}{(2m+1)!} + \cdots .$$
例えば、これを用いて、近似式も得られる。$f^{(5)}(x) = \cos x$ であることに注意すると、
$$\sin x = x - \frac{x^3}{3!} + \frac{f^{(5)}(\theta x)x^5}{5!}, \;(0<\theta<1)$$
と書けるから、例えば、$-\pi/2\leq x\leq \pi/2$ の範囲内では、
$$\left|\sin x - \left(x - \frac{x^3}{3!}\right)\right| = \left|\frac{\cos(\theta x)x^5}{5!}\right| \leq \frac{|x|^5}{5!} \leq \frac{\pi^5}{120\cdot 32}\sim 0.08.$$
$-\pi/4\leq x\leq \pi/4$ の範囲内では、上の値は、$0.0025$ になる。この意味で、特に、$0$ の近くでは、$\sin x$  は、$x - \frac{x^3}{3!}$ に非常に近い。これはまた、
$$\lim_{x\rightarrow 0}\frac{x - \sin x}{x^3} = \lim_{x\rightarrow 0}\left(\frac{x^3}{6} - \frac{x^5}{5!}\cos(\theta x)\right)\Big/x^3 = \lim_{x\rightarrow 0}\left(\frac16 -\frac{x^2}{5!}\cos(\theta x)\right) = \frac{1}{6}.$$
このことから、$O(f(x))$、$o(f(x))$ を次のように定義して、使うこともある。
$$\lim_{x\rightarrow 0}\frac{O(f(x))}{f(x)} = C \mbox{ a constant, } 
 \lim_{x\rightarrow 0}\frac{o(f(x))}{f(x)} = 0$$
 上の場合は、例えば、
$$\sin x = x - \frac{x^3}{3!} + O(x^5)$$
\item $f(x) = e^x$ のマクローリン展開。
$$e^x = 1 + x + \frac{x^2}{2!} + \frac{x^3}{3!} + \cdots + \frac{x^n}{n!} + e^{\theta x}\frac{x^{n+1}}{(n+1)!}$$
\item $f(x) = \cos x$ のマクローリン展開。
$$\cos x = 1 - \frac{x^2}{2!} + \frac{x^4}{4!} + \cdots + (-1)^n\frac{x^{2n}}{(2n)!} + (-1)^{n+1}\cos(\theta x)\frac{x^{2n+2}}{(2n+2)!}$$
\item $f(x) = \log(1+x)$ のマクローリン展開。
$$\log(1+x) = x - \frac{x^2}{2} + \frac{x^3}{3} + \cdots + (-1)^{n-1}\frac{x^n}{n} + (-1)^n\frac{x^{n+1}}{n+1}\left(\frac{1}{1+\theta x}\right)^{n+1}$$
\item $f(x) = (1+x)^\alpha$ のマクローリン展開 ($\alpha$ は、任意の実数)。
$$(1+x)^\alpha = 1 + \alpha x + \frac{\alpha(\alpha-1)x^2}{2!} + \cdots + \frac{\alpha(\alpha-1)\cdots (\alpha - n + 1)x^n}{n!} + R_{n+1},$$
$$\mbox{where }R_{n+1} = \frac{\alpha(\alpha-1)\cdots(\alpha-n)(1+\theta x)^{\alpha -n - 1}x^{n+1}}{(n+1)!}$$
$\alpha$ が、整数ならば、$n = \alpha$ のとき、最後の項が、$0$ となり、2項定理を得る。
\item $f(x) = \arctan x$ のマクローリン展開。まず、$f'(x) = 1/(1+x^2)$ だから、$(1+x^2)f'(x) = 1$。ライプニッツの公式を使うと、
$$(1+x^2)f^{(n+1)}(x) + n2xf^{(n)}(x) + \frac{n(n-1)}{2}2f^{(n-1)}(x) = 0$$
$x = 0$ と置くと、$f^{(n+1)}(0) + n(n-1)f^{(n-1)}(0) = 0$, $f'(0) = 1$, $f''(0) = 0$ だから、$f^{(2m)}(0) = 0$ かつ、$f^{(2m+1)}(0) = (-1)^m(2m)!$。従って、
$$\arctan x = x - \frac{x^3}{3} + \frac{x^5}{5} +\cdots + (-1)^n\frac{x^{2n+1}}{2n+1} + \cdots.$$

これから、$\arctan 1 = \pi/4$ であることを用いると、$\pi$ が、計算できる。
\end{enumerate}
\end{eg}

平均値の定理や、Taylor の定理は、関数のグラフの問題にも様々に応用される。例えば、$f(x)$ を2回微分可能関数とすると、
$$f(x) - f(a) - f'(a)(x - a) = \frac{1}{2}f''(c)(x-a)^2$$
となる、$x$ と、$a$ の間の点 $c$ が存在する。もし、$f''(c) > 0$  とすると、
$$f(x) > f(a) + f'(a)(x-a)$$
だから、$f(x)$ のグラフは、$a$ での接線のグラフより、上にある。


\newpage\mysection{不定積分}
\begin{definition}
関数 $F(x)$ の導関数が、$f(x)$ に等しいとき、すなわち、$F'(x) = f(x)$ が成り立つとき、$F(x)$ を、$f(x)$ の原始関数と言う。
\end{definition}

$F(x)$、$G(x)$ を共に、$f(x)$ の原始関数とする。すると、$F'(x) = G'(x) = f(x)$ であるから、$(F(x) - G(x))' = 0$  となる。導関数が、常に、$0$ となる関数は、命題 \ref{prop:1stderiv-test} (4) によって、定数となる。従って、$G(x) = F(x) + C$ なる定数 $C$ が存在する。逆に、$G(x) = F(x) + C$ と表せる関数は、$f(x)$ の原始関数である。このことをふまえ、原始関数の代表という意味で、$f(x)$ の不定積分と呼び、次のように書く。
$$\int f(x)dx = F(x) + C$$
ここで、$C$ を積分定数と言う。ただし、$C$ を省略して書くことも良くある。

\begin{eg}
\begin{enumerate}
\item ${\displaystyle \int \sin x dx = -\cos x + C}$
\item ${\displaystyle \int \cos x dx = \sin x + C}$
\item ${\displaystyle \int \frac{1}{\cos^2 x}dx = \int \sec^2 xdx = \tan x + C}$
\item ${\displaystyle \int \frac{1}{\sin^2 x}dx = \int \csc^2 xdx = -\cot x + C}$
\item ${\displaystyle \int x^\alpha dx = \frac{1}{\alpha+1}x^{\alpha+1} + C, \;(\mbox{ if }\alpha\neq -1)}$
\item ${\displaystyle \int \frac1x dx = \log|x| + C}$
\item ${\displaystyle \int e^x dx = e^x + C}$\\
${\displaystyle \int a^x dx = \frac{a^x}{\log a} + C,\;(a>0, a\neq 1)}$
\item ${\displaystyle \int \frac{1}{\sqrt{1-x^2}} dx = \arcsin x + C = \sin^{-1}x + C}$
\item ${\displaystyle \int \frac{1}{1+x^2} dx = \arctan x + C = \tan^{-1}x + C}$
\item ${\displaystyle \int \frac{1}{\sqrt{x^2+A}} dx = \log|x+\sqrt{x^2+A}| + C}$
\end{enumerate}
\end{eg}

\begin{eg}
${\displaystyle \frac{d}{dx}\arccos x = \frac{d}{dx}\cos^{-1}x = \frac{-1}{\sqrt{1-x^2}}}$ であったから、$\arcsin x + \arccos x$ は、定数のはずである。この定数を決定せよ。\hfill ($\frac{\pi}2$)
\end{eg}

\newpage\mysection{不定積分の計算}
A. {\gt 置換積分}\quad $F'(x) = f(x)$ であるとすると、
$$\frac{d}{dt} F(\phi(t)) = f(\phi(t))\phi'(t)$$
であるから、$x = \phi(t)$ とおいたときは、
$$\int f(x)dx = \int f(\phi(t))\phi'(t)dt$$
となる。

\medskip
B. {\gt 部分積分}\quad $(fg)' = f'g + fg'$  だから、両辺を積分すると、
$$\int f(x)g'(x)dx = f(x)g(x) - \int f'(x)g(x)dx$$
を得る。

\begin{eg}
\begin{enumerate}
\item ${\displaystyle \int \phi'(t)\sin(\phi(t)) dt = -\cos(\phi(t)) + C}$
\item ${\displaystyle \int \phi'(t)\cos(\phi(t)) dt = \sin(\phi(t)) + C}$
\item ${\displaystyle \int \phi'(t)(\phi(t))^\alpha dt = \frac{1}{\alpha+1}(\phi(t))^{\alpha+1} + C, \;(\mbox{ if }\alpha\neq -1)}$
	\begin{enumerate}
	\item ${\displaystyle \int (5x+2)^{10} dx = \frac{1}{55}(5x+2)^{11} + C}$
	\item ${\displaystyle \int \frac{1}{\sqrt{x-2}} dx = 2\sqrt{x-2}+ C}$
	\item ${\displaystyle \int \frac{x^2}{(x^3+1)^5} dx = \frac{-1}{12(x^3+1)^4} + C}$
	\item ${\displaystyle \int \sin^{10}x\cos x dx = \frac{1}{11}\sin^{11} x+ C}$
	\item ${\displaystyle \int \frac{(\log x)^{10}}x dx = \frac{1}{11}(\log x)^{11}+ C}$
	\item ${\displaystyle \int \frac{(\arctan x)^{10}}{1+x^2}dx = \frac{1}{11}(\arctan x)^{11}+ C}$
	\end{enumerate}
\item ${\displaystyle \int \frac{\phi'(t)}{\phi(t)} dt = \log|\phi(t)| + C}$
	\begin{enumerate}
	\item ${\displaystyle \int \frac{1}{5x-3} dx = \frac{1}{5}\log|5x-3|+ C}$
	\item ${\displaystyle \int \frac{x^2}{x^3+1} dx = \frac{1}{3}\log|x^3+1|+ C}$
	\item ${\displaystyle \int \tan x dx = -\int \frac{-\sin x}{\cos x}dx = -\log|\cos x| + C}$
	\item ${\displaystyle \int \frac{2}{x^2- x - 6}dx  = \frac25\int\left(\frac{1}{x-3} - \frac{1}{x+2}\right)dx = \frac25\log\left|\frac{x-3}{x+2}\right| + C}$
	\item ${\displaystyle \int \frac{x}{x^2- x - 6}dx  = \frac15\int\left(\frac{3}{x-3} + \frac{2}{x+2}\right)dx}$\\
	$ = \frac15(3\log|x-3| + 2\log|x+2|)+ C$
	\end{enumerate}
\item ${\displaystyle \int \phi'(t)e^{\phi(t)} dt = e^{\phi(t)} + C}$
\item ${\displaystyle \int \frac{\phi'(t)}{\sqrt{1-(\phi(t))^2}} dt= \arcsin(\phi(t)) + C = \sin^{-1}(\phi(t)) + C}$
\item ${\displaystyle \int \frac{\phi'(t)}{1+(\phi(t))^2} dt = \arctan (\phi(t))+ C = \tan^{-1}(\phi(t)) + C}$
\item ${\displaystyle \int \frac{\phi'(t)}{\sqrt{(\phi(t))^2+A}} dt = \log|\phi(t)+\sqrt{(\phi(t))^2+A}| + C}$
\end{enumerate}
\end{eg}

\newpage\mysection{有理関数の積分}
{\gt 復習 }
\begin{enumerate}
\item ${\displaystyle \int \frac{dx}{(x-a)^n} = \frac{1}{(-n+1)(x-a)^{n-1}}, \; (n\neq 1)}$\\
${\displaystyle \int \frac{dx}{x-a} = \log|x-a| + C}$
\item ${\displaystyle \int \frac{2x}{(x^2+a^2)^n} dx = \frac{1}{(-n+1)(x^2+a^2)^{n-1}},\;(n\neq 1)}$
\item ${\displaystyle \int \frac{2x}{x^2+a^2} dx = \log(x^2+a^2)}$
\item $a\neq 0$, $n\geq 1$ の時、${\displaystyle I_n = \int\frac{dx}{(x^2+a^2)^n}}$ とおくと、
$$ a^2I_n(x) = \frac{1}{2n-2}\frac{x}{(x^2+a^2)^{n-1}} + \frac{2n-3}{2n-2}I_{n-1}(x),\;(n\geq 2)$$
$$ I_1(x) = \frac{1}{a}\arctan\frac{x}{a}$$
\proof
両辺を微分すると、
\begin{eqnarray*}
LHS & = & \frac{a^2}{(x^2+a^2)^{n}}\\
RHS & = & \frac{1}{(2n-2)(x^2+a^2)^{n-1}} 
+\frac{-(n-1)2x^2}{2(n-1)(x^2+a^2)^{n}} + \frac{(2n-3)}{(2n-2)(x^2+a^2)^{n-1}}\\
& = & \frac{1}{(x^2+a^2)^n}\left(\frac{1}{2n-2}(x^2+a^2) -x^2 + \frac{2n-3}{2n-2}(x^2+a^2)\right)
\\
& = & \frac{a^2}{(x^2+a^2)^{n}}
\end{eqnarray*}
従って、両辺の差は、定数。すなわち、$I_n$ に定数部分の差異を含め、上記の等式が成り立つ。
または、部分積分を用いることも出来る。
\begin{eqnarray*}
I_{n-1} & = & \int\frac{1dx}{(x^2+a^2)^{n-1}} \\
	& = & \frac{x}{(x^2+a^2)^{n-1}} - \int \frac{x\cdot (-(n-1)2)
	x}{(x^2+a^2)^{n}}dx\\
	& = & \frac{x}{(x^2+a^2)^{n-1}} + 2(n-1)\int\frac{x^2+a^2}{(x^2+a^2)^{n}}dx -
	2(n-1)\int\frac{a^2}{(x^2+a^2)^{n}}dx\\
	& = &  \frac{x}{(x^2+a^2)^{n-1}} + 2(n-1)I_{n-1} - 2(n-1)I_n
\end{eqnarray*}
これより、上の公式を得る。
\end{enumerate}

多項式と、多項式の商として表される関数を{\gt 有理関数}という。有理関数は、上記の積分と、部分分数分解と言われる方法を組み合わせると、初等関数の範囲で不定積分できる。

\begin{lemma}
有理関数 $f(x)/g(x)$ は、多項式および、
$$\frac{A}{(x-a)^k},\;\frac{Bx+C}{((x-\alpha)^2+\beta^2)^k}, \;(k\geq 1)$$
の有限個の和で表される。ここで、$k$ は、$g(x)$ に現れる、因数 $x-a$, $(x-\alpha)^2+
\beta^2$ の最大べきを越さない。
\end{lemma}

\begin{eg}
$g(x) = (x-1)^3(x-2)(x^2+2x+3)^2(x^2+4)$ であれば、どんな多項式 $f(x)$ に対しても、ある多項式 $h(x)$ と、定数が、とれて、
\begin{eqnarray*}
\frac{f(x)}{g(x)} & = & h(x) + \frac{A}{(x-1)^3} + \frac{B}{(x-1)^2} + \frac{C}{x-1}\\
& & + \frac{D}{x-2} \\
& & + \frac{Ex + F}{((x+1)^2+(\sqrt{2})^2)^2} + \frac{Gx + H}{(x+1)^2+(\sqrt{2})^2}\\
& & + \frac{Ix+J}{x^2+2^2}
\end{eqnarray*}
と表される。$h(x)$ は、まず、$f(x)$ の $g(x)$ による商である。
\end{eg}

\begin{eg}
\begin{enumerate}
\item ${\displaystyle \int\frac{x^3+2x^2+1}{x^2+3}dx}$
\begin{eqnarray*}
\mbox{} &= & \int(x+2)dx -\frac32\int\frac{2x}{x^2+3}dx - 5\int\frac{1}{x^2+3}dx \\
   	& = & \frac{x^2}{2} + 2x -\frac32\log(x^2+3) - \frac5{\sqrt{3}}\arctan\frac{x}{\sqrt{3}}
\end{eqnarray*}
\item ${\displaystyle \int\frac{1}{2x^2-x-3}dx = \frac15\int\left(\frac{2}{2x-3} - \frac{1}{x+1}\right)dx = \frac15\log\left|\frac{2x-3}{x+1}\right| + C}$
\item ${\displaystyle \int\frac{x^2-3x+5}{(x-1)^3}dx}$
$$ = \int\left(\frac{1}{x-1} - \frac{1}{(x-1)^2} + \frac{3}{(x-1)^3}\right)dx = \log|x-1| + \frac{1}{x-1} - \frac{3}{2(x-1)^2} + C$$
\item ${\displaystyle \int\frac{3}{x^3-1}dx}$
\begin{eqnarray*}
\mbox{}& = & \int \left(\frac{1}{x-1} - \frac{x+2}{x^2+x+1}\right)dx\\
& = & \int\left(\frac{1}{x-1}-\frac12\frac{2x+1}{x^2+x+1}-\frac32\frac{1}{(x+\frac12)^2+\frac34}\right)dx \\
& = & \log|x-1| - \frac12\log(x^2+x+1) - 2\int\frac{1}{\left(\frac{2x+1}{\sqrt{3}}\right)^2+1}dx\\
& = & \log|x-1| -\frac12\log(x^2+x+1)-\sqrt{3}\arctan\frac{2x+1}{\sqrt{3}}+C
\end{eqnarray*}
\item ${\displaystyle \int\frac{1}{x^2(x-3)}dx}$
\begin{eqnarray*}
\mbox{} &= & \frac19\int\frac{1}{x-3}dx - \frac19\int\frac2xdx - \frac13\int\frac{1}{x^2}dx\\
& = & \frac19\log|x-3| -\frac{1}{9}\log x - \frac13(-x^{-1})\\
& = & \frac19\log{|x-3|}{|x|} + \frac{1}{3x}
\end{eqnarray*}
\end{enumerate}
\end{eg}

\newpage\mysection{有理関数の積分の応用}
A. {\gt 三角関数の積分}

$t = \tan\frac{x}2$, とおくと、$\sin x = \frac{2t}{1+t^2}$, $\cos x = \frac{1-t^2}{1+t^2}$, $dx = \frac{2}{1+t^2}dt$ となるので、三角関数の有理関数の積分は、有理関数の積分に帰着できる。

\begin{eg}
\begin{enumerate}
\item ${\displaystyle \int\frac{\cos x}{1+\cos x + \sin x}dx}$
\begin{eqnarray*}
\mbox{} & = & \int\frac{\frac{1-t^2}{1+t^2}}{1+\frac{1-t^2}{1+t^2}+\frac{2t}{1+t^2}}\cdot\frac{2}{1+t^2}dt\\
& = & \int\frac{1-t^2}{(1+t^2)+(1-t^2)+2t}\cdot\frac{2}{1+t^2}dt\\
& = & \int\frac{1-t}{1+t^2}dt\\
& = & \int\frac{1}{1+t^2}dt - \frac{1}{2}\int\frac{2t}{1+t^2}dt\\
& = & \arctan t - \frac{1}{2}\log(1+t^2)\\
& = & \frac{x}{2} - \frac{1}{2}\log(\sec^2\frac{x}{2})\\
& = & \frac{x}{2} + \log(\cos\frac{x}{2})
\end{eqnarray*}
\end{enumerate}
\end{eg}

\medskip
B. {\gt 無理関数の積分}

$x, f(x)$ に関する、有理関数を、$R(x,f(x))$ と表すと、
$$I = \int R(x,^n\!\!\sqrt{\frac{ax+b}{cx+d}})dx,\; ad-bc\neq 0, \;n\neq 0, \;n\in \bZ$$
の時は、次のようにおけばよい。
$$t = ^n\!\!\sqrt{\frac{ax+b}{cx+d}}, \; x = \frac{-dt^n+b}{ct^n-a},\; dx = \frac{n(ad-bc)t^{n-1}}{ct^n-a)^2}dt$$
これより、次のように有理関数の積分に帰着できる。
$$I = n(ad-bc)\int R(\frac{-dt^n+b}{ct^n-a}, t)\frac{t^{n-1}}{ct^n-a)^2}dt$$

\medskip
C.{\gt その他、有理関数の積分に帰着できるもの}
\begin{eg}
\begin{enumerate}
\item ${\displaystyle \int\frac{1}{e^x+5e^{-x}+6}dx}$
\begin{eqnarray*}
\mbox{} & = & \int\frac{e^x}{e^{2x} + 5 + 6e^x}dx\\
	& = & \int\frac{1}{t^2+6t+5}dt \quad t = e^x\\
	& = & \frac14\int\left(\frac1{t+1}-\frac{1}{t+5}\right)dt\\
	& = & \frac14(\log|t+1| - \log|t+5|)\\
	& = & \frac14\log\frac{t+1}{t+5}\\
	& = & \frac14\log\frac{e^x+1}{e^x+5}
\end{eqnarray*}
\end{enumerate}
\end{eg}

\newpage\mysection{定積分と、微積分学の基本定理}
\begin{definition}
関数 $f(x)$ が、区間 $[a,b]$ 上有界とする。(この区間で定義されており、また、無限大になったりはしない。)分割 $\Delta = \{a = x_0<x_1<\cdots < x_n = b\}$ と実数 $t_i\in [x_{i-1},x_i]$ の集合 $\{t_1,t_2,\ldots,t_n\}$ に対して、
$$R_{\Delta,\{t_i\}}(f) = \sum_{i=1}^n f(t_i)(x_i-x_{i-1})$$
を、リーマン和又は、積和という。分割 $\Delta$ を限りなく細かくしていくとき、(すなわち、
$$|\Delta| = \max\{|x_i-x_{i-1}|\mid i = 1, 2, \ldots\}$$
 が、$0$ に近づくようにとっていくとき)$R_{\Delta,\{t_i\}}(f)$ が、$\{t_i\}$ の取り方に関係なく一定の実数 $I$ に近づくならば、$f(x)$ は、$[a,b]$ 上で、(定)積分可能という。この $I$ を $[a,b]$ 上 $f(x)$ の定積分と言い、
$$\int_a^b f(x)dx$$
と書く。このことを記号的に、次のようにも書く。
$$\int_a^b f(x)dx = \lim_{|\Delta|\to 0}\sum_{i=1}^n f(t_i)(x_i-x_{i-1})$$
\end{definition}

\begin{prop}
有限個の点を除いて連続で有界な関数は、積分可能である。
\end{prop}

\begin{prop}
関数 $f(x)$ と $g(x)$ は、区間 $[a,b]$ で積分可能とする。このとき、次が成り立つ。
\begin{itemize}
\item[$(1)$] ${\displaystyle \int_a^b (f(x) \pm g(x))dx  = \int_a^b f(x)dx \pm \int_a^b g(x)dx}$。
\item[$(2)$] ${\displaystyle \int_a^b k\cdot f(x)dx = k\int_a^b f(x)dx}$、($k$ は、定数。)
\item[$(3)$] $a\leq x\leq b$ で、$f(x)\geq g(x)$ ならば、${\displaystyle \int_a^b f(x)dx \geq \int_a^b g(x)dx}$。
\end{itemize}
\end{prop}

\begin{prop} {\rm (積分の平均値の定理)}
関数 $f(x)$ が、閉区間 $[a,b]$ 上で連続ならば、ある、$c\in (a,b)$  で、
$$\int_a^b f(x)dx = (b-a)f(c)$$
を満たすものがある。
\end{prop}
\proof
関数 $f(x)$ は、閉区間 $[a,b]$ で連続だから、最大、最小をとる。最大値を $M$ 最小値を $m$ とすると、リーマン和の定義から、
$$m(b-a) \leq \int_a^b f(x)dx \leq M(b-a)$$
これより、$m$ と、$M$ の間のある値 $A$ で、
$$m(b-a) \leq A = \int_a^b f(x)dx \leq M(b-a)$$
となる。$f(x)$ は、連続だから、中間値の定理により、$m$ と、$M$ の間の値は全てとる。従って、$f(c) = A$ を満たす $a < c < b$ を満たす $c$ が存在する。これは、命題の、条件を満たすものである。
\qed

\medskip
$a < b$ のとき、
$$\int_b^a f(x)dx = -\int_a^b f(x)dx, \; \int_a^a f(x)dx = 0$$
と定義する。こう定義すると、$b$ を変数とみなして、関数
$$F(x) = \int_a^x f(x)dx$$
が定義できる。実は、こうすると、$x = a$ で、$F(x)$ が、連続であることが分かる。さらに、次が成り立つ。
$$\int_a^x f(x)dx = \int_a^c f(x)dx + \int_c^x f(x)dx$$

\begin{thm} {\rm (微積分学の基本定理)}
関数 $f(x)$ が、閉区間 $[a,b]$ で連続であるとする。
$$F(x) = \int_a^x f(x)dx$$
とすると、$F(x)$ は、開区間 $(a,b)$ で、微分可能であり、$F'(x) = f(x)$ が成立する。
\end{thm}
\proof
$a < c < b$ とする。
\begin{eqnarray*}
\frac{F(x) - F(c)}{x-c} & = & \frac{{\displaystyle \int_a^x f(x)dx - \int_a^c f(x)dx}}{x-c}\\
& = & \frac{{\displaystyle \int_c^x f(x)}}{x-c}\\
& = & f(d(x))
\end{eqnarray*}
を満たす、点、$d(x)$ が、$x$ と $c$ の間にある。従って、$f(x)$ が、連続なことを考えると、
$$F'(c) = \lim_{x\to c}\frac{F(x) - F(c)}{x-c} = \lim_{x\to c}f(d(x)) = f(c)$$
これより、$F'(x) = f(x)$ を得る。
\qed

\medskip
さて、一般に、$F(x)$ を 関数 $f(x)$ の原始関数。すなわち、$F'(x) = f(x)$ を満たすものとする。すると、微分積分学の基本定理より、$\int_a^x f(x)dx$ も $f(x)$ の一つの原始関数だから、ある、定数 $C$ が、存在して、
$$F(x) = \int_a^x f(x)dx  + C$$ 
と書ける。$x = a$ とおくと、$F(a) = C$ を得るから、$x = b$ とおくと、
$$F(x) - F(a) = \int_a^x f(x)dx$$
特に、
$$\int_a^x f(x)dx =F(b) - F(a)$$
を得る。これを、
$$\int_a^x f(x)dx =F(b) - F(a) = [F(x)]_a^b = F(x)\mid^b_a$$
ともかく。

\begin{eg}
\begin{enumerate}
\item ${\displaystyle \int_0^1\frac{dx}{1+x^2} = \arctan x\mid^1_0 = \arctan 1 - \arctan 0 = \frac{\pi}4}$。これより、リーマン和の定義に戻ると、
\begin{eqnarray*}
\frac{pi}4 & = & \int_0^1\frac{dx}{1+x^2}\\
& = &  \lim_{n\to\infty}\sum_{k=1}^n\frac{1}{1+(\frac{k}{n})^2}\cdot\frac{1}{n}\\
& = &  \lim_{n\to\infty}\sum_{k=1}^n\frac{n}{n^2 + k^2}
\end{eqnarray*}
\item $x = \phi(t)$ とおくと、
$$\int_a^bf(x)dx = \int_{phi(a)}^{\phi(b)}f(\phi(t))\phi'(t)dt$$
\end{enumerate}
\end{eg}

\newpage\mysection{広義積分}
\begin{definition}
関数 $f(x)$ を、$[a,b)$ 上連続、かつ、
$$\int_a^bf(x)dx = \lim_{t\to b-0}\int^t_af(x)dx$$
とする。このとき、右辺の極限値が存在すれば、広義積分
$$\int_a^bf(x)dx$$
は、収束するという。$(a,b]$ 上で、連続な関数や、$b = \infty$、$a = -\infty$ の時も同様。
\end{definition}

\begin{eg}
\begin{enumerate}
\item ${\displaystyle \int_1^\infty\frac{1}{1+x^2}dx }$
$$= \lim_{t\to\infty}\int^t_1\frac{1}{1+x^2}dx = \lim_{t\to\infty}(\arctan t - \frac{\pi}{4}) = \frac{\pi}{2} - \frac{\pi}{4} = \frac{\pi}{4}$$
\item 次が成り立つ。
$$\int_0^k\frac{1}{x^r}dx = \left\{\begin{array}{lll}
\frac{k^{1-r}}{1-r} & (r<1) & \mbox{収束}\\
\infty & (r\geq 1) & \mbox{発散}\end{array}
\right.$$
$$\int_k^\infty\frac{1}{x^r}dx = \left\{\begin{array}{lll}
\frac{1}{(r-1)k^{r-1}} & (r>1) & \mbox{収束}\\
\infty & (r\leq 1) & \mbox{発散}\end{array}
\right.$$
\end{enumerate}
\end{eg}

\mysection{数値積分}
教科書に良い説明がありますので(4−6節)省略します。

\newpage\mysection{積分の応用}
A. 面積

関数 $f(x)$、$g(x)$、$x = a$, $x = b$ で、囲まれた部分の面積は、
$$S = \int_a^b |f(x) - g(x)|dx$$
で、与えられる。

\bigskip
B. 極座標と面積

$r = f(\theta)$、$\theta = \alpha$、$\theta = \beta$ で囲まれる部分の面積を考える。
まず、半径が $r$、角が、$\theta$ である、扇形の、面積は $r\theta^2/2$ であることに注意する。$\alpha$ から、$\theta$ までの、面積を $S(\theta)$ で表すと、$0\leq \epsilon(\Delta\theta)\leq \Delta\theta$ なる、$\epsilon(\Delta\theta)$ において、
$$S(\theta+\Delta\theta) - S(\theta) = \frac{1}{2}(f(\theta + \epsilon(\Delta\theta))^2\Delta\theta$$
となる。これより、
$$\lim_{\Delta\theta\to 0}\frac{S(\theta+\Delta\theta) - S(\theta)}{\Delta\theta} = \frac12\lim_{\Delta\theta\to 0}(f(\theta + \epsilon(\Delta\theta))^2 = \frac12(f(\theta))^2$$
微分の定義より、$S'(\theta) = \frac12(f(\theta))^2$。これより、求める面積は、
$$S = \frac12\int_{\alpha}^{\beta}(f(\theta))^2d\theta$$
となる。

\begin{eg}
\begin{enumerate}
\item $r = a(1+\cos\theta)$ (カーディオイド)で囲まれる部分の面積。
\begin{eqnarray*}
S & = & 2\cdot \frac12\int_0^\pi (a(1+\cos\theta))^2d\theta\\
   & = & 4a^2\int^\pi_0\cos^4\frac\theta2d\theta\\
   & = & 8a^2\int_0^{\pi/2}\cos^4 tdt\quad t = \frac\theta2, \;d\theta = 2dt\\
   & = & 8a^2\cdot \frac34\cdot\frac12\cdot\frac\pi2\\
   & = & \frac32\pi a^2
\end{eqnarray*}
最後のところは、次を、使った。すなわち、$n\geq 2$ とすると、
\begin{eqnarray*}
I_n & =  & \int_0^{\pi/2}\cos^nxdx\\
     & = & \int_0^{\pi/2}\cos x\cos^{n-1}xdx\\
     & = & [\sin x\cos^{n-1}x]^{\pi/2}_0 + (n-1)\int_0^{\pi/2}\sin^2x\cos^{n-2}xdx\\
     & = & (n-1)\int_0^{\pi/2}(1-\cos^2 x)\cos^{n-2}xdx\\
     & = & (n-1)I_{n-2} - (n-1)I_n, \;\mbox{ 従って、}\\
I_n & = & \frac{n-1}{n} I_{n-2}
\end{eqnarray*}
$I_0 = \pi/2$、$I_1 = 1$ に注意すると、
\begin{eqnarray*}
I_{2n} & = & \frac{(2n-1)\cdot(2n-3)\cdots1}{(2n)\cdot(2n-2)\cdots2}\cdot\frac\pi2\\
I_{2n+1} & = & \frac{(2n)\cdot(2n-2)\cdots2}{(2n+1)\cdot(2n-1)\cdots3}
\end{eqnarray*}
\end{enumerate}
\end{eg}

\bigskip
C. 平面上の曲線の長さ

曲線 $y = f(x)$ の $x$ が、$a$ から、$b$ まで、変化する、部分の長さ $L$ を考える。$a\leq x\leq b$ である、任意の $x$ に対して、$a$ から、$x$ までの、部分の長さを $L(x)$ とおく。$x$ を、$\Delta x$ だけ変化させたときの曲線の長さは、$\Delta x$、$\Delta y$ を2辺とする、直角三角形の斜辺の長さとみなされる。これより、
$$\lim_{\Delta x\to 0}\frac{L(x + \Delta x) - L(x)}{\Delta x} = \lim_{\Delta x\to 0}\frac{\sqrt{(\Delta x)^2 + (\Delta y)^2}}{\Delta x} = \lim_{\Delta x\to 0}\sqrt{1 + \left(\frac{\Delta y}{\Delta x}\right)^2} = \sqrt{1+(y')^2}$$
微分の定義より、$L'(x) = \sqrt{1+(y')^2}$。積分の定義より、求める長さは、
$$L = \int_a^b\sqrt{1+(y')^2}dx$$
となる。

\begin{eg}
\begin{enumerate}
\item 半径 $a$ の円周の長さ。円を $x^2 + y^2 = a^2$ とすると、上半円は $y = \sqrt{a^2-x^2}$ だから、
$$y' = \frac{-x}{\sqrt{a^2-x^2}}, \; 1+(y')^2 = \frac{a^2}{a^2 - x^2}$$
これより、長さを $L$ とすると、
$$L = 4\int_0^a\sqrt{1+(y')^2}dx = 4\int_0^1\sqrt{\frac{a^2}{a^2-x^2}}dx = 4a\left[\arcsin\frac{x}{a}\right]_0^a = 2\pi a$$
\end{enumerate}
\end{eg}

D. 極座標と曲線の長さ

$r = f(\theta)$ が、$\theta = \alpha$ から、$\theta = \beta$ まで変化する部分の長さを考える。
まず、半径が $r$、角が、$\theta$ である、扇形の弧の長さは $r\theta$ であることに注意する。$\alpha$ から、$\theta$ までの、弧の長さを $L(\theta)$ で表す。$L(\theta + \Delta\theta) - L(\theta)$  を、直角三角形の斜辺として近似すると、
$$L(\theta+\Delta\theta) - L(\theta) = \sqrt{(r\Delta\theta)^2 + (\Delta r)^2}$$
これより、
$$\lim_{\Delta\theta\to 0}\frac{L(\theta+\Delta\theta) - L(\theta)}{\Delta\theta} = \lim_{\Delta\theta\to 0}\sqrt{r^2 + \left(\frac{\Delta r}{\Delta\theta}\right)^2} = \sqrt{r^2 + \left(\frac{dr}{d\theta}\right)^2}$$
微分の定義より、
$$L'(\theta) = \sqrt{r^2 + \left(\frac{dr}{d\theta}\right)^2}$$
これより、求める長さは、
$$L = \int_\alpha^\beta\sqrt{r^2 + \left(\frac{dr}{d\theta}\right)^2}d\theta$$
となる。

\begin{eg}
\begin{enumerate}
\item 半径 $a$ の円周の長さ。円を $r = r(\theta) = a$ とすると、
$$\frac{dr}{d\theta} = 0, \; r^2 + \left(\frac{dr}{d\theta}\right)^2 = a^2$$
これより、長さを $L$ とすると、
$$L = \int_0^{2\pi}\sqrt{r^2 + \left(\frac{dr}{d\theta}\right)^2}d\theta = \int_0^{2\pi}a d\theta = [a\theta]_0^{2\pi} = 2\pi a$$
\end{enumerate}
\end{eg}

E. 回転体の体積

$y = f(x)$ と、$x = a$、$x = b$ で囲まれる部分を $x$ 軸の周りに回転させて出来る回転体の体積を考える。$a\leq x\leq b$ なる、任意の $x$ に対して、$a$ から、$x$ の部分の体積を $V(x)$ と表すと、ある、$0<\theta<1$ で、
$$V(x+\Delta x) - V(x) = \pi(f(x+\theta\Delta x))^2\Delta x$$
従って、微分の定義より、
$$V'(x) = \lim_{\Delta x\to 0}\frac{V(x+\Delta x) - V(x)}{\Delta x} = \pi\lim_{\Delta x\to 0}(f(x+\theta\Delta x))^2 = \pi(f(x))^2$$
これより、積分の定義を用いると、
$$V = \pi\int_1^b(f(x))^2dx$$


\end{document}
%%%%%%%  End of File %%%%%%%%%%%

\end{document}
