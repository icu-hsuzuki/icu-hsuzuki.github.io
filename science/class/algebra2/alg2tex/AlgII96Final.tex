%Imported from Algebra II Midterm : revised : October 16, 1996
%Original : November 15, 1996

\documentstyle[12pt]{jarticle}

%A4 Size Setting
\topmargin = -0.5cm
\oddsidemargin = 0cm \evensidemargin = 0cm
\textheight = 24cm \textwidth = 16cm
%A4 size Setting End

\pagestyle{empty}

\newtheorem{thm}{定理}
\newcommand{\bZ}{\mbox{\boldmath $Z$}}
\newcommand{\bQ}{\mbox{\boldmath $Q$}}
\newcommand{\bR}{\mbox{\boldmath $R$}}
\newcommand{\bC}{\mbox{\boldmath $C$}}
\newcommand{\bigx}{{\large $\times$}}

\begin{document}
\begin{center}
{\gt\LARGE Algebra II  FINAL}\\
{\gt Friday 22, 1996}
\end{center}

\noindent
ID 番号、氏名を、各解答用紙に、また、問題番号も忘れずに書いて下さい。

\begin{enumerate}
\item 次のうち正しいものには $\bigcirc$、誤っているものには \bigx を解答欄に記入せよ。(20pts)
     \begin{enumerate}
     \item 多項式環 $\bZ[x]$ は、Euclid 整域である。
     \item $(0)$ が極大イデアルである可換環は、体。
     \item $(0)$ が素イデアルである可換環は、整域。
     \item $I$ を整域 $R$ の素イデアルとすると、$I$ の積、$I\cdot I$ が素イデアルになることはない。
     \item ネーター環上有限生成な環は、ネーター環。
     \end{enumerate}

\medskip     
\item $R$ を整域とするとき次を示せ。(40pts)
     \begin{enumerate}
     \item ユークリッド整域は、単項イデアル整域である。
     \item $\bZ[\sqrt{-1}] = \{a + b\sqrt{-1}\mid a, b\in \bZ\}$ は、ユークリッド整域である。
     \item $R$ が、単項イデアル整域ならば、$R$ の $(0)$ でない素イデアルは、極大イデアルである。
     \item $R$ が、単項イデアル整域ならば、$R$ は、ネーター環である。
     \end{enumerate}

\medskip      
\item $R$ を整域とするとき次を示せ。(20pts)
     \begin{enumerate}
     \item $R$ 上の多項式環 $R[x]$ は、整域である。
     \item $R$ 上の多項式環 $R[x_1, x_2, \ldots, x_n]$ は、整域である。
     \end{enumerate}

\medskip     
\item $R$ を可換環、$I$、$J$ を $R$ のイデアルで、$I + J = R$ を満たすものとする。(20pts)
     \begin{enumerate}
     \item $I \cap J = IJ$ を示せ。
     \item $R/IJ \simeq R/I $ \bigx $R/J$ を示せ。(すなわち、Chinese Remainder's theorem を示せ。)
     \end{enumerate}

\medskip      
\item $R = \bZ_{12}$ とする。以下のそれぞれの場合に、$S^{-1}R$ の元の個数を求めよ。$S^{-1}R$ が定義できないときは、その理由を述べよ。(20pts)
	\begin{enumerate}
	\item $S = \{1, 5, 7, 11\}$。
	\item $S = \{1,3,5,7,9,11\}$。
	\item $S = \{1,2,4,8\}$
	\item $S = \{1,4\}$。
	\end{enumerate}

\medskip   
\item $M$ を $R$-左加群、$N$ をその部分加群とするとき次を示せ。(20pts)
	\begin{enumerate}
	\item $U$、$V$ を共に、$M$ の部分加群で、$U\subset V$  を満たすものとする。このとき次を示せ。
$$U = V \Leftrightarrow U+N = V+N \mbox{ and } U\cap N = V\cap N$$
	\item $M$ を $R$-左加群、$N$ をその部分加群とするとき、
$N$ と、$M/N$ がネーター加群ならば、$M$ も、ネーター加群である事を示せ。
	\end{enumerate}

\newpage	
\item $R = \bZ[\sqrt{-1}]$ とする。$x,y,z\in\bZ$ とし、$x^2 + y^2  = z^2$。さらに、$x,y,z$ の最大公約数は、$1$ とする。このとき、以下を示せ。(60pts)
	\begin{enumerate}
	\item $R$ の単数群 $U(R)$ は、$\{1,-1,\sqrt{-1},-\sqrt{-1}\}$ である。
	\item $z$ は、奇数である。(ヒント:$4$ で割ったあまりで考えよ。)
	\item $x+y\sqrt{-1}$ と、$x-y\sqrt{-1}$ で生成されたイデアルをある整数 $p,q$ について、 
	$$I = (x+y\sqrt{-1}) + (x-y\sqrt{-1}) = (p+q\sqrt{-1})$$
	と置く。($R$ は、単項イデアル整域であるからこれは、可能。)このとき、$2\in I = (p+q\sqrt{-1})$ であることを示せ。(ヒント:$x$ と $y$ は、互いに素である。)
	\item $z\in I = (p+q\sqrt{-1})$ である。これと、(c) より、$N(p+q\sqrt{-1}) = 1$、すなわち、$I = R$ である。
	\item $(x+y\sqrt{-1})(x-y\sqrt{-1}) = z^2$ より、$x + y\sqrt{-1} = \epsilon(u+v\sqrt{-1})^2$ となる、$R$ の元 $u+v\sqrt{-1}$ がある。ただし、$\epsilon\in U(R)$。
	\item 条件を満たす $x,y,z$ にたいしては、整数 $u,v$ で以下のどちらかの条件を満たすものが存在する。
		$$(x,y,z) = (u^2-v^2,2uv,u^2+v^2), \;\mbox{ または、}(x,y,z) = (2uv,u^2-v^2,u^2+v^2)$$
	\end{enumerate}
\end{enumerate}

 

\vspace{5ex}
\noindent Algebra II を受講した感想、コメント、アドヴァイス、なんでも構いませんから書いて下さい。(これによって、成績に不利益を及ぼすことはありませんが、同時に、利益を受けることもありません。)

\vspace{2ex}
\noindent
Midterm を 100点、演習を 100点、FINAL を 200 点満点として計算して、成績を出す予定です。
         

\begin{flushright}
鈴木寛@国際基督教大学数学教室
\end{flushright}
\end{document}