%Algebra II Midterm : original  October 10, 1996

%\documentstyle[12pt]{jarticle}
\documentstyle{jarticle}

%A4 Size Setting
\topmargin = -0.5cm
\oddsidemargin = 0cm \evensidemargin = 0cm
\textheight = 24cm \textwidth = 16cm % default 16cm
%A4 size Setting End

\pagestyle{empty}

\newtheorem{thm}{定理}
\newcommand{\bZ}{\mbox{\boldmath $Z$}}
\newcommand{\bQ}{\mbox{\boldmath $Q$}}
\newcommand{\bR}{\mbox{\boldmath $R$}}
\newcommand{\bC}{\mbox{\boldmath $C$}}
\newcommand{\bigx}{{\large $\times$}}

\begin{document}
\begin{center}
{\gt\LARGE Algebra II  Take Home Midterm}\\
{\gt Autumn 1996 : Due Wednesday 16}
\end{center}

\noindent
ID 番号、氏名を、各解答用紙に、また、問題番号も忘れずに書いて下さい。

\begin{enumerate}
\item $R = \bZ_{18} = \bZ/18\bZ$、$\pi : \bZ \to R$ を自然な準同型としたとき、次の各問について、正しければ証明し、誤っていれば、そのことを示せ。
     \begin{enumerate}
     \item $R$ の零因子全体は、$R$ のイデアルである。
     \item $P$ を $\bZ$ の素イデアルとするとき、$\pi(P)$ は、いつでも、$R$ の素イデアルである。
     \item $Q$ を $R$ の素イデアルとするとき、$\pi^{-1}(Q)$ は、いつでも、$\bZ$ の素イデアルである。
     \item $R$ の単数群 $U(R)$ の補集合 $R - U(R)$ は、$R$ の素イデアルである。
     \item $R$ の素イデアルは、すべて、極大イデアルである。
     \end{enumerate}

\item $R$ を整域とし、$U(R)$ をその単数群とするとき次を示せ。
     \begin{enumerate}
     \item $(a) = (b) \Rightarrow a = ub \;\mbox{ for some }\;u\in U(R)$.
     \item $R$ の極大イデアルは、素イデアルである。
     \item 素イデアルであるが、極大イデアルではないものの例を一つ挙げよ。($R$ とその素イデアルで、極大イデアルではないものの例)
     \item $R$ が、単項イデアル整域ならば、$R$ の $0$ とは、異なる素イデアルは、極大イデアル。
     \end{enumerate}
     
\item $\bQ[x]$ を、有理数体上の多項式環とし、$p(x) = x^2 - 2$ とする。このとき、以下を示せ。
     \begin{enumerate}
     \item $f(x)$ は、$\bQ$ 上既約であることを示せ。
     \item $\bQ[\sqrt{2}] = \{f(\sqrt{2})\mid f(x)\in \bQ[x]\}$ とすると、
     $$\bQ[\sqrt{2}] = \{a_0 + a_1\sqrt{2}  \mid a_0, a_1\in \bQ\}.$$
     \item $\bQ[\sqrt{2}]$ は、体である。
     \item $\bQ[\sqrt{2}]$ の元で、モニックな、整数係数の多項式の根となるものをすべて求めよ。これを、$R$ とする。
     \item $R$ には、単数が無限個あることを示せ。
     \item 有理数を成分とする、$4\times 4$ の行列 $A$ で、以下を満たすものを一つ求めよ。
     $$\bQ[A] = \{f(A)\mid f(x)\in \bQ[x]\} \simeq \bQ[\sqrt{2}].$$
     \end{enumerate}
\end{enumerate}

 

\vspace{5ex}
\noindent Algebra II を受講した感想、コメント、アドヴァイス、なんでも構いませんから書いて下さい。(これによって、成績に不利益を及ぼすことはありませんが、同時に、利益を受けることもありません。)

\vspace{2ex}
\noindent
Midterm を 100点、演習を 100点、FINAL を 200 点満点として計算して、成績を出す予定です。
         

\begin{flushright}
鈴木寛@国際基督教大学数学教室
\end{flushright}
\end{document}