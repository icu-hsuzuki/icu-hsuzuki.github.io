%Algebra II Final : original  November 20, 1995

\documentstyle[12pt,ascmac]{jarticle}

%A4 Size Setting
%\topmargin = 0cm
%\oddsidemargin = 0cm \evensidemargin = 0cm
%\textheight = 23cm \textwidth = 15cm % default 16cm
%A4 size Setting End

\newtheorem{thm}{定理}
\newcommand{\bZ}{\mbox{\boldmath $Z$}}
\newcommand{\bQ}{\mbox{\boldmath $Q$}}
\newcommand{\bR}{\mbox{\boldmath $R$}}
\newcommand{\bC}{\mbox{\boldmath $C$}}
\newcommand{\bigx}{{\large $\times$}}

\begin{document}
\begin{center}
{\gt\LARGE Algebra II  FINAL}\\
{\gt Autumn 1995}
\end{center}

\noindent
ID 番号、氏名を、各解答用紙に、また、問題番号も忘れずに書いて下さい。

\begin{enumerate}
\item 次のうち正しいものには $\bigcirc$、誤っているものには \bigx を解答欄に記入せよ。(15pts)
     \begin{enumerate}
     \item 多項式環 $\bZ[x]$ は、単項イデアル整域である。
     \item 多項式環 $\bZ[x,y]$ は、一意分解整域である。
     \item 多項式環 $\bZ[x,y,z]$ は、ネーター環である。
     \item 多項式環 $\bZ_5[x]$ は、Euclid 整域である。
     \item 多項式環 $\bZ_5[x,y]$ のイデアルは有限生成である。
     \end{enumerate}
     
\item $R = \bZ_{18}$ としたとき、次の各問に答えよ。(25pts)
     \begin{enumerate}
     \item $R$ の零因子をすべて求めよ。 
     \item $R$ の素イデアルをすべて求めよ。
     \item $R$ の単数群 $U(R)$ の最大位数の元の位数。
     \item $R$ の全商環の元の数。
     \item $S = \{\bar{1},\bar{9}\}$ としたとき、$S^{-1}R$ の元の数。
     \end{enumerate}

\item $R$ を整域とし、$U(R)$ をその単数群とするとき次を示せ。(40pts)
     \begin{enumerate}
     \item $(a) = (b) \Rightarrow a = ub \;\mbox{ for some }\;u\in U(R)$.
     \item $R$ の極大イデアルは、素イデアル。
     \item $R$ が、単項イデアル整域ならば、$R$ の $(0)$ とは異なる素イデアルは、極大イデアル。
     \item $R$ が、単項イデアル整域ならば、$R$ は、ネーター環。
     \end{enumerate}
     
\item $R$ を整域とするとき次を示せ。(10pts)
     \begin{enumerate}
     \item $R$ 上の多項式環 $R[x]$ は、整域である。
     \item $R$ 上の多項式環 $R[x_1, x_2, \ldots, x_n]$ は、整域である。
     \end{enumerate}
     
\item $R$ を可換環、$I$、$J$ を $R$ のイデアルで、$I + J = R$ を満たすものとする。(15pts)
     \begin{enumerate}
     \item $I \cap J = IJ$ を示せ。
     \item $R/IJ \simeq R/I $ \bigx $R/J$ を示せ。(すなわち、Chinese Remainder's Theorem を示せ。)
     \end{enumerate}

\item $R$ を一意分解整域、$S$ をその積閉集合とする。$S^{-1}R$ の素元は、$R$ の素元 $\pi$ で、$(\pi)\cap S = \emptyset$ となるものと同伴であることを示せ。(10pts)
     
\item $\bQ[x]$ を、有理数体上の多項式環とする。このとき、 \\
$\bQ[x]/(x^3-3x^2+6x -3)$ は、体であることを示せ。 (10pts)

\item $R = \bZ[\sqrt{10}] = \{a + b\sqrt{10}\mid a, b\in\bZ\}$ とするとき、以下のことを示せ。ただし、$N(a + b\sqrt{10}) = a^2 - 10b^2$ とする。(25pts)
     \begin{enumerate}
     \item $\alpha\in R$ において、$\alpha\in U(R)\Leftrightarrow N(\alpha) = \pm1$.
     \item $R$ には、無限個単元があることを示せ。(ヒント:$\alpha\in U(R)$ に対して、$\alpha^n$ および、その、絶対値を考えよ。) 
     \item $\alpha\in R$ とするとき、$N(\alpha)\neq \pm 2, \pm 3$ であることを示せ。(ヒント:$\bZ_5$ において、$\{a^2\mid a\in \bZ_5\} = \{\bar{0},\bar{1},\bar{4}\}$ であることを用いよ。
     \item $R$ には、素元 $\pi$ で、$(\pi)$ は素イデアルではないものが存在することを示せ。
     \end{enumerate}
\end{enumerate}

 

\vspace{5ex}
\noindent Algebra II を受講した感想、コメント、アドヴァイス、なんでも構いませんから書いて下さい。(これによって、成績に不利益を及ぼすことはありませんが、同時に、利益を受けることもありません。)

\vspace{2ex}
\noindent
FINAL を 150 点満点とし、小テストの点を20点満点演習の点を 30 点満点で計算して、成績を出す予定です。
         

\begin{flushright}
鈴木寛@国際基督教大学数学教室
\end{flushright}
\end{document}