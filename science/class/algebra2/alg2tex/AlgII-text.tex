%original : 「環・体・整域」 September 10, 11, 1996
%original : 「イデアルと剰余環」 September 16, 20, 1996
%original : 「準同型定理」September 23, 1996
%original : 「素イデアルと極大イデアル」September 6, 1997
%original : 「環の直和」September 7, 1997
%original : 「商環」September 7, 1997
%original : 「一意分解整域」September 8, 11, 12, 1997
%original : 「加群」September 13,1997
%original : 「ヒルベルトの基定理」September 14, 1997
%%revised : 1997.9.11, 15


\documentstyle[12pt]{jarticle}
%A4 Size Setting
\topmargin = -0.5cm
\oddsidemargin = 0cm \evensidemargin = 0cm
\textheight = 23cm \textwidth = 15cm % default 16cm
%A4 size Setting End

\title{ALGEBRA II}
      
\author{Hiroshi SUZUKI\thanks{E-mail:hsuzuki@icu.ac.jp}\\ 
        Department of Mathematics \\ 
        International Christian University}

\newtheorem{thm}{定理}[section]
\newtheorem{prop}[thm]{命題}
\newtheorem{lemma}[thm]{補題}
\newtheorem{cor}[thm]{系}
\newtheorem{exercise}{練習問題}[section]
\newtheorem{example}{例}[section]
\newtheorem{problem}{問題}[section]
\newtheorem{defin}{定義}[section]
\newenvironment{definition}{\begin{defin} \rm}{\end{defin}}
\newenvironment{ex}{\begin{exercise} \rm}{\end{exercise}}
\newenvironment{eg}{\begin{example} \rm}{\end{example}}
\newenvironment{prob}{\begin{problem} \rm}{\end{problem}}
\newcommand{\remarks}{\vspace{2ex}\noindent{\bf Remarks.\quad}}
\newcommand{\note}{\vspace{2ex}\noindent{\gt 注\quad}}
\newcommand{\proof}{{\gt 証明\quad}}
\newcommand{\pf}{({\it Pf.})\quad}
\newcommand{\qed}{\hfill\hbox{\rule{6pt}{6pt}}}
\newcommand{\bZ}{\mbox{\boldmath $Z$}}
\newcommand{\bN}{\mbox{\boldmath $N$}}
\newcommand{\bR}{\mbox{\boldmath $R$}}
\newcommand{\bC}{\mbox{\boldmath $C$}}
\newcommand{\bQ}{\mbox{\boldmath $Q$}}
\newcommand{\cQ}{{\cal Q}}
\newcommand{\mat}{\mbox{{\rm Mat}}}
\newcommand{\GL}{\mbox{{\rm GL}}}
\newcommand{\mod}{\mbox{{\rm mod} }}
\renewcommand{\ker}{\mbox{{\rm Ker}}}
\newcommand{\sker}{\mbox{{\scriptsize\rm Ker}}}
\newcommand{\im}{\mbox{{\rm Im}}}

\begin{document}
\setcounter{page}{0}
\maketitle

\newpage
\section{環・体・整域}
\begin{definition}
加法と乗法という二つの演算が定義された集合 $R$ が環 (ring) であるとは、次の R1 $\sim$ R4 を満たすことである。
\begin{itemize}
\item[R1] $R$ は加法に関して加群 (commutative [abelian] group)。
\item[R2]  任意の $a, b, c\in R$ に対して、$(ab)c = a(bc)$。(結合律 (associative law))
\item[R3]  任意の $a, b, c\in R$ に対して、$a(b+c) = ab + ac, \;(a+b)c = ac + bc$。(分配律 (distributive law))
\item[R4] $R$ の $0$ ($R$ の加法の単位元)と異なる元 $1$ で、$1x = x1 = x$ を任意の $x\in R$ に対して満たすものがある。(乗法の単位元)

さらに次の R5 を満たすとき、可換環 (commutative ring) という。
\item[R5] $ab = ba$ for all $a, b\in R$。
\end{itemize}
\end{definition}

\note 
\begin{itemize}
\item R1 $\sim$ R3 のみを満たすものを環と呼び R4 を満たすものを「単位元を持つ環 (unital ring)」と呼び区別することも多い。
\item 環 $R$ は乗法に関して R2、R4 を満たすから、モノイドである。従ってその正則元全体 $U(R)$ は群となる。これを単数群という。 
\item $0x = x0 = 0\;(\neq 1)$ だから、$0$ は正則元ではない。すなわち、$U(R)\subset R-\{0\}$。$R-\{0\}$ を $R^{\#}$ とも書く。\\
\pf  $0 = 0x + (-0x) = (0+0)x + (-0x) = 0x + 0x + (-0x) = 0x + 0 = 0x$。
\end{itemize}

\begin{definition}
$U(R) = R - \{0\} = R^{\#}$ となる環を斜体 (skew field) 、可換な(すなわち R5 を満たす)斜体を、体 (field) という。
\end{definition}

\begin{definition}
環 $R$ の元 $a$ に対し、$b\neq 0$ で $ab = 0$ [$ba = 0$] となるものが存在するとき、$a$ は左零因子 (left zero divisor) [右零因子 (right zero divisor)] という。可換環の時は単に零因子 (zero divisor) という。$0$  以外に零因子のない可換環を整域という。すなわち、
\begin{itemize}
\item[R6] $ab = 0 \longrightarrow a = 0 \mbox{ or } b = 0$。
\end{itemize}
を満たす可換環、または R1 $\sim$ R6 を満たすもの。
\end{definition}

\note 体は、整域である。

\begin{eg}
\begin{enumerate}
\item 有理整数環 $\bZ$ は、整域である。
\item 有理数体 $\bQ$、実数体 $\bR$、複素数体 $\bC$ は、いずれも可換体である。
\item $R$ を環とするとき、$R$ 上の全行列環、$\mat_n(R)$ は、非可換な環である。
\item $n$ を自然数としたとき、$\bZ_n = \bZ/n\bZ = \{\bar{0}, \bar{1}, \ldots, \overline{n-1}\}$ に、通常の和、積の $n$ による剰余によって演算を定義すると、元の数が $n$ である可換環になる。
\end{enumerate}
\end{eg}

以下では、有理整数環 $\bZ$ とともに、非常に重要な可換環である多項式環について基本事項を学ぶ。

可換環 $R$ の元を係数とする文字 $x$ の整式
$$f(x) = a_0 + a_1x + \cdots + a_nx^n, \;a_i\in R \mbox{ for } i = 0,1,\ldots, n$$
を $x$ を不定元とする $R$ の多項式といい、$x$ を不定元という。また、$R[x]$ で、$x$ を不定元とする $R$ 上の多項式全体を表すものとする。
$$f = f(x) = a_0 + a_1x + \cdots + a_nx^n, \;g = g(x) = b_0 + b_1x + \cdots + b_mx^m$$
を $R[x]$ の元とするとき、和および積を以下のように定義する。
\begin{eqnarray*}
f + g  & = & \sum_i (a_i + b_i)x^i\\
fg & = & \sum_i\Bigl(\sum_ja_jb_{i-j}\Bigr)x^i
\end{eqnarray*}
この演算に関して $R[x]$ は環になる。これを、$R$ 上の多項式環という。

\smallskip
$f = f(x) = a_0 + a_1x + \cdots + a_nx^n \in R[x], \;a_n\neq 0$ の時、$n = \deg f$ と書き $f$ の次数という。$f(x) = 0$ の時は、$\deg f = \deg 0 = -\infty$ とする。

\begin{prop}
$R$ を整域、$f, g\in R[x]$ とする。このとき、次が成立する。
\begin{itemize}
\item[$(1)$] $\deg(f+g) \leq \max(\deg f, \deg g)$。
\item[$(2)$] $\deg(fg) = \deg f + \deg g$。特に、整域 $R$ 上の多項式環は、また整域である。
\end{itemize}
\end{prop}
\proof
(1) (2) ともに明らか。$fg = 0$ とする。(2) を用いると、
$$-\infty = \deg fg = \deg f + \deg g.$$
従って、$\deg f = -\infty$ または $\deg g = -\infty$。すなわち、$f = 0$ または $g = 0$ を得る。
\qed

\begin{thm} \label{thm:poly-euclid}
$R$ を可換環。$f, g\in R[x]$ とし、$g$ の最高次の係数は、$R$ の正則元だとする。このとき、$q, r\in R[x]$、$\deg r < \deg g$ で、$f = gq + r$ となるものが存在する。さらに、$R$ が整域ならば、この様な $q, r\in R[x]$ は、ただ一つに決まる。
\end{thm}
\proof
$f = a_nx^n + \cdots + a_1x + a_0$、$a_n\neq 0$、$g = b_mx^m + \cdots + b_1x + b_0$ とする。まず、$n<m$ の時は、$q = 0$、$r = f$ とすれば良い。

$n\geq m$ と仮定し、$n = \deg f$ に関する帰納法で証明する。$b_m$ は、仮定より正則元だから、逆元が存在する。$h = f - (a_nb_m^{-1})x^{n-m}g$ とすれば、$f$ の最高次の係数が消えるから、$\deg h < n$。従って、帰納法の仮定より、$R[x]$ の元 $q_1, r$ で、$\deg r < \deg g$ かつ、$h = gq_1 + r$ となるものがある。従って
$$f = g(q_1 + (a_nb_m^{-1})x^{n-m}) + r$$
と表される。よって、$q = q_1 + (a_nb_m^{-1})x^{n-m}$  と置けばよい。

$R$ を整域とし、一意性を示す。
$$f = gq + r = gq' + r', \; \deg r, \deg r' < \deg g$$
とする。すると、$g(q - q') = r' - r$。ここで、次数を比べると、
$$\deg g + \deg(q - q') = \deg(g(q-q')) = \deg(r' -r) \leq \max(\deg r', \deg r) < \deg g.$$
$g\neq 0$ より、$q - q' = 0$。従って、$r' - r = 0$。すなわち、$q = q'$、$r = r'$ を得る。
\qed

\medskip
$n$ 変数多項式環は、帰納的に、$R[x_1,\ldots, x_n] = (R[x_1,\ldots, x_{n-1}])[x_n]$ によって定義する。この元は、一般には、次のように書ける。
$$\sum_{i_1, \ldots, i_n}a_{i_1,\ldots, i_n}x_1^{i_1}\cdots x_n^{i_n}, \; a_{i_1,\ldots, i_n}\in R.$$
また、$R[x,y] = (R[x])[y] = (R[y])[x]$ と見ることも出来る。

\newpage
\section{イデアルと剰余環}
$R$ を環とすると、加法に関しては、加群だから、加法に関する部分群 $I$ は、すべて、正規部分群である。従って、$R/I$ は加群となる。どのような条件のもとで、$R/I$ が環になるであろうか。

$xy \in (x+I)(y+I)$ だから、積が自然に定義できるためには、
$$(x+I)(y+I) \subset xy + I$$
であることが必要である。逆に、上の条件を満たせば、積が定義できる。ここで、$x = 0$ または、$y = 0$ とおくことによって、$xI\subset I$、$Iy\subset I$ を満たすことが必要であることが分かる。

\begin{definition}
環 $R$ の部分集合 $I\neq \emptyset$ が、次の二つの条件、
\begin{itemize}
\item $a, b\in I\longrightarrow a+b\in I$
\item $a\in I, \;r\in R \longrightarrow ra\in I,\;[ar\in I]$.
\end{itemize}
を満たすとき、$I$ を $R$ の左イデアル [右イデアル] と呼び、左右イデアルを両側イデアルと呼ぶ。
\end{definition}

$A$、$B$ を環 $R$ の部分集合とするとき、これらの和および積を次のように定義する。特に、積の定義に注意。
\begin{itemize}
\item $A + B = \{a + b\mid a\in A, b\in B\}$.
\item $AB = \{\sum_ia_ib_i \mid a_i\in A, \;b\in B\}$.
\end{itemize}

\begin{ex} 以下を示せ。
\begin{enumerate}
\item 環 $R$ の左(右、両側)イデアル $I, J$ に対して、$I\cap J$、$I+J$ は共に、左(右、両側)イデアルである。
\item $I, J$ が環 $R$ の両側イデアルならば、$IJ$ も両側イデアルで、$IJ\subset I\cap J$ を満たす。また、等号が成り立たない例をあげよ。
\end{enumerate}
\end{ex}

この節の始めに見たように、$I$ を環 $R$ の両側イデアルで $I\neq R$ とすると、$R/I$ は、
$$(a + I) + (b + I) = (a+b) + I, \; (a + I)\cdot (b + I) = (ab) + I$$
と、和と積を定義する事により、$R/I$ は環になる。この環を剰余環 (quotient ring) と言う。

\begin{itemize}
\item $a\in R$ のとき、$Ra$ [$aR$] は、左イデアル [右イデアル] になるが、これを単項 (principal) 左 [右] イデアルと言う。$R$ が可換環の時は、$Ra = aR$ を $(a)$ ともかく。
\item $0 = \{0\}$、$R$ は、$R$ の両側イデアルであるが、これらを、$R$ の自明なイデアルという。
\item $I$ を $R$ の左 [右] イデアルとする。このとき、
$$I = R \Leftrightarrow U(R)\cap I \neq \emptyset.$$
\pf $I = R$ とすると、$1\in I\cap U(R)$ より、$U(R)\cap I \neq \emptyset$。逆に、$u\in U(R)\cap I$ とする。このとき、$r\in R$ とすと、
$$r = r(u^{-1}u) = (ru^{-1})u \in RI \subset I.$$
従って、$R \subset I$。よって、$I = R$。
\end{itemize}

\begin{prop}\label{prop:field}
$R$ を環としたとき、次は、同値。\\
$R$ は、斜体 $\Leftrightarrow$ $R$ の左 {\rm [右]} イデアルは、$0$ と $R$ のみ。
\end{prop}
\proof
($\Rightarrow$)  $I$ を $0$ とは異なる $R$ の左イデアルとする。$a\in I - \{0\}$ とすると、$a\in U(R)$。従って、上の注より $I = R$。

($\Leftarrow$) $a\neq 0$ とすると、$a\in Ra$ より、$Ra$ は $0$ でない左イデアルだから、仮定より $1\in R = Ra$。従って、$R$ の元 $b$ で、$ba = 1$ となるものがある。特に、$b\neq 0$ だから、同様にして、$R = Rb$。特に、$R$ の元 $c$ で、$cb = 1$ となるものがある。すると、
$$c = c1 = c(ba) = (cb)a = 1a = a$$
だから、$ab = ba = 1$。よって、$R$ の $0$ 以外の元は、すべて、単元である。従って、$R$ は斜体である。
\qed

\medskip
順序集合 $X$ が、任意の空でない部分集合に最小元を持つとき、整列集合 (well-ordered set) という。

\begin{definition}
\begin{enumerate}
\item 任意のイデアルが、単項である整域を単項イデアル整域 (PID : principal ideal domain) と言う。
\item 整域 $R$ から、整列集合 (well-ordered set) $X$ への写像 $\rho : R \rightarrow X$ があって、次の二つの条件を満たすとき、$R$ はユークリッド整域 (Euclidean domain) であると言う。
	\begin{enumerate}
	\item $0\neq a\in R \Rightarrow \rho(0) < \rho(a)$.
	\item $a, b\in R$ $(a\neq 0)\Rightarrow  b = aq + r,\;\rho(r) < \rho(a)$ となる $q, r\in R$ がある。
	\end{enumerate}
\end{enumerate}
\end{definition}

\begin{thm} \label{thm:euclid-pid}
ユークリッド整域は、単項イデアル整域である。
\end{thm}
\proof
$R$ をユークリッド整域、$I$ を $R$ の イデアルとする。$I = 0$ ならば、明らかに、単項イデアルだから、$I\neq 0$ とする。
$$\emptyset \neq \{\rho(x)\mid 0\neq x\in I\}\subset X$$
の最小元を、$\rho(a)$ $a\in I$ とする。ここで、$b\in I$ とすると、$b = aq + r$、$\rho(r) < \rho(a)$ となる、$q, r\in R$ がある。$r = b - aq \in I$ だから、$a$ の取り方から、$r = 0$ を得、$b\in Ra$。$b$ は任意だから、$I = Ra$、すなわち、すべてのイデアルは単項である。
\qed

\begin{eg}
\begin{enumerate}
\item $\rho : \bZ \rightarrow \{0\}\cup \bN$ を $\rho(a) = |a|$ によって定義すると、$\bZ$ はユークリッド整域になる。特に、定理~\ref{thm:euclid-pid} より、$\bZ$ は、単項イデアル整域である。実は、$\bZ$ においては、イデアルであることと、部分加群であることは同じであるから、単項イデアル整域であることは、単に、$\bZ$ の部分群が巡回群であることを主張しているに過ぎない。
\item $K$ を体とする。$\rho : K[x] \rightarrow \{-\infty, 0\}\cup \bN$ を $\rho(f) = \deg f$ によって定義すると、定理~\ref{thm:poly-euclid} により、$K[x]$ はユークリッド整域になる。特に、定理~\ref{thm:euclid-pid} より、$K[x]$ は単項イデアル整域である。
\end{enumerate}
\end{eg}

\newpage
\section{準同型定理}
\begin{definition}
環 $R$ から、$R'$ への写像 $f : R\rightarrow R'$ が、
$$f(a + b) = f(a) + f(b), \; f(ab) = f(a)f(b), \; f(1_R) = f_{R'}$$
を満たすとき、$f$ を $R$ から、$R'$ への(環)準同型 ((ring) homomorophism) と言う。$f$ が全単射の時同型と言い、$R\simeq R'$ と書く。
\end{definition}

\begin{eg}
$I$  を、環 $R$ の両側イデアルで、$(R\neq I)$ とするとき、
$$f : R \longrightarrow R/I, \; (a\mapsto a + I)$$
は、環準同型で、全射である。
\end{eg}

\begin{definition}
環 $R$ の部分集合 $S$ が次の条件
$$a, b\in S \Rightarrow a - b\in S, \; ab\in S, \; 1_R\in S$$
を満たすとき、$S$ は、$R$ の部分環 (subring) であるという。また、$R$ は、$S$ の拡大環 (extension ring) であるという。
\end{definition}

\begin{ex}
部分環は、環である。
\end{ex}

\begin{prop} \label{prop:ringhomo}
$f : R \rightarrow R'$ を環準同型とする。
\begin{itemize}
\item[$(1)$] $\ker f = \{a\in R\mid f(a) = 0\}$ は、両側イデアル。
\item[$(2)$] $\im f = \{f(a) \mid a\in R\}$ は、$R'$ の部分環。
\end{itemize}
\end{prop}
\proof
練習問題 3.2.
\qed

\begin{thm} (準同型定理)
$R$、$R'$ を環、$f: R \rightarrow R'$ を環準同型とすると、
$$R/\ker f \simeq \im f.$$
\end{thm}
\proof
命題~\ref{prop:ringhomo} より、$R/\ker f$ も、$\im f$ も、環。群の準同型定理より、
$$\bar{f} : R/\ker f \rightarrow \im f,\; (a + \ker f \mapsto f(a))$$
は、well-defined で、加群としての同型写像。
$$\bar{f}((a+\ker f)(b+\ker f)) = \bar{f}(ab + \ker f) = f(ab) = f(a)f(b) = \bar{f}(a+\ker f)\bar{f}(b+\ker f)$$
$$\bar{f}(1_{R/{\sker} f}) = \bar{f}(1+\ker f) = f(1_R) = 1_{R'}.$$
よって、$\bar{f}$ は、環として同型。
\qed

\medskip
上の証明で、群の準同型定理を用いたが、そこでの鍵は、以下の同値であった。
$$f(a) = f(b) \Leftrightarrow f(a-b) = 0 \Leftrightarrow a-b\in \ker f \Leftrightarrow a+\ker f = b + \ker f$$
これは、上で定義された $\bar{f}$ が、well-defined かつ全単射であることを示している。

\begin{eg}
$K\subset L$  を体、$\alpha\in L$ とする。$\phi:K[x] \to L \;(f(x) \mapsto f(\alpha))$ を環準同型とする。$\im\phi$  を $K[\alpha]$ と書く。すなわち、$K[\alpha] = \{f(\alpha)\mid f(x)\in K[x]\}$。すると、準同型定理により、$K[x]/\ker\phi\simeq K[\alpha]$ となるが、$\ker\phi$  は、単項イデアル整域 $K[x]$ のイデアルだから、ある $p(x)\in K[x]$ によって、$\ker\phi = K[x]p(x) = (p(x))$ と書ける。$p(x) = 0$ すなわち $\ker\phi = 0$ の時、$\alpha$ を $K$ 上超越的な元と呼ぶ。$p(x)\neq 0$ の時は、$p(x)$ として、モニック(最高次の係数が $1$ のもの)をとる事が出来る。実は、$p(x)$ は、$\ker\phi$ の $0$ でない多項式の内で、次数が最小で、モニックなものとして、一意的に決まる。定理~\ref{thm:euclid-pid} の証明参照。このとき、$\alpha$ を $K$ 上代数的な元、$p(x)$ を $\alpha$ の $K$ 上の最小多項式という。

以下 $K = \bQ$、$L = \bC$ とする。$\mbox{e}$、$\pi$ は、$\bQ$ 上超越的な元である。($\bQ$ 上超越的な元を超越数 (transcendental number) と呼ぶ。一般的には、超越数であることを証明することは、とても難しい。

$\alpha = \sqrt{-1}$ とすると、$\ker\phi = (x^2 + 1)$ であり、$\bQ[\sqrt{-1}] = \{a + b\sqrt{-1}\mid a,b\in \bQ\}$ が環であることもこれより分かった。次回には、これが体になることも分かる。

$\alpha = {}^3\!\!\sqrt{2}$ とすると、$\ker\phi = (x^3 -2)$ であり、$\bQ[{}^3\!\!\sqrt{2}] = \{a + b{}^3\!\!\sqrt{2} + c({}^3\!\!\sqrt{2})^2\mid a,b\in\bQ\}$ となる。
\end{eg}

\begin{eg}
$A\in\mat_n(\bC)$、$\psi:\bC[x]\to \mat_n(\bC)\;(f(x)\mapsto f(A))$ とする。Hamilton-Cayley の定理により、$\ker\psi\supset (\det(xI - A))\neq 0$。よって、monic な多項式によって、$\ker\psi = p_A(x)$ と書ける。この $p_A(x)$ を最小多項式という。上の事から $p_A\mid \det(xI - A)$ である。例えば、
$$A = \left(\begin{array}{cc} 1 & 0 \\ 0 & -2\end{array}\right),\; B = \left(\begin{array}{cc} 0 & 1 \\ -4 & 4\end{array}\right),\;C = \left(\begin{array}{cc} 2 & 0 \\ 0 & 2\end{array}\right)$$
としたとき、それぞれの最小多項式は、
$$p_A(x) = (x-1)(x+2), \; p_B(x) = (x-2)^2,\;p_C(x) = x-2$$
である。
\end{eg}

\newpage
\section{素イデアルと極大イデアル}
$R$ を可換環、$I$ をイデアルとする。このとき、剰余環 $R/I$ が、整域や体となるイデアル $I$ の満たすべき条件を考える。

\begin{definition}
\begin{itemize}
\item[$(1)$] 可換環 $R$ のイデアル $I (\neq R)$ について、
$$ab\in I \longrightarrow a\in I \mbox{ または } b\in I$$
が成立するとき、$I$ を素イデアル (prime ideal) という。
\item[$(2)$] 可換環 $R$ のイデアル $I (\neq R)$ について、
$$I \subset J : \mbox{ $R$ のイデアル} \longrightarrow I = J \mbox{ または } J = R$$
が成立するとき、$I$ を極大イデアル (maximal ideal) という。
\end{itemize}
\end{definition}

\begin{thm} \label{thm:R/I}
$R$ を可換環、$I$ をそのイデアルとする。
\begin{itemize}
\item[$(1)$] $R/I$ は整域 $\Leftrightarrow$ $I$ は素イデアル。
\item[$(2)$] $R/I$ は体 $\Leftrightarrow$ $I$ は極大イデアル。
\item[$(3)$] 極大イデアルは、素イデアル。
\end{itemize}
\end{thm}
\proof
$(1)$ $R/I$ が整域であることは、以下のことと同値である。
\begin{eqnarray*}
\lefteqn{\bar{a}, \bar{b}\in R/I, \;\bar{a}\bar{b} = \bar{0} \rightarrow \bar{a} = \bar{0}\mbox{ または }\bar{b} = \bar{0}}\\
&\Leftrightarrow& a,b\in R,\; (a+I)(b+I) = ab+I = I \rightarrow a+I = I \mbox{ または }b+I = I\\
&\Leftrightarrow& a,b\in R,\; ab\in I \rightarrow a\in I \mbox{ または }b\in I\\
& & (x+ I = y+I \leftrightarrow x-y\in I\mbox{ に注意})
\end{eqnarray*}

$(2)$ 命題~\ref{prop:field} により、$R/I$ が体であることと、$R/I$ の $0$ でないイデアルは、$R/I$ のみであることは同値である。これは、言い換えると、$R$ のイデアル $J$ で $I$ を真に含むものは、$R$ に限られるということと同値であるから(練習問題参照)、$I$ が $R$ の極大イデアルであることと同値である。

$(3)$ $I$:極大イデアル $\Leftrightarrow$ $R/I$:体 $\Rightarrow$ $R/I$:整域 $\Leftrightarrow$ $I$:素イデアル。
\qed

\medskip
\note
$R$ を可換環とすると、上の定理から、零イデアル $(0)$ が素イデアルであることと、$R$ が整域であることが同値であり、また、$(0)$ が極大イデアルであることと、$R$ が体であることが同値である。

\begin{prop} \label{prop:pid}
$R$ を単項イデアル整域 {\rm (PID)}、$I$ を $R$ のイデアルで $I\neq (0)$ なるものとする。このとき、次は同値。
$$I\mbox{:素イデアル} \Leftrightarrow I\mbox{:極大イデアル}$$
\end{prop}
\proof
定理~\ref{thm:R/I} により、極大イデアルは、常に素イデアルだから、$I = (a) \neq (0)$ を素イデアルとして、$I$ が極大イデアルであることを示す。$J$ を $I$ を真に含む $R$ のイデアルとする。$R$ は、単項イデアル整域だから、$J = (b)$ とおける。$a\in (a) = I \subset J = (b)$ だから、$a = bc$ となる $c\in R$ が存在する。$I = (a)$ は、素イデアルだから $b\in I$ または $c\in I$。$b\in I$ とすると、$J = (b) \subset (a) = I$ となり $J$ が $I$  を真に含むイデアルであることに反するから、$c\in I = (a)$。すなわち、$c = ad$ となる$d\in R$ が存在する。これより、$$a = bc = bad = abd\rightarrow a(bd - 1) = 0$$
を得る。$a\neq 0$ だったから $bd = 1$ すなわち $R = (1) \subset (b) = J$ となり $J = R$ となるから、$I$ は極大イデアルである。
\qed

\begin{prop} \label{prop:integer}
$n$ を有理整数環 $\bZ$ の零でない元とする。ことのき、次は同値。
$$(n)\mbox{:極大イデアル} \Leftrightarrow (n)\mbox{:素イデアル}\Leftrightarrow n\mbox{ は素数}$$
\end{prop}
\proof
まず、$(n) \subset (m)$ は、$m\mid n$ と同値であることに注意する。これより、$(m) = (n)$ であることと、$m = \pm n$ は同値であることが分かる。さて、「$(m)$ が極大であること」と、「$(m)\subset (n)$ ならば、$(n) = (m)$ または $(n) = (1)$ であること」とは、同値である。これより、$n$ の約数は、$\pm n$ であるか、または $\pm 1$ であるかのどちらかであることを得る。極大イデアルは、$(1) = \bZ$ とは異なるから、
$$(n)\mbox{:極大イデアル} \Leftrightarrow  n\mbox{ は素数}$$
$\bZ$ は、単項イデアル整域であるから、命題~\ref{prop:pid} より、零でないイデアルが極大イデアルであることと、素イデアルであることは、同値であることが分かる。
\qed

\medskip
\note
この命題により、$\bZ_n = \bZ/(n)$ が体であることと、整域であることと、$n$ が素数であることは、全て同値であることもわかった。

\begin{definition}
整域 $R$ 上の次数が $1$ 以上の多項式 $f(x)$ は、$R[x]$ において、$f(x) = g(x)h(x)$ $\deg g>0$、$\deg h>0$ と分解されるとき($R$ 上)可約、そうでないとき、($R$ 上)既約であるという。
\end{definition}

\begin{prop}\label{prop:poly}
$K$ を体とし、$f(x)\in K[x]$ を零でない多項式とする。このとき、次は同値。
$$(f(x))\mbox{:極大イデアル} \Leftrightarrow (f(x))\mbox{:素イデアル}\Leftrightarrow f(x)\mbox{ は既約}$$
\end{prop}
\proof
$f(x)$ が定数の時は、上の3つのどの条件も満たさないから考えなくて良い。そこで、$\deg f(x)\geq 1$ とする。$K[x]$ は、単項イデアル整域であるから、$(f(x))$ が極大イデアルであることと、素イデアルであることは、同値である。このことと、$f(x)$ が既約であることが同値であることを示す。

$f(x)$ を可約とする。すなわち、$f(x) = g(x)h(x)$、$\deg g(x)>1$、$\deg h(x)>1$ とする。すると、
$$(f(x)) \subset (g(x)) \subset K[x]$$
でどちらも等号は成り立たない。$U(K[x]) = U(K) = K^*$ だから、練習問題より以下が同値であることから明か。
$$(f_1(x)) = (f_2(x)) \Leftrightarrow f_1(x) = cf_2(x)\;(c\in K^*).$$

逆に、$(f(x))$ は極大イデアルではないとする。$(g(x))$ を $(f(x))$ を真に含みかつ $K[x]$ とは異なるイデアルとする。すると、$f(x) = g(x)h(x)$ とかけ、条件から、$f(x)$ は、可約であることが分かる。
\qed

\begin{eg}
$x^2+1$、$x^2-2$ が $\bQ$ 上既約であることは簡単に確かめられるから、$(x^2+1)$、$(x^2-2)$ は、$\bQ[x]$ の極大イデアルであり、従って、$\bQ[\sqrt{-1}]\simeq \bQ[x]/(x^2+1)$、$\bQ[\sqrt{2}]\simeq \bQ[x]/(x^2-2)$ は、体であることが分かる。
\end{eg}

次数の高い多項式について、既約かどうかはどのように判定すればよいのだろうか。実は、一般には非常に難しい。しかし、次の判定法は有効である。

\begin{prop}{\rm [Eisenstein の既約性判定法]}\label{prop:eisenstein}
$p$ を素数、$f(x) = a_nx^n + \cdots + a_1x + a_0\in \bZ[x]$ で以下の条件を満たすとする。
$$a_n\not\equiv 0 \;(\mod p),\;a_{n-1}\equiv \cdots \equiv a_1 \equiv a_0 \equiv 0\;(\mod p),\;a_0 \not\equiv 0\;(\mod p^2)$$
このとき、$f(x)$ は、$\bZ$ 上既約である。
\end{prop}
\proof
可約として矛盾を導く。
$$f(x) = g(x)h(x),\;r = \deg g>0,\;s = \deg h>0,$$
$$g(x) = b_rx^r + \cdots + b_0,\;h(x) = c_sx^s + \cdots + c_0$$
とする。$a_0 = b_0c_0$ は仮定より $p$ で割り切れるが、$p^2$ では割り切れない。従って、$p$ は、$b_0$ は割らないが、$c_0$ は割ると仮定する。一方、$a_n = b_rc_s$ は仮定から $p$ で割れないから、$c_s$ も $p$ で割れない。$c_0$ は $p$ で割れるとしているから、今 $i$ を $c_i$ が $p$ で割り切れない最小の整数とする。従って、
$$c_0 \equiv c_1 \equiv \cdots \equiv c_{i-1} \equiv 0 \not\equiv c_i\;(\mod p).$$
すると、
$$a_i = b_0c_i + b_1c_{i-1} + \cdots + b_ic_0\equiv b_0c_i \not\equiv 0\;(\mod p).$$
である。仮定から $n = i < s = \deg h$ となり、これは、矛盾である。従って、$f(x)$ は既約である。
\qed

\medskip
\note 
この命題は、$\bZ$ 上既約かどうかの判定法であるが、実は、練習問題にもあるように、ガウスの補題(命題~\ref{prop:gauss'lemma})といわれるものにより $\bQ$ 上既約であることも分かる。

例えば、$x^n - 2$、$x^3 - 3x^2 - 9x -6$ は、$\bZ$ 上(そして、$\bQ$ 上)既約である。

\newpage
\section{環の直和}
\begin{definition}
環 $R_1, R_2, \ldots, R_n$ が与えられたとき、直積
$$R = R_1\times R_2 \times \cdots \times R_n = \{(a_1,a_2,\ldots, a_n)\mid a_i\in R_i,\;i = 1,\ldots, n\}$$
に加法と乗法を次のように定義する。
\begin{itemize}
\item[加法:] $(a_1,\ldots, a_n) + (b_1,\ldots,b_n) = (a_1 + b_1,\ldots, a_n+b_n)$
\item[乗法:] $(a_1,\ldots, a_n) \cdot (b_1,\ldots,b_n) = (a_1\cdot b_1,\ldots, a_n\cdot b_n)$
\end{itemize}
このとき、$R$ は環になる。$R$ を $R_1,\ldots, R_n$ の直和といい、以下のように書く。
$$R = R_1\oplus R_2 \oplus \cdots \oplus R_n.$$
\end{definition}

\note
$1_R = (1_{R_1},1_{R_2},\ldots, 1_{R_n})$、$0_R = (0_{R_1},0_{R_2},\ldots, 0_{R_n})$ である。

また、$R_i^* = \{(0,\ldots, 0,a,0,\ldots, 0)\mid a\in R_i\}$ (第 $i$ 成分以外は、$0$)とすると、$R_i^*$  は、$R$ の両側イデアルである。

\medskip
可換環 $R$ の二つのイデアル $I,J$ が $I+J = R$ を満たすとき、$I$ は $J$ と互いに素であるという。すなわち次の同値条件を満たすことである。
$$I+J = R\Leftrightarrow x+y = 1\mbox{ となる }x\in I, \;y\in J \mbox{ が存在する。}$$

\medskip
$\bZ$ においては、$(m)$ と $(n)$  が互いに素な事と、$(m,n) = 1$ すなわち、$m$ と $n$ の最大公約数が $1$ であることは同値である。実際、$x+y = 1$ となる $x\in (m), \;y\in (n)$  が存在するということは、$am+bn = 1$ となる $a,b\in\bZ$ が存在することであり、このことは、$(m,n) = 1$  と同値であるからである。

\medskip
「3 で割って 1 余り、10 で割って 3 余り、7 では割り切れ、13 で割ると 11 余るような数はあるだろうか。またあるならばそれをすべて求めることが出来るか。」という種類の問題は、古くからいろいろと考えられていたようで孫子の「兵法」に軍隊の編成の問題から議論されていることなどから、この問題を取り扱った次の定理は中国剰余定理 (Chinise Remainer's Theorem) と呼ばれているとのことである。

\begin{thm} {\rm [中国剰余定理]}\label{thm:crt}
$R$ を可換環、$I_1, I_2, \ldots, I_n$ をどの二つも互いに素なイデアル(すなわち、$i\neq j$  のとき、$I_i + I_j = R$)とする。$a_1, a_2, \ldots, a_n\in R$  を任意の元とするとき、$x\equiv a_i\;(\mod I_i)$ がすべての $i = 1, 2, \ldots, n$ に対して、成り立つ元 $x\in R$ が存在する。
\end{thm}
\proof
\underline{$n = 2$ のとき}\quad 仮定より、$1 = c_1 + c_2$ となる $c_1\in I_1,\;c_2\in I_2$ がある。そこで、$x = a_1c_2 + a_2c_1$ とおくと、$(\mod I_1)$ で、
$$x\equiv a_1c_2 + a_2c_1 \equiv a_1c_2 \equiv a_1(1-c_1) \equiv a_1 - a_1c_1 \equiv a_1$$
となる。$x \equiv a_2 \;(\mod I_2)$ も同様にして得る。

\underline{$n > 2$ のとき}\quad まず、各 $i$ について、次の性質を満たす $x_i\in R$ が存在することを示す。
$$x_i \equiv 1 \;(\mod I_i),\;j\neq i \mbox{ の時は } x_i\equiv 0\;(\mod I_j).$$
記号を見やすくするため、$i = 1$ のときを考える。$j\geq 2$ については、$I_1 + I_j = R$ だから、$c_1^{(j)}+c_j = 1$ となる、$c_1^{(j)}\in I_1$、$c_j\in I_j$ がある。すべてを掛け合わせると、
$$1 = \prod_{j=2}^n(c_1^{(j)}+c_j) \equiv c_2\cdots c_n\;(\mod I_1)$$
だから、$1 - c_2\cdots c_n = c_1$ とおくと、$c_1\in I_1$ である。とくに、
$R = I_1 + I_2\cdots I_n$ すなわち、二つのイデアル $I_1$、$I_2\cdots I_n$ は互いに素であることが分かる。上記 $n=2$ の時は、既に示してあるから、$x_1\in R$ で、
$$x_1\equiv 1 \;(\mod I_1),\;x_1\equiv 0\;(\mod I_2\cdots I_n)$$
を満たすものが存在することが分かる。ところが、$j\geq 2$ に対して、$I_2\cdots I_n \subset I_j$ であるから、$x_1\equiv 0\;(\mod I_j)$ でもある。これで最初の主張が示された。

今、各 $i$ について、$x_i$ をとり、$x = a_1x_1 + \cdots + a_nx_n$ とおくと、
\begin{eqnarray*}
x & \equiv & a_1x_1 + \cdots + a_nx_n \;(\mod I_i)\\
& \equiv & a_ix_i \;(\mod I_i)\\
& \equiv & a_i \;(\mod I_i)
\end{eqnarray*}
となり、求めるものが得られた。
\qed

\newpage
\section{商環}
この節では、可換環に逆元をつけ加えてどれぐらい体に近く出来るかを考える。

\begin{definition}
可換環 $R$ の部分集合 $S$ が次の条件
$$(i)\;a,b\in S\rightarrow ab\in S\qquad (ii)\;1\in S,0\not\in S$$
を満たすとき、$S$ は $R$ の乗法的部分集合、積閉集合 (multiplicative subset) と言う。
\end{definition}

\begin{eg}
\begin{enumerate}
\item $R$ の非零因子全体(零因子以外の元すべて)は、乗法的部分集合である。
\item $P$ を $R$ の素イデアルとしたとき、$R-P$ は、乗法的部分集合である。
\end{enumerate}
\end{eg}

$R$ を可換環 $S$ を乗法的部分集合とする。

$R\times S$ に次のような関係を定義する。
$$(a,s)\sim (a',s') \Leftrightarrow (as'-a's)t = 0 \mbox{ となる }t\in S\mbox{ が存在する。}$$
すると、これは同値関係になる。$(a,s)$ を含む同値類を $a/s$ で表し、同値類全体を $S^{-1}R$ で表す。$S^{-1}R$ に加法と、乗法を次のように定義する。
\begin{itemize}
\item[加法:] $(a_1/s_1) + (a_2/s_2) = (a_1s_2 + a_2s_1)/s_1s_2$
\item[乗法:] $(a_1/s_1)(a_2/s_2) = (a_1a_2/s_1s_2)$
\end{itemize}
これらの和・積は、$S^{-1}R$ の表し方によらず、一意的に定まり、可換環になる。これを$R$ の $S$ による商環 (quotient ring) という。

\note
\begin{enumerate}
\item $0_{S^{-1}R} = 0/1$、$1_{S^{-1}R} = 1/1$、$-(a/s) = (-a)/s$  であり、$s\in S$ ならば、$s/1\in U(S^{-1}R)$ である。
\item $S$ が零因子を含まないときは、$a/s = a'/s' \Leftrightarrow as'-a's = 0$。
\item $\phi_S:R \to S^{-1}R\;(a\mapsto a/1)$ を自然な準同型という。\\
$\phi_S$:単射 $\Leftrightarrow$ $S$ は、零因子を含まない。
\end{enumerate}

\begin{eg}
\begin{enumerate}
\item $S$ を $R$ の非零因子全体の時、$S^{-1}R$ を $R$ の全商環 (ring of total quotients) という。
\item $R$ が整域のときは、$R$ の全商環は体になる。これを商体 (quotient field) と呼び、$\cQ(R)$ とかく。
	\begin{enumerate}
	\item $\cQ(\bZ) = \bQ$、有理整数環の商体は、有理数体である。
	\item $\cQ(K[x_1,\ldots, x_n]) = K(x_1,\ldots, x_n) = \{f/g\mid f,g\in K[x_1,\ldots x_n], \;g\neq 0\}$ であり、これを有理関数体という。
	\item $P$ を可換環 $R$ の素イデアル、$(R-p)^{-1}R$ を $R_{P}$ と書き、$R$ の $P$ による局所化 (localization) と呼ぶ。
	\end{enumerate}
\end{enumerate}
\end{eg}

\begin{definition}
可換環 $R$ がただ一つの極大イデアル $M$ を持つとき、$R$ は、局所環 (local ring) であるという。
\end{definition}

\note
体は、$(0)$ がただ一つの極大イデアルであるから、局所環である。

$R$ を局所環、$M$ をその極大イデアルとする。$I$ を $R$ とは異なるイデアルとすると、Zorn の補題を用いることにより、$I$ を含む極大イデアルが一つ存在する。$R$ は、局所環であるから $I\subset M$ であることが分かる。すなわち、$M$ は、$R$ の真のイデアルをすべて含む。

\begin{prop} \label{prop:local_ring}
可換環 $R$ について、次の二つは、同値。{\rm [(1) $\Rightarrow$ (2) には、選択公理が必要]}
\begin{itemize}
\item[$(1)$] $R$ は局所環。
\item[$(2)$] $R-U(R)$ は、$R$ のイデアル。
\end{itemize}
\end{prop}
\proof
$(1)\Rightarrow (2)$\quad $M$ を $R$ のただ一つの極大イデアルとする。$M\neq R$ だから、$M\cap U(R) = \emptyset$ すなわち、$M\subset R - U(R)$。ここで、$a\in R- U(R)$ とすると、$Ra \neq R$ だから、$a\in Ra \subset M$。よって、$R - U(R)\subset M$。従って、$M = R - U(R)$ であり、これはイデアルである。

$(2)\Rightarrow (1)$\quad $J$ を $R\neq J$ なる $R$ のイデアル、$I = R - U(R)$ とする。このとき、$J\cap U(R) = \emptyset$  だから、$J\subset I$。よって $I$ は $R$ のただ一つの極大イデアルである。
\qed

\begin{prop}
$P$  を可換環 $R$ の素イデアルとすると、局所化 $R_P$ は局所環で、
$$P' = \{a/s\mid a\in P,\;s\not\in P\}$$
がそのただ一つの極大イデアルである。
\end{prop}
\proof
\underline{$P'$ は、$R_P$ のイデアルである。}

\pf
$(a/s)+(b/t) = (at+bs)/st$、$a,b\in P,\;s,t\not\in P$ とすると、$at+bs\in P$、$st\not\in P$ だから、$(at+bs)/st\in P'$。同様に、$r\in R$ の時、$(r/t)(a/s) = ar/ts\in P'$。従って、$P'$ は $R_P$ のイデアルである。

\smallskip
\underline{$a/s\in P'\Leftrightarrow a\not\in P$.}

\pf
($\Rightarrow$) $a\in P$ ならば、$a/s\in P'$ だから、明らか。

($\Leftarrow$) $a/s\in P'$ とすると、$a/s = a'/s'$ となる $a'\in P$、$s'\not\in P$ が存在する。従って、$(as' - a's)t = 0$ を満たす $t\not\in P$ が存在する。これより、$as't = a'st\in P$ だから、仮定より $a\in P$ を得る。

\smallskip
\underline{$R_P - P' = U(R_P)$}

\pf
(`$\subset$' であること。) $a/s\not\in P'$ とすると、$a\not\in P$ だから、$s/a\in R_P$。すなわち、$a/s\in U(R_P)$。

(`$\supset$' であること。) $1\not\in P$ だから、$1/1\not\in P'$。$a/s\in U(R_P)\cap P'$ とすると、$a\in P$ であり、かつ、$(a/s)(b/t) = 1/1$ となる、$b\in R$、$t\not\in P$ が存在する。これより、ある $t'\not\in P$ により、$abt' = stt'$ となるが、この式の右辺は、$P$ に属さず、左辺は、$P$ に属することになり矛盾。従って、$U(R_P)\cap P' = \emptyset$。これより、$R_P - P' = U(R_P)$ を得る。
\qed

\newpage
\section{一意分解整域}
\subsection{一意分解整域と単項イデアル整域}
$R$ を整域、$a,b\in R$ とする。
\begin{itemize}
\item $(a) \subset (b) \Leftrightarrow a = bc$ となる $c\in R$ がある。このとき、$b\mid a$ と書く。
\item $(a) = (b) \Leftrightarrow a = bu$ となる $u\in U(R)$ がある。このとき、$a\approx b$ と書き同伴という。
\item $R$ の元 $p\neq 0$ が正則元でなくかつ、$p = uv \to u\in U(R)$ 又は、$v\in U(R)$ の時、$p$ を素元という。
\end{itemize}

\begin{definition}
整域 $R$ が次の二つの条件を満たすとき、$R$ を一意分解整域 (UFD = Unique Factorization Domain) であるという。
\begin{itemize}
\item[$(i)$] $a\in R$ を零でない単元でもない元とする。$a = p_1p_2\cdots p_r$ ($p_i$ は素元) と書ける。
\item[$(ii)$] $a = p_1p_2\cdots p_r = q_1q_2\cdots q_s$ ($p_i,q_j$ は素元)ならば $r = s$ で番号を付け替えれば $p_i\approx q_i$。
\end{itemize}
\end{definition}

\begin{prop} \label{prop:ufd:prime}
$R$ を整域、$0\neq p\in R$ とする。
\begin{itemize}
\item[$(1)$] $(p)$ が素イデアル ならば、 $p$ は素元。
\item[$(2)$] $R$ が {\rm UFD} ならば $(p)$ が素イデアルことと、$p$ は素元であることは同値。
\end{itemize}
\end{prop}
\proof
$(1)$ $p = ab$ とする。仮定より、$a\in (p)$ または $b\in (p)$。$a\in (p)$ とする。
$(a)\subset (p) = (ab) \subset (a)$ だから、$a\approx p$ で $p = au$、$u\in U(R)$ と書ける。$a(b - u) = p - p = 0$ で、$R$ は整域だから $b = u\in U(R)$。

$(2)$ $p$ を素数とする。$ab\in (p)$ とすると、$a$ または $b\in U(R)$ のときは明らか。
$ab = pc$、$a = p_1\cdots p_r$、$b = q_1\cdots q_s$、$c = v_1\cdots v_t$ を素元分解とする。
$$p_1\cdots p_rq_1\cdots q_s = pv_1\cdots v_t$$
素元分解の一意性より $p\approx p_i$ 又は $p\approx q_j$。そこで、
$p\approx p_i$ とすると、$a = p_1\cdots p_r\in (p_i) = (p)$。$p\approx q_j$ とすると、$b = q_1\cdots q_s\in (q_j) = (p)$。
\qed

\medskip
\note
この命題は、ある環 $R$ が一意分解整域ではないことを示すためにも用いられる。すなわち、素元ではあるが、それで生成されたイデアルが、素イデアルではない元の存在が示されればそれで良い。

\begin{prop} \label{prop:pid:prime}
$R$ を単項イデアル整域、$p\neq 0$ とすると次は同値。
\begin{itemize}
\item[$(1)$] $p$ は素元。
\item[$(2)$] $(p)$ は素イデアル。
\item[$(3)$] $(p)$ は極大イデアル。
\end{itemize}
\end{prop}
\proof
命題~\ref{prop:pid} により $(2)\Leftrightarrow (3)$、また、命題~\ref{prop:ufd:prime} により $(2)\Rightarrow (1)$ も示してあるから、$(1)\Rightarrow (3)$ を示せばよい。
$(p) \subset I = (q) \subset R$ とすると、$p = qa$ と書ける。仮定より、$q$ が単元か、$a$ が単元。それぞれ、$(q) = R$ または、$(p) = (q)$ となる。従って、$(p)$ は極大イデアルである。
\qed

\begin{thm} \label{thm:pidisufd}
単項イデアル整域は一意分解整域である。
\end{thm}
\proof
$R$ を単項イデアル整域とし、$0\neq a\in R - U(R)$ とする。このとき、$(a) \neq R$ だから $(a)$ を含む極大イデアル $(p_1)$ が存在する。命題~\ref{prop:pid:prime} より $p_1$ は素元である。
$(a)\subset (p_1)$ より、$a = p_1a_1$ と表すことが出来、$p_1\not\in U(R)$ より、$(a_1)$ は、$(a)$ を真に含む。$a_1\not\in U(R)$ ならば素元 $p_2$ が存在して、$a_1 = p_2a_2$、($a = p_1p_2a_2$) と書くことが出来る。この様にして順に $a_i$ を取っていくとき、正則元でない限りにおいて、真に増加する列
$$(a) \subset (a_1) \subset (a_2) \subset \cdots \subset (a_i)\subset \cdots$$
がつくれる。
$\bigcup_{i=1}^\infty (a_i)$  は $R$ のイデアルだから、$\bigcup_{i = 1}^\infty (a_i) = (d)$ と書ける。従って、ある $i$ について、$d\in (a_i)$ となるから、$(a_i) = (a_{i+1})$ となり真に増加することはない。よってある $r$ について $a_r$ は正則元、すなわち、$p_ra_r$ は素元で、$a = p_1p_2\cdots (p_ra_r)$。

\underline{一意性}:
$a = p_1p_2\cdots p_r = q_1q_2\cdots q_s$、$r\leq s$ とし、$r$ に関する帰納法を用いる。$q_1q_2 \cdots q_s = a\in (p_1)$ で、$(p_1)$ は素イデアルだから、$q_i\in (p_1)$ となる $i$ がある。しかし、$(q_i)\subset (p_1)$ で、どちらも極大イデアルであるから、$(q_i)\approx (p_1)$ である。番号を付け替え、$q_1 = p_1u$、$u\in U(R)$ とすると、
$$p_1p_2\cdots p_r = p_1uq_2\cdots q_s$$
を得るから、$p_2\cdots p_r = uq_2\cdots q_s$。帰納法により、$r = s$ かつ、番号の付け替えにより、$p_i\approx q_i$ となることが分かる。
\qed

\medskip
これにより、ユークリッド整域は、単項イデアル整域であり、単項イデアル整域は、一意分解整域であることが分かった。しかし、これだけでは、$\bZ[x]$ や、$\bQ[x_1,\cdots, x_n]$ が一意分解整域かどうかは分からない。

\begin{eg}
$\bZ[\sqrt{-5}] = \{a + b\sqrt{-5}\mid a,b\in \bZ\}$ は一意分解整域ではない事を示す。
上でも注意したように、2  は素元であるが、$(2)$ は素イデアルではないことを示す。
\begin{itemize}
\item $\alpha = a +b\sqrt{-5}$ のとき、$N(\alpha) = \alpha\bar{\alpha} = a^2 + 5b^2$ とすると、
$$\alpha\in U(\bZ[\sqrt{-5}]) \Leftrightarrow N(\alpha) = 1 \Leftrightarrow \alpha = \pm 1.$$

\pf 
$\pm 1\in U(\bZ[\sqrt{-5}])$ は明らか。逆に $\alpha\beta = 1$ とすると、
$$1 = N(\alpha\beta) = N(\alpha)N(\beta)$$
だから、$a^2 + 5b^2 = N(\alpha) = 1$。これを満たす $a,b\in \bZ$ を考えると、$b = 0$、$a = \pm 1$ であることが分かる。
\item $2$ は素元。

\pf 
$2 = \alpha\beta$、$N(\alpha)\neq 1$、$N(\beta)\neq 1$、$\alpha = a + b\sqrt{-5}$ とする。
$$4 = N(2) = N(\alpha\beta) = N(\alpha)N(\beta)$$
だから、$a^2 + 5b^2 = N(\alpha) = 2$。しかしこれは不可能である。従って、$N(\alpha) = 1$ 又は、$N(\beta) = 1$ すなわち、$\alpha, \beta$ のうちどちらかは、単元である。
\item $(2)$ は、素イデアルではない。

\pf
$(1+\sqrt{-5})(1-\sqrt{-5}) = 6 \in (2)$。ここで、$1\pm\sqrt{-5}$ のどちらかが、$(2)$ に入るとすると、$1\pm \sqrt{-5} = 2\gamma$ と書いたとき、
$$6 = N(1\pm\sqrt{-5}) = 4N(\gamma)$$
となり、これは不可能である。従って、$1\pm\sqrt{-5}$ どちらも $(2)$ に入らない。これは、$(2)$ が素イデアルではないことを示す。
\end{itemize}
\end{eg}

\newpage
\subsection{一意分解整域上の多項式環}
ここでは、$R$ を一意分解整域、$K = \cQ(R)$ を商体とする。

\begin{itemize}
\item $d$ が、$a_1, \ldots, a_n\in R$ の最大公約元であるとは、以下の2条件を満たすことである。
	\begin{enumerate}
	\item $d\mid a_i$, $i = 1, 2,\ldots, n$。
	\item $c\mid a_i$, $i = 1, 2, \ldots, n$ ならば、$c\mid d$。
	\end{enumerate}
\item $l$ が、$a_1, \ldots, a_n\in R$ の最小公倍元であるとは、以下の2条件を満たすことである。
	\begin{enumerate}
	\item $a_i\mid l$, $i = 1, 2,\ldots, n$。
	\item $a_i\mid m$, $i = 1, 2, \ldots, n$ ならば、$l\mid m$。
	\end{enumerate}
\item $a_1, a_2, \ldots, a_n$ の最大公約元が 1 であるとき、$a_1, a_2, \ldots, a_n$ は、互いに素 (coprime) であるという。
\item $a_0, a_1, \ldots, a_n$ が互いに素である時、$f(x) = a_0 + a_1x + \cdots + a_nx^n\in R[x]$ を原始多項式 (primitive polynomial) という。
\end{itemize}

練習問題にもあるように、$R$ が一意分解整域ならば最大公約元、最小公倍元は存在する。

\begin{thm} \label{thm:poly_ufd}
単項イデアル整域 $R$ 上の多項式環 $R[x_1,x_2,\ldots,x_n]$ は、単項イデアル整域である。
\end{thm}

\begin{lemma} \label{lemma:inhalt}
$f(x)\in K[x]$ とすると、$c\in K$ と、原始多項式 $f_0(x)\in R[x]$ で、$f(x) = cf_0(x)$ となるものがある。この $c$ は $R$ の正則元倍をのぞいて一意的に決まる。これを $I(f)$ と書く。
\end{lemma}
\proof
$f(x) = (b_0/a_0) + (b_1/a_1)x + \cdots + (b_n/a_n)x^n$、$0\neq a_i, b_j\in R$。$m$ を $a_0, a_1, \ldots, a_n$ の最小公倍元、$m = a_ic_i$、$d$ を $b_0c_0, b_1c_1, \ldots, b_nc_n$ の最大公約元、$de_i = b_ic_i$ とする。$e_0, e_1, \ldots, e_n$ は互いに素である。さらに、
\begin{eqnarray*}
f(x) & = & (b_0/a_0) + (b_1/a_1)x + \cdots + (b_n/a_n)x^n\\
	& = & \frac1m(b_0c_0 + b_1c_1x + \cdots + b_nc_nx^n)\\
	& = & \frac dm(e_0 + e_1x + \cdots + e_nx^n)
\end{eqnarray*}
ここで、$c = d/m$、$f_0(x) = e_0 + e_1x + \cdots + e_nx^n$ とおけばよい。

$f(x) = cf_0(x) = c'f'_0(x)$、$f_0(x)$、$f_0'(x)$ は、$R$ 上の原始多項式、$c = b/a$、$c' = b'/a'$、$a$ と $b$、$a'$ と $b'$ は互いに素な $R$ の元とする。
$a'bf_0(x) = ab'f'_0(x)$ だから、それぞれの係数の最大公約元を考えると、最大公約元は、正則元倍をのぞいて、一意に決まり、$f_0(x)$、$f_0'(x)$ はともに原始多項式だから、$a'b = ab'u$ となる $u\in U(R)$ がある。従って、$c = b/a = (b'/a')u = c'u$。
\qed

\medskip
$K$ の2元 $c, c'$ について、$c' = cu$ となる $u\in U(R)$ が存在するとき、$c\approx c'$ と書く。このとき、$f(x)\in K[x]$ について、
\begin{itemize}
\item $f(x)\in R[x] \Leftrightarrow I(f)\in R$。
\item $f(x) \mbox{ が原始多項式 } \Leftrightarrow I(f)\approx 1$。
\end{itemize}

\begin{lemma} \label{lemma:productofprimitive}
\begin{itemize}
\item[$(1)$] 原始多項式の積は原始多項式。
\item[$(2)$] $f(x), g(x)\in K[x]$ ならば、$I(fg)\approx I(f)I(g)$。
\end{itemize}
\end{lemma}
\proof
$(1)$ $f(x) = a_0 + a_1x + \cdots + a_lx^l$、$g(x) = b_0 + b_1x + \cdots + b_mx^m$ を原始多項式、
$$h(x) = f(x)g(x) = c_0 + c_1x + \cdots + a_nx^n,\;p\mid c_i, ;i = 0,1,\ldots, n$$
$p$ は素元、とする。$a_i$ のうち、$p$ で割れない最小の $i$ を $i_0$ とする。また、$b_j$ のうち、$p$ で割れない最小の $j$ を $j_0$ とする。すると、
\begin{eqnarray*}
c_{i_0+j_0} & = & a_0b_{i_0+j_0}  + \cdots + a_{i_0-1}b_{j_0+1}+ a_{i_0}b_{j_0}+ a_{i_0+1}b_{j_0-1} + \cdots + a_{i_0+j_0}b_0\\
& \equiv & a_{i_0}b_{j_0} \;(\mod (p))\\
& \not\equiv & 0\;(\mod (p))
\end{eqnarray*}

$(2)$ $f(x) = I(f)f_0(x)$、$g(x) = I(g)g_0(x)$ で、$f_0(x)$、$g_0(x)$ は原始多項式と書く。すると、$f(x)g(x) = I(f)I(g)f_0(x)g_0(x)$ で、$f_0(x)g_0(x)$ は、$(1)$ より、原始多項式だから、$I(f)I(g)\approx I(fg)$。
\qed

\begin{prop} \label{prop:gauss'lemma}
$f(x)\in R[x]$ に対して、$f(x)$ が $R[x]$ の元として既約であることと、$K[x]$ の元として既約であることは同値である。
\end{prop}
\proof
$f(x)$ が $K[x]$ の元として既約ならば、$R[x]$ の元として既約であることは明らか。$K[x]$ において、$f(x) = g(x)h(x)$、$g(x), h(x)\in K[x]$ とする。ここで、$g(x) = I(g)g_0(x)$、$h(x) = I(h)h_0(x)$、$f_0(x), g_0(x)$ は原始多項式とすると、$f(x) = I(g)I(h)g_0(x)h_0(x)$、$f(x)\in R[x]$ より、$I(g)I(h)\approx I(gh)\in R$。従って、$\deg g_0 = \deg g = 0$ 又は、$\deg h_0 = \deg h = 0$。従って、$K[x]$ においても既約である。
\qed

\begin{lemma} \label{lemma:primeinpoly}
$f(x)$ を $R[x]$ の素元とすると、次のいずれかが成立。
\begin{itemize}
\item[$(i)$] $\deg f = 0$ で、$f$ は $R$ の素元。
\item[$(ii)$] $\deg f>0$ で、$f$ は既約な原始多項式。
\end{itemize}
\end{lemma}
\proof
$U(R[x]) = U(R)$ である事に注意すると、かつ上の $(i),(ii)$ が素元であることは明か。

逆に $f(x)$ を素元とする。$f = gh$ とすると、$g, h$ のいずれかは、$U(R[x]) = U(R)$ の元だから、$f\in R$ 又は、同じことだが $\deg f = 0$ ならば、$f$ は、$R$ の素元である。$\deg f >0$ ならば、$f$ は既約で、かつ $f = I(f)f_0$ より $I(f)\in U(R)$ となり $f$ は原始多項式。従って、この場合は、$(ii)$ が成立する。
\qed

\medskip
{\gt 定理~\ref{thm:poly_ufd} の証明\quad}
$R[x_1,\ldots, x_{n-1},x_n] = (R[x_1,\ldots,x_{n-1}])[x_{n}]$ だから、$n = 1$ の場合、すなわち、$R$ が一意分解整域の時、$R[x]$ が一意分解整域であることを示せばよい。

$0\neq f(x)\in R[x]$ が素元分解可能であることを $\deg f$ に関する帰納法で示す。$\deg f = 0$ の時は、$R$ が一意分解整域であるから、補題~\ref{lemma:primeinpoly} $(i)$ に注意すれば $R[x]$ の素元に分解できることが分かる。$\deg f > 0$ かつ可約の時は、$f = gh$、$\deg g>0, \deg h>0$ と表すと、$\deg g<\deg f$、$\deg h<\deg f$ だから、帰納法の仮定により、$g$、$h$ ともに素元分解できる。従って、$f$ も素元分解できる。そこで既約とする。すると、$f = I(f)f_0$、$f_0$ は原始多項式と書くと、$f_0$ は既約でもあるから、補題~\ref{lemma:primeinpoly} $(ii)$ により素元、後は、$I(f)$ に $R$ における素元分解を適用すれば $R[x]$ における素元分解が得られる。

\underline{一意性: }
$f = p_1\cdots p_kf_1\cdots f_l = q_1\cdots q_mg_1\cdots g_n$ を $f$ の素元分解とし、\\
$p_1, \ldots, p_k, q_1, \ldots, q_m\in R$、$f_1, \ldots, f_l, g_1, \ldots, g_n$ は次数が 1 以上の既約原始多項式とする。すると、$f_1\cdots f_l$、$g_1\cdots g_n$ は 補題~\ref{lemma:productofprimitive} により、ともに原始多項式だから、
$$I(f) \approx p_1\cdots p_k \approx q_1\cdots q_m$$
を得、ある $u\in U(R)$ によって、$up_1\cdots p_k = q_1\cdots q_m$ と書けるから、$R$ が一意分解整域であることより、この部分の一意性は得られる。一方、
$K[x]$ は、体上の多項式環だからユークリッド整域、とくに一意分解整域で $uf_1\cdots f_l = g_1\cdots g_n$ に一意性を適用すると、適当に順番を入れ替えると、$c_if_i = g_i$、$c_i\in K$ と書くことが出来る。$I(g_i) = 1$ だから $c_i\in R$ を得、$g_i$ が原始多項式であることより、$c_i\in U(R)$ を得る。従って、分解は一意的である。
\qed

\newpage
\section{加群}
\begin{definition}
$R$ を環、$M$ を加群とし、写像、
$$R\times M \to M,\;(r,m)\mapsto rm$$
が与えられ、次の条件を満たすとき、$M$ を $R$-左加群(または単に $R$-加群)という。
$$r(x+y) = rx+ry, \;(r+s)x = rx+ry,\;(rs)x = r(sx), \;1x = x$$
$(x,y\in M,\;r,s\in R)$。
\end{definition}

\begin{itemize}
\item $R$-右加群も同様に定義される。$R$ が可換の時は、単に $R$-加群と呼ぶ。
\item $N\subset M$ が $R$-部分加群であるとは、$N$ が 部分加群で、かつ、$rx\in N$ がすべての、$r\in R$、$x\in N$ が成り立つことを言う。$RN\subset N$ なる条件を $N$ が $R$ の作用で閉じているとか安定であるとも言う。
\item $f:M\to M'$ が $R$-加群の準同型であるとは、
$$f(a+b) = f(a)+f(b),\;f(ra) = rf(a),\;(r\in R,\;a,b\in M)$$
を満たす時を言う。$f(ra) = rf(a)$ なる条件を、$f$ は、$R$ の作用と可換などとも言う。
\end{itemize}

\begin{eg}
\begin{enumerate}
\item 加群は、$\bZ$-加群である。
\item 環 $R$ は、$R$-加群であり、$I$ が $R$-加群 $R$ の部分加群であることと、$I$ が $R$ の左イデアルであることは同値である。
\item $K$ を体としたとき、$K$-加群は、$K$-ベクトル空間の事である。
\end{enumerate}
\end{eg}

\begin{definition}
\begin{enumerate}
\item $M$ を $R$-加群、$S\subset M$ とするとき、
$$<U> = \left\{\sum_{i}r_iu_i\mid r_i\in R, \;u_i\in U\right\}$$
を $U$ で生成される $R$-部分加群という。
\item $|U|<\infty$ なる$U$ について、$M = <U>$ となるとき、$M$ を $R$-有限生成という。このときは、その生成元を $u_1, u_2, \ldots, u_n$ とすると、$M = Ru_1 + Ru_2 + \cdots + Ru_n$。
\item $r_1u_1 + r_2u_2 + \cdots r_nu_n = 0$、$(r_i\in R)$ ならば、$r_1 = r_2 = \cdots= r_n = 0$ が成り立つとき、$u_1, u_2, \ldots, u_n$ は、$R$-自由であるという。$M$ を生成する部分集合 $U$ が $R$-自由(すなわち $U$ の任意の有限部分集合が $R$-自由)
であるとき、$M$ は、$U$ を基とする $R$-自由加群であるという。
\end{enumerate}
\end{definition}

\begin{itemize}
\item $V$ を体 $K$ 上の $K$-有限生成なベクトル空間とすると、$V$ は $K$-自由加群で、その基に属する元の個数は基の解き方によらず一定である。
\end{itemize}

\begin{definition}
$R$ を 可換環とする。$R$-加群でかつ環である $A$ が次の条件を満たすとき $A$ は $R$ 上の多元環($R$-代数)であるという。
$$a, b\in A,\;r\in R\mbox{ に対し }(ra)b = a(rb) = r(ab).$$
\end{definition}

\begin{eg}
\begin{enumerate}
\item $R$ 上の全行列環は、$R$ 多元環である。
\item $G = \{1 = u_1, u_2, \ldots, u_n\}$ を有限群とし、$G$ の元を基とする $R$-自由加群 $R[G] = Ru_1 \oplus \cdots \oplus Ru_n$ に次のように積を定義したものを群環という。
$$\left(\sum_{i=1}^n\alpha_iu_i\right)\left(\sum_{j=1}^n\beta_ju_j\right) = \sum_{i,j = 1}^n\alpha_i\beta_ju_iu_j.$$
\end{enumerate}
\end{eg}

$G$ を有限群、$A = \bC[G]$、$V$ を $A$-加群とする。$g\in G$ のとき $\phi(g) : V\to V, \;(v\mapsto gv)$ とすると、$\phi(g)\in \GL(V)$、また、$\phi: G\to \GL(V), \;(g\mapsto \phi(g))$ は、群としての準同型である。逆に、群の準同型 $\phi:G\to GL(V)$ が与えられると、$V$ は、$A$ 加群となる。

\medskip
$M$ を $R$-加群とする。$M$ が $0$ と $M$ 以外に部分加群を持たないとき、$M$ を既約と言う。既約でないとき、可約と言う。

\begin{thm}{\rm (Schur's Lemma)} \label{thm:schurslemma}
$M$、$N$ を共に既約 $R$-加群とする。
\begin{itemize}
\item[$(1)$] $f:M\to N$ を $R$-準同型で恒等的に $0$ でなければ、$f$ は同型である。
\item[$(2)$] End${}_R(M)$ で $M\to M$ なる準同型全体とすると、End${}_R(M)$ は斜体となる。
\end{itemize}
\end{thm}
\proof
$f$ を $R$-準同型とすると、$\ker f$、$\im f$ は、共に $R$-部分加群である。

$(1)$ $f\neq 0$ とすると、$\ker f \neq M$、$\im f\neq 0$ だから $\ker f = 0$、$\im f = N$ となる。これは、$f$ が同型写像であることを意味する。

$(2)$ $(1)$ より明か。
\qed

\newpage
\section{ヒルベルトの基定理}
\begin{definition}
\begin{enumerate}
\item $R$-(左)加群 $M$ に対して、その $R$-部分加群の任意の空でない集合に極大 [極小] なものが存在するとき、$M$ は、ネーター [アルチン] 加群であると言う。
\item 環 $R$ が $R$-(左)加群として、ネーター [アルチン] 環であるとき $R$ は (左)-ネーター [アルチン] 環であるという。
\item $M$ の $R$-部分加群の任意の列
$$M_1\subset M_2 \subset \cdots \subset M_i\subset \cdots\;(M_1\supset M_2 \supset \cdots \supset M_i\supset \cdots)$$
に対して、ある $n$ が存在して、$M_n = M_{n+1} = \cdots $ となるとき、$M$ は昇鎖律 [降鎖律] を満たすという。
\end{enumerate}
\end{definition}

\begin{prop} \label{prop:dcc:acc}
$R$-加群 $M$ がネーター{\rm [アルチン] }加群であるという事と、$M$ が昇鎖律 {\rm [降鎖律]} を満たすことは同値である。
\end{prop}
\proof
$R$-加群 $M$ がネーター加群だとする。$M_1\subset M_2\subset \cdots $ を部分加群の列とすると、$\{M_i\mid i\in\bN\}$ の中に極大なもの $M_n$ が存在するから、$M_n = M_{n+1} = \cdots $。逆に、空でない部分加群の族 $S$ に極大なものが無ければ、$M_1\subset M_2\subset \cdots \subset M_i$ を真に増大する鎖として取る。すると、$M_i$ は $S$ の中で極大ではないから、$M_i\subset M_{i+1},\;M_i\neq M_{i+1}$ となるものを含む、これを続けていくと、真に増大する部分加群の無限列がとれるので昇鎖律を満たさない。
アルチン加群であることと、降鎖律を満たすことが同値であることの証明も同様。
\qed

\begin{prop} \label{prop:noether}
$R$-加群 $M$ について、次は同値。
\begin{itemize}
\item[$(i)$] $M$ はネーター加群。
\item[$(ii)$] $M$ の任意の $R$-部分加群は $R$-有限生成。
\end{itemize}
\end{prop}
\proof
$(i)\Rightarrow (ii)$ $N$ を $M$ の部分 $R$-加群、$S$ を $N$ の $R$-部分加群で、$R$-有限生成なもの全体とする。仮定から、$S$ に極大元 $N_0$ が存在する。$N\neq N_0$ ならば、$x\in N-N_0$ とすると、$Rx + N_0$ は、有限生成でかつ $N_0$ を真に含むことになり $N_0$ の極大性に反するから $N = N_0$、すなわち、$N$ も有限生成である。

$(ii)\Rightarrow (i)$ $M_1\subset M_2 \subset \cdots $ を $M$ の部分加群の列とする。
$N = \bigcup_iM_i$ は、$R$-加群だから、仮定より有限生成で、$N = <u_1,u_2,\ldots, u_n>$ となる生成元があり、$N$ の仮定よりある $M_m$ にすべての $u_1, u_2, \ldots, u_n$ が入る。従って、
$$N \subset M_m\subset M_{m+1}\subset \cdots \subset N.$$
よって、$M$ は昇鎖律を満たす。命題~\ref{prop:dcc:acc} により $M$ はネーター加群である。
\qed

\begin{cor}
単項イデアル整域は、ネーター環である。
\end{cor}
\proof
任意のイデアルは、1 個の元で生成されるから、明らか。
\qed

\begin{thm} \label{thm:polyovernoether}
可換ネーター環 $R$ 上の多項式環 $R[x_1,x_2,\ldots, x_n]$ はネーター環である。
\end{thm}
\proof
$n=1$ の時を示せばよい。$I$ を $R[x]$ のイデアルとする。
$$I_i = \{r\in R\mid f(x) = a_ix^i + \cdots + a_1x + a_0 \in I \mbox{ で } a_i = r\mbox{ となるものがる。}\}$$
とおくと、これは $R$ のイデアルである。また、$f(x) = a_ix^i + \cdots + a_1x + a_0\in I$ ならば、$xf(x) = a_ix^{i+1} + \cdots + a_1x^2 + a_0x\in I$ だから、$I_0\subset I_1\subset I_2 \subset \cdots $ である。仮定より、$R$ はネーター環で、命題~\ref{prop:dcc:acc} より昇鎖律を満たすから$I_r = I_{r+1} = \cdots $ となる $r$ が存在する。命題~\ref{prop:noether} により、各 $I_0, I_1, \ldots, I_r$ は有限生成だから、$a_{i_1},\ldots, a_{i_{s_i}}$ を $I_i$ $(i = 0, 1, \ldots, r)$ の $R$ 上の生成元とする。$f_{i_j}$ を最高次の係数が、$a_{i_j}$ となる$I$ の $i$ 次多項式とする。このとき、これらが $I$ を生成すること、すなわち次が成立することを示す。
$$I = \sum_{i=0}^r\sum_{j = 1}^{s_i} R[x]f_{i_j}(x).$$
$f = a_mx^m + \cdots + a_1x + a_0\in I$ とし、$m = \deg f$ に関する帰納法で示す。

$m = 0$ ならば、$f = a_0 \in I_0 = \sum_{j=1}^{s_0}Ra_{0_j} = \sum_{j=1}^{s_0}Rf_{0_j}$ だから、この場合は良い。

$m>0$ とする。$r<m$ の時は、$e = m-r$、$r\geq m$ の時は、$e = 0$ と置くことにすると、
$$a_m \in I_m = I_{m-e} = \sum_{j = 1}^{s_{m-e}}Ra_{(m-e)_j}$$
だから、$a_m = \sum_{j = 1}^{s_{m-e}}c_ja_{(m-e)_j}$ とすると、
$$\deg (f(x) - x^{e}\sum_{j = 1}^{s_{m-e}}c_jf_{(m-e)_j}(x)) < \deg f(x)$$
だから、帰納法により、$f\in \sum_{i=0}^r\sum_{j = 1}^{s_i} R[x]f_{i_j}(x)$ であることが分かった。

$R[x]$ の任意のイデアルが、有限生成だから、命題~\ref{prop:noether} より、$R[x]$ はネーター環である。
\qed

\medskip
ネーター加群の剰余加群はネーター加群であることは簡単に分かるから、ネーター環の剰余環はネーター環である。可換環 $S$ が 可換環 $R$ を部分環として含み、さらに $s_1, \ldots, s_n\in S$ に対して、$R$ と、$\{s_1,\ldots, s_n\}$ を含む $S$ の部分環は、$S$ であるとする。(このとき、$\{s_1,\ldots, s_n\}$ は、$R$-上 $S$ を環として生成するという。例えば、$\bZ[x]$ において、$x$ は、$\bZ$-上 $\bZ[x]$ を環として生成するが、$\bZ$-加群としては、$\bZ + \bZ x$ すなわち 1次以下の多項式全体が生成されるものである。

\begin{cor}
可換ネーター環上有限生成な可換環はネーター環である。
\end{cor}
\proof
$R$ を可換ネーター環とする。$R$-上有限生成な可換環は、$R$ 上の多項式環の準同型像であるから、$R$ 上の多項式環の剰余環と同型である。従って、ネーター環である。
\qed

\end{document}
