\documentclass[12pt]{jarticle}
%A4 Size Setting
\topmargin = -0.5cm
\oddsidemargin = 0cm \evensidemargin = 0cm
\textheight = 23cm \textwidth = 16cm % default 16cm
%A4 size Setting End
\title{高校以上の数学は何のためか\\
2001 C-Week Open Lecture}
\author{鈴木 寛\\
Hiroshi Suzuki}

\renewcommand{\thepage}{%
	C-Week Open Lecture 2001 -- Script Page \arabic{page}}

\begin{document}
\maketitle

%\section*{Introduction}
\section{C-Week Open Lecture?}
今回、キリスト教週間のオープンレクチャー担当の学生から公開授業を頼まれた時は正直言って当惑してしまいました。C-Week のオープンレクチャーであること、そしてテーマが「光」だということ。どちらも、この「数学の方法」の授業とはかけはなれていると思えたからです。しかし、私がこの大学に8年程前に移ってきたのは、この大学がキリスト教主義大学で、かつ、教授会メンバーはクリスチャンという「キリスト教条項」とよばれる一般大学としては「異常」な方針を掲げているということが主たる理由でした。それゆえに、キリスト教週間の行事は、赴任した最初の年から極力すべて出席するようにしてきました。何か頼まれれば、もちろんお引受けする以外に私には選択肢はないように思われました。

そこで選んだのが本日のテーマです。「高校以上の数学は何のためか?」このコースの内容は、集合と論理、線形代数、微分積分で、このテーマは最初の予定に入っていなかったものですが、最近、私がおりに触れて考えているテーマでもありますし、「ICUにおける数学教育は何をすべきか」とも密接に関わる問題ですから、皆さんにも一緒に考えていただきたいと思います。実は、このコースは、今年から始まった新しいコースですが、私が是非、このようなコースを作るべきだと主張してつくっていただいたコースですから、このテーマは、このコースと切っても切れない関係にあることも確かなのです。

今朝、C-Week の早朝礼拝のあとで、宗務部長の森本あんり牧師から、「先生のオープンレクチャーのテーマは、C-Week のテーマと関連があるのですか。」と聞かれてしまいました。「関係があると良いのですが、ともかく努力します」とお返事しました。どうなりますか。正直心配です。

\section{Motivation}

ICUに来てからのことですが、ある時ある大学院生から「先生たちの殆んどはそれぞれの授業で何を教えるべきかを考えていないのではないか。とくに、一般教育科目でそれぞれの分野で何を教えるべきなのか真剣に考えている先生がたくさんいるとは思えない。」と言われてしまいました。ICU生らしいなかなか辛口のコメントですね。実はそのころ私は、はじめてそう言った問題を考えはじめたところだったのですが、「その通りかも知れない」としか言えませんでした。情けないことですね。私は、数学を教えはじめてから23年ほどたっています。それでいて何を教えるか考えていなかったというのですから、悲しいことです。まず、今まで私は何をしてきたかを振り返り、皆さんにお話することは、ざんげ録てきな意味合いもあり、C-Week にある意味でふさわしいのかと思っています。

しかし、まず皆さんにお話しておきたいのは、今までの私の歩みは今から考えると大分間違っていると思っていますが、それぞれの時に真剣に歩んできたということです。

今週の日曜日のICU教会では、山本香織先生のメッセージでしたが、新約聖書のピリピ人への手紙3章12節から14節の箇所から話されました。

\begin{itemize}
\item[12:] わたしは、既にそれを得たというわけではなく、既に完全な者とな
    っているわけでもありません。何とかして捕らえようと努めている
    のです。自分がキリスト・イエスに捕らえられているからです。
\item[13:] 兄弟たち、わたし自身は既に捕らえたとは思っていません。なすべ
    きことはただ一つ、後ろのものを忘れ、前のものに全身を向けつつ、
\item[14:] 神がキリスト・イエスによって上へ召して、お与えになる賞を得る
    ために、目標を目指してひたすら走ることです。
\end{itemize}

私のホームページからもリンクがある理学科のホームページにメッセージが載っていますから、興味のある方は読んでみて下さい。内容には触れませんが、「自分はまだとらえてはいないが、目標を目指してひたすら走っているのだ。自分をそのように駆り立てるのは、自分がキリストにとらえられているからだ」という箇所です。私は、現在数学の先生をしていますが、数学を教えるものとして生きるこの私を神様が善しとして下さることを願い、神様の善しとされることをもとめつつ、いきていきたいと願っています。

\section{大阪教育大学で}

この大学は教員養成大学ですから、今は、教員になることが簡単ではありませんが、当初はこの大学に入ってくる学生は、ほとんどが教員を目指していました。私が属していたのは、数学の教員を養成する学科でしたから、学生はほとんどが一生数学と関わっていく学生たちでした。

私のしていたことは、トップの何人か数学の研究ができそうな能力を持っているものを見い出すこと。あとの学生には、与えられた科目はしっかり教えるが、中心的には課外活動で、人生を共有する。ある意味で、自分が教えている数学は、ほとんどの学生にとって、必要ではないのではないかと思っていたのです。もちろん、学生も同じように思っていたと思います。そのような学生が、今、中学、高校の教員になっているわけです。

楽しい思いでがたくさんあります。旅行、スキー、そして何人かは研究者となり、でも正直いって、何を教えるべきか良く分かっていなかったと思っています。

もう一つは、数学教育という分野にたいする、抵抗感。数学教育を指向する学生に対する、偏見。
二つのグループの人たち。数学教育専門のひとと、数学の専門家で、数学教育に時々口を出す人たち。

\section{高校以上の数学は何のためか}

\subsection{問題点:}
\begin{itemize}
\item 大学入試に目標がおかれている。
	\begin{itemize}
	\item 実際の学問、そして社会でどのように数学が使われているかが教えられていない。
	\item 数学を学ぶ意義が教えられていない。
	\item 大学側も十分な配慮なしに、どちらかと言うと経営的な面から入試科目に入れない。
	\item 様々な制約からじっくり考える問題を、入試で問いにくい。
	\end{itemize}

\item 目的を考えて授業がされているか。

\item 理論を教えても良い。[4]
	\begin{itemize}
	\item 9月にICUに入りました。イギリスでは数学を教えるとき、ちゃんと論理
を説明してくれます。日本で分からなくなった数学も向こうで問題の解き方を
聞いて好きになりました。だから大学でも物事のすじみちを知りたいです。
	\end{itemize}

\end{itemize}

\subsection{数学を学ぶ意義:}
\begin{itemize}
\item 数学自体の研究のため。[33, 46, 55, 72]
\item (科学技術、社会科学のための) 道具として使うため。[3, 7, 8, 12, 13, 14, 15, 18, 19, 21, 23, 25, 27, 28, 29, 30, 41, 51, 56, 60, 61, 62, 69, 71]
\item 論理的思考の訓練。[1, 2, 3, 11, 16, 24, 29, 31, 32, 35, 36, 37, 38, 39, 40, 43, 45, 48, 49, 50, 52, 57, 58, 59, 63, 65, 68] 
\item たのしさ。趣味。[6, 12, 22, 29, 31, 44, 51, 53, 54, 58, 61, 71, 74]
\item 教養、思想として [10, 16, 25, 42, 47, 49, 64, 66, 67, 70, 75]
\item 意義はない。 [17, 26, 27, 35, 53, 56]
\item 受験のため。[51]
\item その他 [20, 59, 73]
\end{itemize}

\subsection{学生へのメッセージ:}

\paragraph{趣味:}

大学院生のころ数学は何のために勉強するのだろうかと考えたことがあります。専門家を目指そうとしていたわけですから、重要な問いです。いろいろとその重要性を考え、論理的に証明しようとしたあとで、しかし、自分が数学をする理由は、面白いから、知的好奇心のゆえだというところに行き着きました。このこと自体を誤魔化してはいけないと思っています。この意味では、高度の数学をするのは、「趣味でしょう」というのは当たっていますね。人間がその生の営みの中で感動するものに命をかけるのは、どんな分野でも素晴らしいことですが。

北野宏明:1984年理学科卒 ソニーコンピューターサイエンス研究所\\
「人が熱中するのは、役には立たないこと。」

\paragraph{科学における有用性:}

大辞林第二版 (三省堂)
\begin{itemize}
\item[(1)] 学問的知識。学。個別の専門分野から成る学問の総称。「分科の学」ないしは「百科の学術」に由来する。
\item[(2)] 自然や社会など世界の特定領域に関する法則的認識を目指す合理的知識の体系または探究の営み。実験や観察に基づく経験的実証性と論理的推論に基づく体系的整合性をその特徴とする。研究の対象と方法の違いに応じて自然科学・社会科学・人文科学などに分類される。狭義には自然科学を指す。
\end{itemize}

\paragraph{間接的有用性:}

思考の論理性を高める、論証力をつけるといった、大体誰でも考えることがあります。これらは、数学をすることによって最も力がつくことです。それ以外にも、計算を、それも少し難しい計算をすることにより、ルールを守りながら一つ一つのステップを注意深くこなしていく力もつきます。他にも、定義をして、その上で様々な議論を積み上げていく訓練は、同じ環境のもとで育ち、なーなーで何でもやっていける時は問題ないでしょうが、背景の全く違った人たちが理解し合い(例えば国際的に)協力して何かを成し遂げていこうという時には本当に重要な力だと思います。

\paragraph{真理追求:}

数学・科学は正直な真理追求の学問です。誤魔化したり、間違ったりする、そのときはそれが受け入れられることもありますが、いずれ、それは明るみに出され正されていく。結果オーライではありません。その意味でも、真理に対する、特別の感情をこの分野の人たちは持っています。ちょっと、持ち上げすぎかもしれませんが。科学者にクリスチャンが多いのも自然なことだと私は思っています。絶対的真理というものの存在を学ぶことは、特に日本では重要だと思います。

\paragraph{なぜ楽しくなくなっているのか:}

ではなぜ、数学を学校であまり楽しめないのでしょうか。私も、そのようなことの専門家ではないからわかりませんが、人間の思考の本質に関わるような数学では、理解するのに方法も、時間も、道筋も人によって非常に異なるということです。これは一括の多人数授業で対応することはできないということではないでしょうか。それと、チェックしやすい能力の教育に力と時間を注ぎ、いろいろな方法でじっくり考えることをおろそかにしたり、証明の力を養うことを減らし、共通テストでも証明問題が入れられないシステムにしてしまったりということも関係していると思います。そのような中で育ってきた人がまた教師となり、楽しさを伝えることがなかなかできなくなってきていることも問題でしょう。私は、ICUに来る前は、大阪教育大というところで教えていました。私自身の責任も強く感じています。
 
\paragraph{本当に文系では不要なのか:} 

例えばアメリカでは、大学で数学は必修です。日本では、文化系といわれる分野では数学は殆んどしません。今までは、高校までのレベルでの数学の必修がある程度ありましたからそれでもどうにかなっていたと思います(実際、アメリカの大学で文化系の人に教えている数学は日本の高校レベルが殆んどです)。しかし、日本でも最近、数学は特に必修の時間が大幅に減り、必修に関しては1/6になったと言われています。これでは、これからの特に文化系の学生さんは大変ですね。外国の大学院に入るような人はどうするのでしょうか。皆さんは是非積極的に数学の授業をとっていって下さい。たとえは、微分積分は、ICUでは高校で、微分積分などを全く勉強していない人用の授業も出ています。(初等微分積分学)この NSI の授業のような、体験的な数学だけでなく、実際に使っていくための数学も積極的に学んでほしいと思います。

\section{ICUにおける数学教育は何をすべきか}

\subsection{学生からの提言:}

\section{まとめ}

\end{document}

