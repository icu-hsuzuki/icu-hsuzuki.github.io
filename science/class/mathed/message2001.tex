\documentclass[twocolumn]{jarticle}
%A4 Size Setting
\topmargin = -0.5cm
\oddsidemargin = 0cm \evensidemargin = 0cm
\textheight = 23cm \textwidth = 16cm % default 16cm
%A4 size Setting End
\renewcommand{\thepage}{%
	C-Week Open Lecture 2001 -- Page \arabic{page}}
\begin{document}
\section*{高校以上の数学は何のためか\\
2001 C-Week Open Lecture}
\subsection*{学生からのメッセージ : AY2001}
\begin{enumerate}
\item 哲学と同じで、考える愉しみを味わうためのもの。
\item ロジカルに物事を考える手段ではないでしょうか(文系にとって)。
\item 自分の職業の基礎に数学が必要ならばその基礎のため。その他の人には論理的思考の涵養が主だと思う。もちろん、ひたすら学ぶこと自体が目的であるということもありだろう。
\item 高校までは小中の延長。ただもっと理論的なことから入ってもいいと思う。
\item 抽象的なことがらを理論立てて考える訓練。
\item わからない。生活上(日常)あまり高校以上の数学は使わないので。興味のため?
\item “高校以上の数学”の概念がよくわからないのでなんとも言えないが、社会が発展するためのtoolなのではないでしょうか。それは、論理的思考という面もあるし、技術発展のベースという側面があると思います。
\item 実際に科学技術などに応用するためのもの。
\item 論理的思考力を磨くもの。また、社会科学などその他の学問にも応用したい。
\item 日常で数学を使うことは計算くらいしかないので、教養だと思う。
\item わかりません。脳の訓練?
\item NSの人:研究のため。以外の人:楽しみ。
\item 専門分野で使う数学の学習。
\item 以前も書いたようにも思えますが、より実践的なものに用いられるように学ぶ。
\item 世の中のしくみなどをより分かりやすくとらえるための道具として使えれば良いと思います。
\item 高校までの数学がなんのためだったという前提なのかが謎ですが、自分は教養、論理的思考の訓練だと解釈しています。
\item 数学を学ぶ意味、目的を見つけたいです。一体何のためにこんなことをしなきゃいけないの!役に立たないし!と思いがちでした。
\item 現象を数式であらわすためのもの。
\item 世界平和のための道具。
\item 必要。自分自身、大学で数学嫌いが数学好きになったから。
\item 知識、社会で働きに出たとき、数学と向き合う機会は多いはずなので。また、いろんなことを分析していくのに役立つはず。
\item 知的な遊び・楽しみの増幅。
\item 理系を専攻する人には将来その知識を用いて、世の中へ利益をもたらす為。それ以外の人は社会生活を送る上での必要あるもの。
\item 思考の訓練のため。というのも1つの目的だと思う。
\item 実生活への応用。思想としての数学を学ぶため。
\item 正直よく分かりません。
\item 高校の数学も何のためだかわかりませんが(日常生活においてあまり必要性がないと思っています)、高校以上の数学は人類の科学技術発達のため、より豊かな生活のため、問題解決のため、意思決定のために必要だと思います。けれど、高校以上の数学が、一体どんなものかわからないので何とも言えません。
\item 世界をより深くみるため。
\item 理系や経済、経営学の分野においては基礎であり学問を助けるもので、実際には数学と関連しない文系の分野では、知的好奇心の対象であったり思考能力を育成するための良い場であったりすると思う。
\item 様々な考え方ができるようになるための1つの手段としての数学。
\item 高校以上は、“頭の体操”としての数学であって欲しい。楽しく自分の思考回路を磨くことができるものと思う。
\item どの学問に応用できるように論理的思考能力を養うため。そして、数学的思考を他の学問の媒体として使えるようにするため。
\item 数学という一つの分野への限りない追及のため。
\item 特に目的は無いと思う。高校以上は自主性に任されているのだから、本人に興味があれば良く、それ以上でも以下でもない。目的が欲しければ勝手に創ればよい。
\item 論理的思考を身につけるため。でも必ずしも高校以上の数学を学ばなければならないとは限らないと思います。
\item 頭の訓練。物事の原理をよりよく知るため。
\item 論理的な思考の訓練のため。
\item 脳の活性化。
\item 抽象的な思考(ある種の非日常)に出会うため。
\item 思考力を高める。現状の高校教育では数学や物理などでしか、それが期待できない。
\item 理論を実際にいかに利用するか。数学的考えを、応用する方法を学ぶためにあると思います。
\item 新しい考え方を身に付けるため。
\item 考えて解くものが多いから(公式があるにせよ、考えなければ出来ない)考える力をつけるというか数学的思考で何か他のことを考えるようにできるためか? 正直なところよくわかりません。
\item 時々、数学をやってハッとするような感動を覚えることがある。その感覚を知るため、とは言っても、僕の場合高校ではそんな体験はなく、予備校もしくは中学の頃読んだガモフ・シリーズなどで感動を経験した。
\item やることによって論理的思考力をつける為。スキルの獲得。
\item 知的探究。
\item 自分の可能性を広げる為。
\item 数学は、論理的に考える事が必要で、それは他の学問をする時も、必要だから、思考力を養うためだと思う。
\item 教養のため、かな?(実際に社会で役立つこともあるだろうし)論理的な思考力を養う為。副産物的なものとしては忍耐力がついたりするのかもしれない・・・。私の個人的な理由はやりたい学問に必要で、なかなかおもしろいから、です。
\item 別に高校以上も以下も同じ。論理的思考を非文系的考察方法で行うことに価値があると思う。
\item 高校までは受験のためにやってきた気もするので、これからは楽しめて、かつ実社会で応用できるような数学、考え方を身につけることを目標としたいです。
\item 論理的思考というか、いろんなものごとの考え方とかそういうのを身につけるため・・・でしょうかねえ。
\item 役に立つ役に立たないとかではなく、好きだったらやればいいと思う。日常生活で微積ができないとこまることはないから。日常生活は四則演算がこなせればなんとかなるし。
\item 数学で飯を食っていくわけではないので問題が解けた時の喜びを味わうため。
\item 研究のため。
\item 日常の生活には必要ないと思う。しかし、大学とかで行うたぐいの学問をする上で、いろいろな問題解決の糸口となるために、必要不可欠なものだと思う。
\item 論理的に物事をとらえることができるようになるため。
\item 頭を使うため。趣味。
\item 高校では受け身だったけど、数学を使った様々な方法に触れて形にはまらずに色々な場で数学的思考を利用して問題を解決することができるようになるため。解くことよりもどうやって答が出せるか考える。おもしろさを追及(問題に取り組むための!)。数学の点とかはテストでよくなかったけど興味はあるので期待しています。
\item 各専門分野において必要だから。数学の中に美を求める。
\item 応用的に他の分野で使う。純粋に楽しむため。
\item やりたいことをやるために必要なもの。
\item 論理的思考の養成と論理が通用しないこともあるということをあえて知ること。
\item この社会をもっとよりよく知るため。
%\item 趣味。よくわからない。
\item 論理的思考を身に付ける手段。他にもあると思いますが・・・
\item 探究心を満たすため。
\item 社会人として知っておくべき教養のため(?)(もちろん、将来的にそれを用いて仕事をする人もたくさんいますが・・・。またそれを知らない人も・・・。
\item 何のためかではなく数学そのものの論理性が好きだからやっているので特に目的はない。強いて言えば数学が目的。
\item 受験や、化学・物理の計算のためではなく、自分の周囲に常に存在する数の世界を計算するため。
\item 生活の基盤かな?
\item 高校以上の数学はもしかしてSSに必要かもしれないし、そうでなくても幅広い知識を身につけるのに役立つと思う。
\item 数学を用いる専門を学ぶため。受験数学に疑問を持つため。
\item 高校以上に続けるのはただの数字や記号の操作だけでなく、そこから何かを見出すため。
\item 興味。
\item 自分の知識欲。
\end{enumerate}

{\footnotesize Messages are taken from : \\
\hfill {\it http://science.icu.ac.jp/\textasciitilde hsuzuki/class/ns1b/ns1b/}}
\end{document}

