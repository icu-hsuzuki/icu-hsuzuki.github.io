\documentstyle{article}
%A4 Size Setting
\topmargin = -0.5cm
\oddsidemargin = 0cm \evensidemargin = 0cm
\textheight = 24cm \textwidth = 16cm
%A4 size Setting End

\pagestyle{empty}
\begin{document}
\begin{center}
{\Huge\bf AY1996-7 Calculus II 

\smallskip
RESULT}
\end{center}

{\Large
以下は、今回の、Calculus II Final の結果です。
$$\begin{array}{|c|l|}
\hline
90\sim 100 & \\
\hline
80\sim 89 & 0, 0\\
\hline
70\sim 79 & 7, 7 \\
\hline
60\sim 69 & 0, 2.5, 5, {5}^*, {5}^*, 6, 8\\
\hline
50\sim 59 & 0, 0, 3, 3, 3, 5, {5}^*, 8\\
\hline
40\sim 49 & 0, 3, 5, 5, {7.5}^*\\
\hline
30\sim 39 & 0, 0, 0, 0.5, 5, 5, 5.5, 7.5, 8, 8\\
\hline
20\sim 29 & 0, 0, 0, 0, 2, 3, 3, {3}^*, 5, 5, 5, 5, {5}^*\\
\hline
10\sim 19 & 0, {3}^*, 5, 5, 5\\
\hline
0\sim 9 & 0, 0, {0}^*, 5 \\
\hline
\multicolumn{2}{|l|}{\mbox{Final 欠席 3 名、* は、2年生以上}}\\
\hline
\end{array}$$

\medskip
今回のテストは、少し難しかったようですね。しかし、あまりにも、出来なかったという印象を拭えません。特に、30点に、満たないと言うような結果は、Calculus II で学ぶべき事(8回の講義で、約8項目として)について、一つぐらいしか分かっていない事になります。責任の約半分は、教える側、半分は学ぶ側と、常々思っていますが、まず、教える側の力不足を感じています。

さて、成績を決める場合、特に、E のラインを決めるとき、二通りの方法があると私は思っています。
\begin{itemize}
\item[A: ] このコースを取って、取る前と比べて、明らかに、学んだ点が認められれば単位を出す。
\item[B: ] このコースを取った事を示すことの出来る最低限の事柄について、不十分ながらも、学んだことが、客観的に認められる場合に、単位を出す。
\end{itemize}
私の理解では、自然科学を学ぶ場合は、何を学ぶときも、Calculus I, II, Linear Algebra I, II までは、必修とするのは、日本でも、世界でも全く一般的であり、社会科学を学ぶときは、殆どの場合、Elementary Calculus, Linear Algebra I 程度を必修とするのも、日本以外では一般的だと思います。のみならず、ここで必修と言われている内容まで、世界中で、共通理解があると思います。もちろん、そのうえで、どのレベルを最低線とするかは、大学により、教員によってまちまちです。ICUでは、特に、1から2年の間、外国語教育などのためコースワークがとても忙しく、例えば、Calculus I, II も、1年かけて、学ぶところも多いかと思いますが、ICUでは、2学期間で学んでいます。短期間で、この範囲のものを学ぶことは、大変なことです。その大変さをある程度可能にしてきているのは、教える側の努力も少しあるかと思いますが、私は、ICUの学生の本来の能力の高さと、勤勉さと、努力によっていると思っています。そして、それ抜きにしては、ICUの特に、自然科学系のカリキュラムは成り立っていけないのではとさえ思います。しかし、学生諸君のこの様な素晴らしい特性の陰に、誘惑もあると思います。それは、GPA に代表される、平均主義、相対価値判断です。「あるものは、多少悪くても、他が良ければ取り返すことが出来る。」「クラスの中で、まあまあやっていけば、それほど、ひどいことにはならない。」と言った発想です。しかし、自然科学の場合、それでは、世界の中で活躍することは出来ないと、私は思っています。特に、ICU理学科基礎科目のレベルでは、そこで学ぶことは、世界標準、そのコースを学んだということは、ICUの中で判定せざるを得ませんが、証明する舞台と、基準は、世界です。これは、それぞれの分野での専門に関しても同様です。GPA 制度は、私は大事な制度だと思っています。しかし、それに甘えてしまっては、何もならない制度です。ICUという世界での世渡りがうまいだけでは世界の舞台では活躍できません。そして、世界の舞台で(専門は何であれ)活躍しうる学生諸君の勉学を援助するのが、ICU教員の役目だと思っています。特に、一年生は、これから、2年目へと向かい、ICUになれてくる頃だと思いますが、世界の中での活躍のための自分への挑戦であることを自覚していただきたいと思います。Sophomore は、二つのギリシャ語の言葉  sophos (賢い)+ mohros(愚か)から出来たそうですが、ある意味では「愚か」で、しかし本質的に「賢い」2年生になって欲しいと思います。

成績とは全く関係しませんが、もう一度、本や、ノートを見ても構いませんから、Final に挑戦してみて下さい。Final をもう一度解いて持って来て下さった方には、私の作った解答も差し上げます。(学期間休暇中も殆ど毎日、研究室に来ていますが、連絡をしてから来ることを奨めます。)

成績には、簡単に上の (B) の判定条件を持ち込むことは、今回は出来ないかと思いますが、私も熟慮して決めたいと思います。

\medskip
\begin{flushright}
鈴木寛@数学教室\\
電子メール: hsuzuki@icu.ac.jp、電話: 0422-33-3292
\end{flushright}


}
\end{document}