%\documentclass[twocolumn]{article}
\documentclass{article}
%A4 Size Setting
\topmargin = 0cm
\oddsidemargin = 0cm \evensidemargin = 0cm
\textheight = 23cm \textwidth = 16cm % default 23cm X 16cm
%A4 size Setting End
\begin{document}

\section*{James Earl Carter}

\subsection*{Chapter One: The Early Years}

James Earl Carter, the 39th U.S. President, is best known by his nickname, ``Jimmy.'' He first became involved in politics --- the work of the government --- as a young boy growing up near Plains, Georgia. His father, Earl Carter, belonged to the Democratic Party, which is one of the country's two major political parties. A political party is a group of people who share similar ideas about how to run a government.

Earl Carter was a peanut farmer, just like his son would be one day. He sold tools and other items to local farmers. From his father, Jimmy learned to work hard. He also learned to take an interest in government. Earl took young Jimmy to political barbecues throughout the Georgia countryside. At these events, guests spent the whole day listening to politicians give speeches. They exchanged political talk and enjoyed barbecued pork and chicken. Little did anyone know that Jimmy would one day become the nation's most important leader.

Jimmy's mother, Lillian, was an intelligent, open-minded person. She always listened to Jimmy's ideas. She taught him to care for the poor. She taught him about civil rights, which are the basic rights guaranteed to all American citizens. One of the most difficult issues when Jimmy was growing up in the South was segregation. At that time, people were beginning to push for an end to this system, which used laws to keep black people and white people apart. Lillian taught Jimmy that African Americans deserved to have the same rights as white Americans.

Together, Jimmy's parents encouraged him to learn about the U.S. government and to stand up for his beliefs. Many times in his life, people challenged Jimmy's ideas about what was right and wrong. But he stuck to his beliefs.

In 1941, Jimmy graduated from Plains High School. He attended Georgia Southwestern College and later the Georgia Institute of Technology. In 1943, he was accepted to the U.S. Naval Academy in Annapolis, Maryland. Attending the academy was an honor, and Jimmy worked hard to do his best. He graduated in 1946 with a degree in engineering. The same year, he married Rosalynn Smith, a young woman he know from Plains. He soon began his career in the navy, working as an engineer on submarines.

In 1953, Jimmy learned that his father was dying of cancer. By that time, Jimmy was the chief engineer of a submarine crew. He was faced with an important decision. He, Rosalynn, and their three sons were happy living in New York, where Jimmy was working on a new submarine. But after his father died, Jimmy wanted to help his family, to support his mother, and to run the family farm and business. He longed to return to the South.

Rosalynn was unhappy with his decision, but Jimmy's mind was made up. As always, when he made a decision, nothing stopped him. Leaving the navy would change the course of his life.

Right away, Jimmy set about improving the family's supply business. He began planting peanuts from seeds. It took several years of hard work before the businesses began to make profit. ``The entire first year I was home, our income was less than \$ 300,'' he later recalled. ``But we stuck it out.'' Jimmy never had any doubt that he had made the right decision by going back home. Rosalynn helped run the businesses as well. At first, she worked only one day a week, but soon she was working full-time. In fact, she knew so much about the business that Jimmy often asked her for advice.

Meanwhile, their three sons --- Jack, Chip, and Jeff --- were growing up. Jimmy became a leader at his Baptist church. He also became involved with the local Lions Club. He began working with the Sumter County Board of Education and served as its head for seven years. Jimmy's main goal at the school board was to end segregation in schools. He wanted black children to attend the same schools as his sons. He wanted every child to have the same educational opportunities. This wasn't a popular view at that time in the South, but Jimmy's beliefs never wavered.

In 1962, at age 38, Jimmy decided to run for a political office. He wanted to run for the Georgia State Senate and help make laws. Running for office wasn't something Jimmy had ever imagined himself doing. But in his late 30s, he quietly considered the possibility. He wanted to become more involved in the state's education issues and serve on the senate's education issues and serve on the senate's education committee. On an October morning in 1962, a few weeks before the election, Jimmy drove to the Sumter county seat and entered the race. Rosalynn supported his decision. In fact, she helped with the campaign whenever she had time, addressing letters, telephoning voters, and keeping records. Jimmy won the election. He and his family moved to Atlanta, the state capital. It was the beginning of a long career in politics.

\paragraph{Photo 1: }
Jimmy Carter was born on October 1, 1924. He grew up in the small farm town of Plains, Georgia. From a young age, his goal was to get a good education and to attend college.

\paragraph{Photo 2: }
As a young man, Jimmy Carter enjoyed sports. He played basketball at Plains High School. In this photograph of the team, he is second from left in the top row. When he attended the United States Naval Academy, he ran cross-country and played football.

\paragraph{A Career in Politics}
In 1966, Jimmy Carter ran for governor of Georgia. He lost but then spent the next four years preparing for the next election. He traveled tirelessly around Georgia, trying to understand the state's problems. In 1970, Carter was elected the 76th governor of Georgia.

As governor, Carter made it clear that he would work to help all Georgians, especially those who most needed his assistance. ``I say to you quite frankly that the time for racial discrimination is over,'' he said at his inauguration. ``No poor, rural, weak, or black person should ever have to bear the additional burden of being deprived of the opportunity of an education, a job, or simple justice.'' His words drew attention not just from citizens in Georgia, but from people all over the nation. Together, he and Rosalynn worked to help senior citizens, children, people with mental retardation, and prisoners.

By 1972, Carter was already considering the possibility of running for president. In September of 1973, his mother, Lillian, asked him what he planned to do after leaving the governorship. Carter replied, ``I'm going to run for president.'' She asked, ``President of what?'' ``Momma,'' answered Carter, ``I'm going to run for president of the United States, and I'm going to win.''

\subsection*{Chapter Two: At Home and Abroad}

In 1976, the Democratic Party chose Carter as its candidate for president. As he accepted the honor, Carter said that 1976 would be a year different from any other.  ``It will be a year of inspiration and hope,'' said Carter. ``It will be a year in which we will give the government of this country back to the people of this country.''

Carter also said that human rights would be an important theme of his presidency. Human rights are basic freedoms to which all people, everywhere, are entitled. Carter said it was a privilege to live in a democracy like the United States. Americans can help make decisions about how their government is run. In some nations, governments abuse their citizens, especially those who speak out against their leaders. Carter believed the United States should not support any government that mistreated its people. His idea were honest and admirable. Many Americans had faith in his plan.

It was a very close election, but Carter won. Even so, when he moved into the White House the following year, many Americans knew little about him. The United States was still recovering from the Watergate scandal, which had damaged Americans' trust in their leaders. The Watergate was a building where the national offices of Democratic Party were located. Members of the Republican Party wanted information about the Democrats' plan for the next election. A group of five men broke into the offices to steal important information. By early 1973, it was clear that President Nixon and his aides had been involved in the break-in.

The Watergate scandal brought an end to President Nixon's career in politics. He was forced to leave the presidency, and Gerald Ford became president. Ford served for two years before Carter defeated him in the election of 1976.

Like other Americans, members of Congress were still angry about the Watergate scandal. They needed time to get to know and trust the new president. Its members considered Carter an outsider because he had not worked as a politician in Washington before he became president. They often made it difficult for him to reach his goal.

Even so, Carter had big plans. He wanted to protect human rights not just at home, but all around the world. He wanted to improve educational opportunities for all Americans. He wanted to work for peace in the Middle East, the region where Asia, Africa, and Europe meet. Diverse peoples and cultures live in the region. There often have been conflicts between them. Carter believed the Unite States could help negotiate a peaceful solution to some of the problems.

Another important goal of President Carter was to decrease the number of nuclear arms produced, both in the United States and in the Soviet Union. Nuclear arms are powerful weapons that can attack vast areas of land. They can kill an injure huge numbers of people. Carter hoped to negotiate an agreement with Soviet leaders. He wanted to reduce the risk that such weapons would ever be used.

All of these issues affected the entire world. Global problems took up most of President Carter's time. But the United States needed his attention, too. Inflation had been a major problem since Nixon was president. The prices of food, clothing, and other items were higher than ever before. Unemployment was high, and many people in the United States could not find jobs. Pollution and deforestation were destroying the nation's wilderness. Protecting the nation's environment was a necessity. Energy sources were limited, so the nation was in the midst of an energy crisis. This meant people had to cut back on their use of gasoline and electricity.

Americans doubted that Jimmy Carter could fix everything that had gone wrong in recent years. It would have been a difficult job for anybody. But Carter wanted to tackle the problems his country faced. His way of doing this was to study every part of a problem before making a decision. Some people criticized him for this habit. They said it took him too long to act.

Carter did accomplish several things during his presidency, however.  He created the Department of Energy, which helped the nation use its energy sources more carefully. He also worked to protect the environment. The Alaska National Interest Lands Conservation Act protected 150 million acres --- an area about the size of California --- was used to create new national parks and monuments. According to President Carter, signing this bill was one of the most satisfying acts of his presidency.

He also founded the Department of Education. Carter had always believed that educating young people should be among the nation's most important goals. He wanted more Americans to go to college. Education was a way to ensure that the country had a bright future. But many members of Congress didn't think the president should worry about education.

It took three years to make the Department of Education a reality. Once it was founded, the department made a big difference. For one thing, it lent money to college students to help them pay for school. It also worked to improve public schools all over the nation.



\paragraph{The Energy Crisis (page 21)}
In the 1970s, the energy crisis was one of the most serious problems facing the nations leaders. Americans were using more oil than any other countries in the world. The world's oil supplies were running low. People began to worry that there would be no oil left one day. In addition, the price of oil kept going up. What followed was a shortage of affordable energy. People had to wait in long lines to fill their gas tanks. Some lines were a mile and a half long! Gas stations started to put limits on how much gas each person could buy. Americans had to find a way to use less energy.

Carter was determined to solve the energy crisis. He introduced an energy bill to Congress, which urged Americans to conserve energy. The bill also set aside money to develop new energy sources, such as solar power (energy from the sun), hydroelectric power (energy from running water), and wind power. Congress finally passed a new version of the bill 18 months after Carter introduced it. 

President Carter and his staff created the Department of Energy in 1977. This wasn't easy task, but it may have been one of his greatest achievements. Over the years, it has accomplished many things. It has worked to develop new energy sources and encouraged Americans to carpool. Thanks to the Department of Energy, houses are now built to save energy. Manufacturers make appliances that use less energy. Carter's efforts paid off. From 1977 to 1980, U.S. oil consumption dropped by 20 percent.


\subsection*{Chapter Three: Treaties for Peace}

President Carter devoted much of his time to problems outside the United States. Dealing with international issues was among the most challenging work of his career. Some Americans thought he spent too much time helping people in other parts of the world. They wanted him to solve problems at home first. But Carter wanted to help people wherever he could.

Soon after he became president, Carter began working on the Panama Canal Treaty. The Panama Canal is a waterway that opened in 1914. It was built to allow passage between the Atlantic and Pacific oceans. Before the canal was completed, ships had to travel all the way around the tip of South America. Today they can use the canal to travel through Central America.

The United States built and paid for the canal. Since 1914, it had taken charge of it. But the Panamamians wanted to run the canal themselves, without help from the United States. Carter agreed. He believed it was the only way to ensure friendly relations with Central America. Carter reached an agreement with Panamanian leader Omar Torrijos Herrera. A treaty would give the canal to Panama by the end of 1999. Many Americans were against the treaty. Still, President Carter was able to convince Congress to approve the treaty. The canal would belong to Panama by the end of the 20th century.

Perhaps one of Carter's most important goals was to help promote peace between the nations of Egypt and Israel. It would be a difficult tasks, for these peoples had been at war since Israel became a country in 1948. Arab people of Palestine without a homeland. This angered not only the Palestinians, but other Arabs in the regions as well. They believed the Israelis had wrongly taken land that had belonged to Arabs for thousands of years. There was constant warfare between Israel and neighboring Arab countries.

Only 30 years, four wars were fought between the Arabs and Israelis. Israel took control of more land that had once belonged to Arab nations. In particular, it had seized the Sinai Peninsula, a stretch of land between Israel and Egypt. The peninsula had belonged to the Egyptians, and they wanted it back. Israeli leaders refused to give it up and sent soldiers there to protect it.

Carter wanted to end the terrible warfare in the region. He hoped to forge better relations not only between Israel and Egypt, but among all nations in the Middle East. He decided the United States should help negotiate a solution. In 1978, he organized the Camp David Summit.

Camp David is a small camp located in the mountains of Maryland. It is a peaceful place where the president can go for privacy and rest. Carter thought it was the perfect place to begin peace negotiations. He invited Egyptian President Anwar Sadat and Israeli Prime Minister Menachem Begin to Camp David. Carter was eager to greet the two leaders. He hoped to encourage a spirit of cooperation. He asked them to avoid arguments and angry words during their time at Camp David.

President Carter thought it would take three days to reach an agreement. But it would not come that easily. Begin refused to give back Egypt's land on the Sinai Peninsula. Sadat returned to his cabin and packed his bags. Carter tried to convince Sadat to stay, but he refused. Finally, just as Sadat was about to leave for the airport, Begin agreed to return the land to Egypt.

After 13 days of long, difficult meetings, the leaders finally reached an agreement. They created the Camp David Accords, which were treaty between Egypt and Israel. The other recommended ways to establish peace throughout the Middle East. The accords led to the signing of a peace treaty. Begin, Carter, and Sadat signed the treaty on March 26, 1979. It was a first step toward ending the ancient difficulties in this troubled region.

In 1979, another part of the world demanded attention from President Carter. The Soviet Union invaded Afghanistan in late December. This was a serious problem. Americans worried that the Soviets were trying to expand their system of government, called communism, to other parts of the world. Communism is a system in which a country's government holds great power. Most Americans believed it was dangerous and did not want it to spread to new places.

Carter and the Soviet leader, Leonid Brezhnev, had signed the Strategic Arms Limitations Talks (SALT II) Treaty earlier that year. Now Congress had to approve it. This plan would reduce the number of nuclear arms that each nation produced. As an engineer, Carter had studied nuclear science. He knew these weapons could cause terrible destruction. He hoped SALT II would reduce this risk. But the Soviet invasion of Afghanistan threatened the SALT II Treaty.

After the invasion, Carter knew he had to show the Soviets that the United States did not approve of their actions. He asked Congress to postpone its decision about the SALT II Treaty. Then he called for a grain embargo. The Soviets depended on the United States as a food source. The embargo meant that the United States would no longer supply them with grain.

Carter also withdrew American athletes from the 1980 Olympic Games. This was among the most difficult things he had to do during his presidency. The Olympics were taking place in Moscow, the capital of the Soviet Union. The Soviets would earn a great deal of money by hosting the games. Carter did not want to support them in any way until they took their troops out of Afghanistan. Other nations agreed with Carter's decision. In fact, 62 other nations did not send teams to the Olympics that year.

Unfortunately, Carter's efforts had little positive effect. For one thing, the Soviets did not leave Afghanistan until 1988. Even worse, his action had caused problems for Americans. The embargo hurt American farmers, who suffered because they sold fewer crops. American athletes who had trained all their lives for the Olympics lost their chance to compete in the 1980 games.

Probably the most troublesome issue Carter faced was the Iranian Hostage Crisis. In January of 1979, Iran's leader, Shah Mohammad Reza Pahalvi, had been exiled from his homeland. A religious leader, the Ayatolla Khomeini (pronounced eye-yah-TOL-uh koh-MAY-nee), took over the Iranian government. The Ayatollah did not like the United States. American relations with Iran suffered.

The shah traveled to the Bahamas and then to Mexico, looking for a home. Then he announced he had cancer and wanted to come to America for treatment. But Carter worried that this would ruin America's already troubled relationship with Iran. He also knew the shah had not always cared about the human rights of Iran's citizens. Finally, Carter allowed the shah to come to the United States. In protest, a mob of Iranian students seized the American Embassy in the Iranian capital of Teheran. On November 4, they captured 66 American hostages. They demanded that the United States return the shah to Iran.

The hostage crisis lasted for more than a year. It took up most of President Carter's time. He worried constantly about the American hostages.

\paragraph{A Peace Treaty (page 31)}
The Camp David Accords led to an important treaty between Egypt and Israel that ended conflict between the two nations. Prime Minister Begin, President Carter, and President Sadat signed the treaty at the White House on March 26, 1979. 

When countries sign a treaty, both sides promise to abide by certain agreements. Israel and Egypt agreed to a number of things. First, Israel promised to return Egypt's land on the Sinai Peninsula. It also promised to remove its military forces there. The Egyptians agreed to let the Israelis use its important waterway, the Suez Canal. They also agreed to sell oil from Sinai to the Israelis. Finally, the Israelis promised to begin peace negotiations with other Arabs in the Middle East. They agreed to negotiations that would give the Palastinians more rights in the region, including the right to set up their own government.

A few days after they signed the treaty, Begin and Sadat met again in Cairo. They agreed to set up a special telephone hotline so they could easily contact each other. Israel also planned to return part of Sinai to Egypt ahead of schedule. It seemed that the treaty had truly accomplished something. Perhaps peace in the Middle East was possible.

Not everyone approved of the treaty. Other Arab leaders still believed that Israel had no right to lands in the Middle East. Even some Egyptians believed Sadat had given Israel more than it deserved. His enemies wanted to stop him from making further agreements with the Israelis. Sadat was assassinated on October 6, 1981. Unfortunately, the Middle East still struggles for peace. 


\subsection*{Chapter Four: After the Presidency}

The hostage crisis made the last year of Carter's presidency a difficult one. In the spring of 1980, he approved a plan to rescue the hostages. On April 24, a secret helicopter mission, called ``Operation Eagle Claw,'' was sent to Iran. 

First, a group of eight helicopters was to fly to Desert One, a location about 70 miles south of Teheran. From there, the operation would fly to the hills east of Teheran. Unfortunately, disaster struck at Desert One. Only five helicopters made it there. The other three were forced to land because of a dust storm. A minimum of six helicopters was necessary for a successful mission. The plan had to be canceled. But as one of the helicopters was leaving Desert One, its propellers kicked up a huge cloud of dust. The pilot couldn't see through the dust cloud, and the helicopter smashed into a small airplane. Eight Americans died in the accident.

The failed rescue mission only made things worse between the United States and Iran. Following the rescue attempt, the terrorists hid the hostages throughout the country. The shah died in July, but this made no difference. Americans began to think Carter had seriously mishandled the situation. The next election was coming up in November, and the hostage crisis hurt his chance of winning. On Election Day, Ronald Reagan easily defeated Carter and became the nation's 40th president.

The same month, Iran sent a message to the United States. It had a list of conditions for the hostages' release. An agreement was finally announced on January 19, 1981, the day before Reagan's inauguration.

It wasn't until after Carter left office that the hostages were actually released. On January 20, the day of President Reagan's inauguration, word reached Americans that the hostages were on their way home. Carter worked on negotiations until the moment President Reagan took office. Unfortunately, many Americans gave the new president credit for solving the crisis.

On January 21, President Reagan sent Carter to meet the hostages at a U.S. military base in Germany. It was an emotional homecoming for everyone involved --- especially the former president who had struggled so hard to win their freedom.

Carter's strong commitment to human rights did not end after his presidency. It would guide him in all future activities. The Caters soon began to devote most of their time to helping people all over the world.

In 1982, Carter became a professor at Emory University in Atlanta, Georgia. He enjoyed his work at the university, but he wanted to do more to help people. He realized that people in countries all over the world lived in difficult, life-threatening situations caused by war, disease, famine, and poverty. He believed he could find ways to help.

In 1986, Jimmy and Rosalynn Carter founded the Carter Center. This organization has an important mission --- to end human suffering. The goals of the Carter Center are to ``prevent and resolve conflicts, enhance freedom and democracy, and improve health.'' Its vision is that everybody in the world should be able to live in peace.

Several different teams work at the center. Some teams focus on helping nations build democracies. The center has helped run elections in more than 20 countries, including Venezuela, Mexico, and Peru. By doing so, it has given people the freedom to vote for democratic leaders and take part in their governments. Representatives from the Carter Center also have negotiated peaceful solutions to problems in countries such as Sudan, Bosnia, and Korea.

Some workers from the Carter Center fight disease. Others teach farmers how to grow more food for their families. Rosalynn Carter heads the center's program to aid Americans with mental illness.

One of the Carter Center's biggest programs is The Atlanta Project. It works to help people in Atranta's most troubled neighborhoods. Health clinics and preschools have been built in these neighborhoods. The center also has after-school programs to give young people a safe place to go.

In addition to his work at the center, Carter has written many books. He writes about topics that are important to him. He wrote a book on the history of the Middle East called {\it Blood of Abraham} (1985). He also wrote a book about negotiation called {\it Negotiation: The Alternative to Hostility} (1984). And he wrote a book about the environment called {\it An Outdoor Journal} (1988).

Finally, Jimmy and Rosalynn Carter devote their energy to Habitat for Humanity. This organization builds houses for low-income families in countries all around the world. Without Habitat for Humanity, these families could not afford to buy their own homes.

As Jimmy left the White House in January 1981, he promised himself that he would continue to be a world leader. He spoke the following words that would chart the course of the rest of his life:

``The battle for human rights --- at home and abroad --- is far from over. We should never be surprised or discouraged because the impact of our efforts has had, and will always have, varied results. Rather, we should take pride that the ideals that gave birth to our nation still inspire the hopes of people around the world.''


\subsection*{Interesting Facts}

\begin{itemize}
\item Jimmy Carter was the first U.S. president born in a hospital.

\item Jimmy Carter grew up in a home without electricity or running water.

\item As a young boy, Jimmy Carter won an award at school for reading more books than any other student.

\item Carter was the first president to have graduated from the U.S. Naval Academy.

\item Rosalynn and Jimmy Carter had known each other for years before their first date. Rosalynn was a close friend of Jimmy's younger sister, Ruth. One day, Rosalynn saw a picture of him in Ruth's room. ``I couldn't keep my eyes off the photograph,'' she later wrote. ``I thought he was the most handsome young man I had ever seen.''

\item In 1966, Jimmy Carter's mother, Lillian, volunteered for the Peace Corps. She was 68 years old. President John F. Kennedy created the Peace Corps in the early 1960s. It is an organization made up of volunteers who spend two years working overseas to help people in other nations. Lillian went to the village of Vikhroli, India.

\item Carter was the first president to be elected from the Deep South since Zachary Taylor in 1848.

\item Carter's youngest child, Amy, was nine years old when they moved to the White House. Amy went to public school in Washington. She often brought friends home to play with her at the White House.

\item When President Carter traveled, he preferred to stay in the homes of ordinary Americans instead of in expensive hotels.

\item Jimmy Carter greatly respected his wife, Rosalynn. He always asked for her advice. When he became president, Jimmy and Rosalynn had a private meeting once a week to discuss political issues.

\item A trip from New York to San Francisco by sea is nearly 8,000 miles shorter using the Panama Canal instead of going all the way around South America.

\item A toll must be paid to pass through the Panama Canal. The highest toll ever paid was \$ 165,000 by a cruise ship. The lowest toll was 36c. It was paid by Richard Halliburton, a man who swam the canal in 1928.

\item Before Carter met with Prime Minister Begin and President Sadat, he wanted to learn much about both men as he could. He studied information about their childhoods, their families, and their likes and dislikes.

\item Israeli historians have learned that before the Camp David Accords, the only other agreement for peace between the Egyptians and Israelis took place some 3,000 years ago. It was a peace agreement made between King David's son, King Solomon, and the Egyptian Pharaoh.

\item In 1978, Menachem Begin and Anwar Sadat won an important honor, the Nobel Peace Prize, for their work at the Camp David Summit. Unfortunately, the prize did not recognize the efforts of Jimmy Carter.

\item Four years after Americans boycotted the Olympic Games, the Soviet Union refused to attend the 1984 Olympics in Los Angeles. Few other nations dropped out of the Olympics that year, however. Only 81 nations had competed in Moscow, but 140 went to Los Angeles. In fact, China returned to the Olympics in 1984. Its athletes had not attended the games for 32 years.

\item The Iranian terrorists held the hostages for 444 days. They were not released until the day Ronald Reagan became president. But it was Jimmy Carter's efforts that eventually freed the captives.

\item For recreation, Jimmy Carter enjoys fly-fishing, wood-working, jogging, cycling, tennis, and skiing.

\item Carter teaches Sunday school and is a deacon in the Maranatha Baptist Church in Plains, Georgia.

\item The Carter Center has helped more than one million farmers worldwide increase the yield from their crops.

\item In 1994, Jimmy and Rosalynn Carter traveled to Yugoslavia to help with peace talks between two groups in Bosnia, the Muslims and the Serbs. Their mission resulted in a four month cease-fire.

\item Thanks to President Carter's efforts, the Panama Canal was turned over to Panama on December 31, 1999.

\item The Carters volunteer their time to Habitat for Humanity. Many homeless people have become homeowners thanks to this organization. It offers loans to families in need so they can afford to buy a house. Each family helps to build their own working with volunteers who offer their time to the cause.

\item According to Jimmy Carter, having a place to call home is one of the most basic of all human rights. It was only natural that he and Mrs. Carter became involved with Habitat for Humanity. Each year since 1984, Jimmy and Rosalynn have led team of volunteers that builds houses for the Jimmy Carter Work Project. Just before he left office, Carter's White House staff presented him a carpenter's tool set as a gift, knowing he enjoyed woodworking. The tools have come in very handy!  
\end{itemize}

\subsection*{Time Line}

\begin{itemize}
\item[1924] James Earl ``Jimmy'' Carter is born on October 1 in Plains, Georgia.

\item[1941] Carter graduates from Plains High School. He enrolls at Georgia Southwestern College in Americus, Georgia, that fall.

\item[1942] Carter Transfers to the Georgia Institute of Technology.

\item[1943] Carter is accepted to the U.S. Naval Academy in Annapolis, Maryland.

\item[1946] Carter graduates from the Naval Academy. On July 7, he marries Rosalynn Smith.

\item[1953] After his father dies, Carter leaves the U.S. Navy after seven years of service to return to Plains. There he runs the family business and peanut farm.

\item[1962] Carter is elected to the Georgia State Senate.

\item[1966] Carter runs for governor of Georgia but loses the election.

\item[1970] Carter is elected governor of Georgia. He holds the post until 1975. By 1972, he is already thinking about running for president.

\item[1976] The Democratic Party nominates Carter as its presidential candidate. He runs against President Gerald Ford. Carter wins in a close election.

\item[1977] Carter is inaugurated the 39th president of the United States on January 20. He establishes the Department of Energy on August 4. Carter and Panama leader General Omar Torrijos Herrera sign the Panama Canal treaty on September 7.

\item[1978] The Camp David Summit begins on September 4. Carter hopes to help negotiate a peace treaty between Israel's Prime Minister Begin and Egypt's President Sadat. After 13 days of meetings, the Camp David Accords are created.

\item[1979] The shah of Iran is exiled from his country in January. By February 1, the Ayatolla Khomeini has taken over the Iranian government. On March 26, Begin, Carter, and Sadat sign the Egyptian-Israeli peace treaty. Soviet leader Leonid Breshnev and President Carter sign the SALT II Treaty on June 18. The Department of Education is formed on October 17. Terrorists take over the American Embassy in Teheran on November 4. They take American hostage. The Soviet Union invades Afghanistan on December 27.

\item[1980] Carter boycotts the 1980 Olympic Games, which are held in Moscow, the capital of the Soviet Union. He does this to protest the Soviet invasion of Afghanistan. Carter and his advisors decided to attempt a hostage rescue mission. On April 24, the rescue mission, ``Operation Eagle Claw,'' begins. The attempt fails, and eight Americans are killed. Republicans nominate Ronald Reagan as their presidential candidate in July. Reagan defeats Carter on Election Day. Carter signs the Alaska Lands Bills in December.

\item[1981] The final terms for the release of the American hostages in Iran are negotiated during Carter's final days as president. Ronald Reagan is inaugurated president on July 20 at noon. Twenty minutes later, the hostages are released in Teheran. The next day, President Reagan sends Carter to a U.S. military base in Germany to welcome the hostages home.

\item[1982] Carter becomes a professor at Emory University in Atlanta, Georgia.

\item[1986] Jimmy and Rosalynn Carter found the Carter Center.

\item[1991] Carter announces The Atlanta Project.

\item[1999] Rosalynn and Jimmy Carter receive the Presidential Medal of Freedom in August. On December 14, Carter travels to Central America to prepare for the transfer of control of the Panama Canal. On December 31, a ceremony is held to turn the canal over to Panama.

\end{itemize}

\subsection*{Presidential Facts}
\begin{description}
\item[Qualification] To run for president, a candidate must
	\begin{itemize}
	\item be at least 35 years old
	\item be a citizen who was born in the United States
	\item have lived in the United States for 14 years
	\end{itemize}
\item[Term of Office] A president's term of office is four years. No president can stay in office for more than two terms.
\item[Election Date] The presidential election takes place every four years on the first Tuesday of November.
\item[Inauguration Date] Presidents are inaugurated on January 20.
\end{description}

\subsection*{For Further Information}
\paragraph{Internet Sites}
\begin{itemize}
\item Visit the Carter Presidential Library:\par
http://carterlibrary.galileo.peachnet.edu/
\item Visit the Carter Center:\par
http://www.cartercenter.org/
\item Learn more about the Panama Canal:\par
http://www.panamacanal.com/
\item Learn more about Habitat for Humanity:\par
http://www.habitat.org/
\item Learn more about all the presidents and visit the White House:\par
http://www.whitehouse.gov/WH/glimpse/\par
presidents/html/presidents.html\par
http://www.thepresidency.org/presinfo.htm\par
http://www.americanpresidents.org/
\end{itemize}

\begin{thebibliography}{9} 

\bibitem{LH} Lori Hobkirk, ``James Earl Carter, Our Thirty-Ninth President'', Published in the United States of America by The Child's World, Inc. (www.childsworld.com) ISBN 1-56766-873-9.

\bigskip
\bibitem{A} Altman, Linda Jacobs. {\it The Creation of Israel}. San Diego, CA: Lucent Books, 1998.

\bibitem{C} Carter, J. {\it Talking Peace: A Vision for the Next Generation}. New York: Dutton Childrens Books, 1993.

\bibitem{LG} George, Linda, and Charles George. {\it Jimmy Carter: Builder of Peace}. Chicago: Childrens Press, 2000.

\bibitem{L} Long, Cathryn J. {\it The Middle East in Search of Peace}. Brookfield, CT: Millbrook Press, 1993.

\bibitem{S} Sandak, R. Cass. {\it The Carters {\em (First Families)}}. New York: Crestwood House, 1993.

\bibitem{St} Stein, R. Conrad. {\it The Iran Hostage Crisis}. Chicago: Childrens Press, 1989. 
\end{thebibliography}

\end{document}