%Linear Algebra I 97 Autumn Final Exam 
%% -- Scheduled November 20, 1997 1P
% original : November 14, 15, 16, 1997

\documentclass[11pt]{jarticle}

%A4 Size Setting
\topmargin = -0.5cm
\oddsidemargin = 0cm \evensidemargin = 0cm
\textheight = 24cm \textwidth = 15cm % default 16cm
%A4 size Setting End

\newcommand{\note}{\vspace{2ex}\noindent{\gt 注\quad}}
\newcommand{\proof}{{\gt 証明\quad}}
\newcommand{\qed}{\hfill\hbox{\rule{6pt}{6pt}}}
\newcommand{\ba}{\mbox{\boldmath $a$}}
\newcommand{\bb}{\mbox{\boldmath $b$}}
\newcommand{\bd}{\mbox{\boldmath $d$}}
\newcommand{\be}{\mbox{\boldmath $e$}}
\newcommand{\bu}{\mbox{\boldmath $u$}}
\newcommand{\bv}{\mbox{\boldmath $v$}}
\newcommand{\bw}{\mbox{\boldmath $w$}}
\newcommand{\bx}{\mbox{\boldmath $x$}}
\newcommand{\by}{\mbox{\boldmath $y$}}
\newcommand{\bo}{\mbox{\boldmath $0$}}
\newcommand{\sgn}{\mbox{{\rm sgn}}}
\newcommand{\rank}{\mbox{{\rm rank}}}
\newcommand{\tr}{\mbox{{\rm trace}}}
\newcommand{\batsu}{{\large $\times$}}
\newcommand{\maru}{$\bigcirc$}

\pagestyle{empty}

\begin{document}
\begin{center}
{\gt\Large Linear Algebra I Final Examination 1997}
\end{center}

\noindent
ID 番号、氏名を、各解答用紙に、また、問題番号も忘れずに書いて下さい。

\begin{enumerate}
\item 次のうち正しいものには \maru、誤っているものには \batsu を解答用紙に記入せよ。

     $(a)-(c)$ は、次の連立一次方程式について。
     $$\left\{\begin{array}{ccc}
a_{11}x_1 + a_{12}x_2 + \cdots + a_{1n}x_n & = & b_1\\
a_{21}x_1 + a_{22}x_2 + \cdots + a_{2n}x_n & = & b_2\\
\multicolumn{3}{c}{\cdots\cdots\cdots} \\
a_{m1}x_1 + a_{m2}x_2 + \cdots + a_{mn}x_n & = & b_m
\end{array}\right.$$
     \begin{enumerate}
     \item $m\leq n$ すなわち方程式の数の方が、未知数(変数)の数よりも多くなければいつでも上の連立一次方程式は解を持つ。
     \item $m<n$ すなわち方程式の数の方が、未知数(変数)の数よりも少ないとする。このとき解が丁度一組に決まることはない。
     \item $m<n$ かつ、$b_i = a_{i1}$、$i = 1,2,\ldots, m$ であるとする。このとき、解は無限個存在する。
     
     \medskip\noindent
     $(d)-(f)$ において、$A$ を以下のような行列とする。
     $$A = \left[\begin{array}{cccc}
a_{11} & a_{12} & \cdots & a_{1n}\\
a_{21} & a_{22} & \cdots & a_{2n}\\
\vdots & \vdots &            & \vdots\\
a_{m1} & a_{m2} & \cdots & a_{mn}
\end{array}
\right]$$
     \item $AA^t$ も $A^tA$ もどちらも正方行列である。
     \item $m<n$ ならば、$AA^t$ は、常に可逆行列である。
     \item $m = n$、すなわち $A$ は、正方行列とする。このとき、
     $\det (AA^t) = \det (A)^2$ が常に成立する。
     \end{enumerate}

\item $A$、$\bx$、$\bb$ を次の様にする。%10\times 4 = 40
$$A = \left[\begin{array}{cccc}
1 & 0 & -1 & 2\\
2 & 1 & 1 & 3\\
0 & -1 & 2 & 4
\end{array}\right],\;\bx = \left[\begin{array}{c} x_1 \\ x_2 \\ x_3 \\ x_4 \end{array}\right],\;\bb = \left[\begin{array}{c} b_1 \\ b_2 \\ b_3 \end{array}\right]$$
\begin{enumerate}
\item $A$ の階数 $\rank A$ を求めよ。
\item 行列の方程式 $A\bx = \bb$ は、$b_1, b_2, b_3$ が何であっても解を持つかどうかを判定し、理由を述べよ。もし、ある条件のもとで解を持つときはその条件も求めよ。
\item $b_1 = b_2 = b_3 = 0$ であるときの $A\bx = \bb$ の解をすべて求めよ。
\item $b_1 = b_2 = b_3 = 1$ であるときの $A\bx = \bb$ の解をすべて求めよ。
\end{enumerate}

\item 次の計算をせよ。%10\times 4 = 40
	\begin{enumerate}
	\item 順列 $\rho = (3, 7, 2, 1, 4, 6, 5)$ の追い越し数 $\ell(\rho)$ と、符号 $\sgn(\rho)$。
	\item ${\displaystyle
	\left|\begin{array}{ccc} 1 & 3 & 5 \\ 3 & 7 & 11 \\ 2 & 4 & 6
	\end{array}\right|}$ % 0
	\item ${\displaystyle
	\left|\begin{array}{cccc} a & b & c & d \\ -b & a & d & -c \\ -c & -d & a & b\\
	-d & c & -b & a \end{array}\right|}$ % (a^2 + b^2 + c^2 + d^2)^2
	\item ${\displaystyle
	\left|\begin{array}{ccccc} 0 & 0 & a & 0 & 0 \\ 0 & 0 & 0 & 0 & b \\ c & 0& 0 & 0 & 0\\
	0 & 0 & 0 & d & 0 \\ 0 & e & 0 & 0 & 0 \end{array}\right|}$ %abcde	
	\end{enumerate}
	
\item $P(i,j;c) = I + c\cdot E_{i,j}$ ($i, j = 1,2,3$、$c$ は実数)を基本行列とする。ただし、$I$ は、$3$ 次単位行列、$E_{i,j}$ は、$(i,j)$ 成分が $1$ でそれ以外は、$0$ である $3$ 次の行列単位とする。このとき、次の行列を、$P(i,j;c)$ のいくつかの積で表せ。
$$\left[\begin{array}{ccc} 1 & x & y \\ 0 & 1 & z \\ 0 & 0 & 1 \end{array}\right]$$
%20\times 1 = 20

\item $f(t) = c_0 + c_1t + c_2t^2 + c_3t^3$ を次の条件を満たす多項式とする。%40
$$f(-1) = 2, \;f(1) = 5,\;f(3) = -1,\;f(5) = -3$$
\begin{enumerate}
\item この多項式を求める方程式を行列方程式 $A\bx = \bb$ で表すとき $A$、$\bx$、$\bb$ を書け。
\item $|A|$ を求めよ。(公式を用いるときは公式自体も記せ)
\item $|A|\cdot c_2$ を $4\times 4$ の行列式を用いて表せ。行列式の値は計算しなくて良い。
\item $A$ の逆行列を $A$ の余因子行列を用いて表せ。成分に現れる行列式の値は求めなくて良い。
\end{enumerate}

\item $X = [x_{i,j}]$ を $3$ 次正方行列とするとき、$\tr(X)$ は、$X$ の対角成分の和、すなわち、
$$\tr(X) = \sum_{i = 1}^3 x_{i,i} = x_{1,1} + x_{2,2} + x_{3,3}$$
とする。$A$、$B$ が共に、$3$ 次正方行列であるとき、$\ll A,B \gg = \tr (AB^t)$ とする。
	\begin{enumerate}
	\item $\ll A,B \gg  = \ll B,A \gg$ であることを示せ。
	\item $\ll A,A \gg = 0$ であれば、$A$ は、零行列(成分がすべて零である行列)である。(ただし、行列の成分はすべて実数であるとする。)
	\end{enumerate}
\end{enumerate}

\begin{flushright}
鈴木寛@数学教室
\end{flushright}
\end{document}
