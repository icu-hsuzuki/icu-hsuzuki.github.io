

\item
``ボルケーノ’’という最近のパニック映画(ロスの地下に突然、溶岩の噴出口ができる&その溶岩流を海に導いて待ちが救われるという筋書き)で、またも人命救助のために溶岩にとびこむことを選んだ端役の人が、マタイ伝の一節(?)を死ぬ覚悟をする直前に唱えていました。ハリウッドの人は、こういう表現が大好きみたいです。``切れた電話に対して受話器をみつめる’’など「定型化」した映像表現はどにかならないものでしょうか。もっとも、自分が死ぬ前に一言いうとしたら、何を言うのかなあ?

\item
あります。物語として。内容はおもしろいと思います。でも、キリスト教はよくわかりません。すべての宗教は、メシアとミラクルというものがあると思うのですが、現実主義の私としては、実感がわきません。

\item
読んだことがある。以前にもメッセージの欄で書いたことがあるが小説聖書を読み終わった。ときおり聖書を比べながら読んだ。以外と日本人社会の中にも浸透している考え方、もののとらえ方もあると感じた。

\item
ないです。

私は三浦綾子さんの本がとても好きです。気分が落ち込んだ時にはよく読みます。それは私がキリストを信じているわけではなく、自分以外の人に対して今までよりも優しくすることができる様になるからです。

\item
カトリックの幼稚園に通っていたので、子供用の聖書をずっとよんでいました。中学も、イギリスでReligiousEducationという授業があったので読みました。高校のキリスト教概論でも読みました。ずっと身近にあるけれど宗教を自分の中によりいれることはあまりしようと思ったことがありません。

\item

キリスト教の印象—厳しい 父性的 個人的(←寂しいととる人もいるのでは)

1学期間どうもありがとうございました。

数学が苦手な上に今学期は理由あってたくさん授業をとっていてこの授業を消化できないことも多く、すいません。でも、楽しかったです。

\smallskip
以下3年生

\item
私は高校からプロテスタント系のキリスト教の学校に通っていたので、聖書はその時の聖書のクラスや、毎日ある礼拝、たまに会食とか言って先生と一緒にごはんをたべる時の前にもよんだりした。聖書は読むのが大変だけど、助けられる様な言葉がたくさん書いてある。私は聖書も好きだが賛美歌のほうが好き。特にクリスマス用の賛美歌が良い。

\item
神道にもかかわらず何度か教会へ行きました。すごく人を清らかな気もちにさせるけど(聖書)それを扱った人間自体の歴史がよくなかったと思う(ie 戦争、黒人、ユダヤ人圧迫→キリストはユダヤ人では?)

\item
あります。キリスト教の印象は、少し押しつけがましいという気がしました。歴史的にみても責めの強い宗教だと思いました。人間を超える何か大きな力のようなものがあるとは思いますが、それが神なのかはわかりませんし、人間を保護してくれるようないい物であるのかもわかりません。誰も真実は分かってないのに、それをいい物として、神として決めつける必要があるのかが少し疑問です。

なんだか、全く分からない高校の時の数学の模試を思い出しました。それぐらい分かりません。

\item
ありません。キリスト教の印象…幼稚園がミッション系だったので悪い印象はありません。ただそれについてホトンド何も知らないので、私が自信をもってキリスト教について意見を述べることはできません。キリスト教、仏教、イスラム教、ヒンドゥー教、宗教は文化から生まれ文化を生むもののように最近は思えます。

\item
インクリすらまだ取っていません。

``キリスト教’’は嫌いです。まだしも仏教のほうが人間の真の姿について触れているようなのでマシです。とはいえ、キリスト教を信じている人を否定する気はありません。信じる、信じないはその人の自由ですから。

\item
ないです。

キリストさんは好きだけどキリスト教は好きではない。

\item
高校の時の授業では読んでいました。(それもやはりキリスト教のクラスです。)キリスト教の学校に行っていて、自分自身で内得してしまったのは、特に宗教信が強い人だからといって善人であるわけではない、ということです。こんなことを内得してしまうのは、大変に駄目だとはわかっているのですが。たまたま出会った人がそうだったからだとは思っています。しかし私自身は決してキリスト教の考えや教えは嫌いではありません。

\smallskip
以下2年生

\item
アメリカ留学中にホストファミリーが毎週教会に連れていってくれてそこで改宗してしまいました。

\item
苦手意識が強かったが故に何となくさけてしまった数学ですが、もっとやっておけばよかったなーと今思います。

\item
ICUに入る前、他大学でキリスト教概論を勉強したことがあります。その時は、自分の命は自分1人だけのものではない、という観点から自殺や中絶などの問題について考えました。

\item

あります。

★ICU祭で会った、自転車にのった5人の子どもたちかわいかったです。たのしそうでした。

★ところで、先生のhomepageみたいのですが秋休みになってしまいそうです、みるのが。Finalで痛い目に会ってでも、ホームページみますね。きっと。

Christianの人たち:思っていることが+やっていることが、絶対的に正しいと信じているように見える。でも、実際、何事も信じこんでやらないと達成感など味わえないのでね?+他の人の意見とか考えはなっとくしててもしかkりしたものをもっていて、全面的には受け入れない。

これは今まで出会った、限られた人々についてなので、、、

\item
4、についての質問。

まったく同じ所に玉が当たった(というか通過!?)としたらどうなるのでしょう。それも全部の玉が。スゴイかも。

\item
私は、イエスキリストを自分の救い主と信じてから5年がたちますが、大学に入るまでは、あまり深く、読んだことがなく、新約聖書をパラパラ、旧約は有名なところくらいしか知りませんでした。しかし、大学に入って、神様と自分という関係が大切になるにつれて、もっともっと知りたいと思うようになり、新約を通読したり、旧約も全部読みたいと思うようになりました。Biblestudyに参加したり、日々、自分で、聖書を読んで神様からみことばを頂き、お話する時は私にとて欠かせない大切な時間です。

聖書を読めばよむほど、疑問がたくさんでてきますが、それを考えることで、さらに神様の奥深さ、すばらしさを知ることができ、本当に感謝です。

\item
・はい。

・似たところが多い(イスラムからみて)。

イサは神ではないと思うがキリスト教ではそうらしい?イスラムは、イサ→キリスト教は、イエス

\item
今日は全くひらめきませんでした…。

\item
私自身はクリスチャンではないのですが、クリスチャンの知人が多く、聖書も読んだことはあります。小学校の頃、日曜学校で。そのころはあまり深く考えてはいなかったのですが、今現在、ICUに通い、YMCAでボランティアリーダーをやっていると、キリスト教について、考えることが多くなりました。信仰を持っている人の強さというものも感じますし、ただ、クリスチャンの人は寛容な人というか、人をあたたかくむかえいれてくれる雰囲気をもった人が多いと感じています。

\item
1 今インクリとっていてはじめて聖書を読みました。
また\underline{キリスト教概論Ⅰ(森本アンリ先生)}☆でも同性愛の否定の場面での聖書の用いられ方という点でも扱いました。

そこで思ったのは、とてもきれいな書物ですけれども、悪用もされうるなどの点できけんな存在でもあるということです。また、あんり先生はキリスト教が西方において広まってしまったのが最大のまちがいだたという興味深い見解もおっしゃっていました。

2 私は仏教の方が考えが慣染みます。両者を比較するとおもしろい。☆はとーっても面白いですよ。

\item
小学校以来のミッションスクールで過ごしてきたが、よくわからない。

小学校の頃はおはなし(説話)的で、一番宗教色が強かった。

中学校の頃はイエスについて考える

高校 では今現在生きている人の聖書との関わり方をきくという授業だった。

インクリは未経験。

今は、小学校の頃に育てたイメージと耳に残っている一節とを時折引き出すくらいの関わり方。

\item
先生もICUの教授だからChrisitianなんですよね。

私個人としてはこの制度は反対です。「それがICUの個性なんだ」という人もいますが、教授が皆Christianであることにどれだけのメリットがあるのか疑問です。

キリスト教の印象というと、開発・援助と宗教をセット販売してるという感じ。土着の宗教をふみにじっているという印象をうけます。

\item
僕は信者ではないので授業以外で聖書は読まない。キリスト教を「知識として知っている・知っていない」は自分のものとして信仰しているのと比べればどちらも同じ様なものだと思うから。関係ない話しだが、僕がキリスト教やその他の宗教について考えたり人としゃべったりするときに頭の中には常にある種の辛らつさや意地悪さがあり、そういうものが過度にならない様に気をつけている。

\item
子供の頃なぜか家にあった(我が家はキリスト教徒ではありません)絵本の様な``わかりやすい(たのしい?)聖書’’を読んだことがあります。

キリスト教は基本的にあまり好きではありません。いろいろとおしつけがましい気がするし、けっこう狂信的な部分もあるような気がするのであと一神教という所も…

私は特に宗教を信じているわけではないですが個人的には、山には山の神様がいて海には海の神様がいるといった考え方(アニミズム?)が好きです。

\item
高校がChristian系だったので高校時代毎日礼拝があってよんでいました。

今も時々教会に行ったりしてます。

でも、信じることはできません。いつもうたがいの目をもちながら読んでしまいます。

\item
高校がミッション系の学校だったので「宗教」の時間に聖書を読みました。キリスト教といわれると「隣人を愛せ」としか思い付かないほど無知ですが、考え方は基本的にはすきです。ただ世界中の人がそういう考え方になったら世の中はつまらなすぎるほど平和になりそうだなと思います。(ありえないでしょうが)

インクリで今のキリスト教は他宗教とのつながりを大切にするときいたので割といい印象をもってはいます。私自身が無信仰でいまのところさしさわりがないので宗教は私の生活とは関わりが薄い気がします。

\item
中・高で毎朝礼拝をし、週に1度は聖書の授業がありました。今もHumなのでキリスト教史やキリスト教倫理の授業をとっていますが…まだまだわからないことが多いです。

でも私は三浦綾子さんの本が大好きで、そこに描かれている人々の心にはいつも心を動かされて、もっと知りたいとは常に思っています。

\item
中・高で宗教の時間に。キリスト教の印象は言うこととやることが違うなということ。ま、世界中何でも誰でもそうですけど。ex.宗教戦争とか

\item
『旧約聖書を知っていますか?』あとうだなんとかさん おもしろかった

あとホテルにとまったときひき出しの中に入ってる聖書もよんだ。

\item
友達がクリスチャンなどで、その人の書いたもの(文章とか)などから、また幼稚園がキリスト教主義だったので。

\item
私はクリスチャンなので人にキリスト教の印象を聞くと、``位’’、``ノンクリスチャン’’に対して疎外する態度が見られると言われました。そうかも知れないと思った。

でもある時、15才の高校生から``宗教をもつ人は暗く、弱い人間だとずっと思っていたが宗教をもつことでその人が前向きになり進むことができるのならその人は強いんじゃないか’’と思うといわれ感動しました。

\item
アメリカでプライベートのハイスクールに通っていたのでそこはCatholic系の学校だったのでReligionのクラスがあり、キリスト教について教わったため、聖書を読みました。

\item
聖書を初めて読んだのは小学校の頃でした。Okutama Joy Bible Campというものに4回ぐらい夏休みになると参加していて、そのときにmissionの時間があったので、聖書を読んだりキリストのスキットを見たり演じたりしました。その後、聖書をキャンプのsupervisorからもらい、大事にそれを読んだ、その中の絵をながめて考えごとをしたりしました。でも中学時代がいそがしくなってからは、ほこりをかぶったままで悪い気がします。

\item
私はクリスチャンなので、聖書を読んだことはあります。(通読はまだですけど…涙)今、東神大のMr.Hastingsのインクリをとっています。先生の個人的な(?)解訳が入ってきたり、カトリックの解訳が入ってきたりして時々、混乱してしまいます。

「キリスト教は宗教ではない」とよく聞きますが本当にそうだと思います。「他の宗教でも救いはある。自分にはそれがイエス・キリストだった」と言う人もいるけど、それはちがうと思います。

\smallskip
以下1年生

\item
新約聖書が家に1冊あるのですが、読んだことがありません。宗教に対して全般的に存在しているし、それを信ずる人が自由にいてよいと思っている一方で、自分自身は「最後に信じられるのは自分」と感じているので少し聖書にかんして抵抗があるから読まないのかもしれません。

\item
よんだことない
差別的表現が多いだの何だの言われ、その抜け道をさがして解しゃくの仕方をくふうしているがどんな宗教だっそうさ。だめですか。だめですよね。
冬にインクリとります(とれたら)から、よく考えてみます。

\item
中学・高校がカトリック系だったので、読んだことはあります。キリスト教は「良心」の宗教だと思います。でも僕自身の宗教観は、「心の中にこそ神はいる」ということです。こうある方が、世界で起きている宗教同士の醜い争いは起きないと思うからです。

\item
インクリでも聖書は読みませんでした。
自分がキリスト教系大学にいるという実感はほとんどありません。

\item
あまりありません。(インクリは来学期にとります。)でも興味はあります。「小説『聖書』」も読んでみたいです。(人気あるらしいので。)でもなんでクリスチャンの人って、キリスト教の言葉やキリスト教をあまり疑いもせずに信じ込んでしまうんでしょう(悪く言えば盲信的)。聖書ってたまに非論理的なところがある気がします。

\item
今学期は永田先生のインクリをとっていますが、あまり聖書は読んでいあにです。高校がICUHSだったのですが、1年生の時の夏休みの宿題が「創世紀」を読んで要約のレポートを提出するものでした。いろいろな話しがあって面白かったのですが、登場人物の寿命が時代を経るにつれて短くなっていくのが気になりました。高校のときからつい最近まで、現代まで続いている宗教対立にまつわる戦争ばかりが目についていて、それが私の宗教嫌いの1番の理由でした。ですが、インクリ受けているうちに、それだけの理由で簡単に嫌いだと決めつけることに疑問を持ち初めています。

\item
キリスト教のsituationを使っている漫画や本に出会った時期があった。今回は筆入れを忘れたので、消しゴムが使えませんでした。汚くてすみません。

\item
聖書は、まだ読んでいないんですが、何にしろ読みたいとは思っています。聖書は、何を読むうえでも、というか一般常識として必要なので。

\item
宗教はその地域でmonorityであるほど、信じられるかもしれない、と思わせるほどのまじめな活動をしているように思う。それはやはり、人に伝えたい、とういう熱意の表れなのだろうと思います。政治のために使われ、Topの人間が権力をふりかざすようでは、キリスト教もたいしたことはない、と思ってしまう。

\item
インクリもまだとっていないので聖書を読んだことは一度もありません。ただ、良心ともに、京都の同志社大学出身で特に父は同志社に中学から通っているので、聖書の中のことばをよく話してくれました。そのときのイメージとしては、表現方法が違うものの仏教と似ているところがあるな、と感じました。私は高校が仏教系の学校だったので特にそう思いました。ただ、世界史やユダヤの歴史に私は興味があり、ユダヤ関係の書籍を読んでいると、ユダヤ人差別の根本原因はキリスト教にあるように思われます。キリスト教の地位の確立されていない3〜4世紀ごろから、キリスト教同様にすぐれた宗教であったユダヤ教は、ローマ帝国にキリスト教が認められ国教とされたこともあり、徐々に排せきされていったかんじがします。それが大戦中のホロコーストに直接的に関連してるとはいいませんが、責任の一端があるような気がします。

\item
高校の倫理の時間に少しだけ読んだ。キリスト教はすごく理想的なことを言っている気がしてあまり好きになれない。クリスチャンの言う"神様がいつも守ってくださる"ろいうのも、どうも他力本願的にきこえる。もちろん自分一人で生きていけるわけではないので、謙虚な姿勢はとても大切だし、理想を抱いて努力するのも大事なことだとは思うけれど。

\item
授業中ねていてスミマエセンね。この学校に来るために新聞奨学生をしながらきてるから。朝の3時におきて、くるから、寝ないでいようとがんバってんだけどね。本当にスミマセン。

\item
高校で聖書の時間があったのでその授業中ではあります。授業外では、KKKについて興味があるので、ある映画でKKKが白人至上主義を聖書が正当化せいいると主張をsいていたのでその部分を読んだりしました。あと今思い出したのですが、アメリカに住んでいた頃夏休み中に参加したsummer campで聖書を読み、それについてdiscussionをし、自分の反省をし、心から洗い流すという時間がありました。

\item
眠れないときに たまに聖書(旧約、創世紀)を読むと、すぐに眠くなり、よく眠れます。

\item
僕はカトリックの洗礼を受けていますが、キリスト教はあまり信じていません。僕が思うにカトリックのミサはほとんど儀式化されていしまっていて、やることに意義があるような状況になってしまってるように感じます。あと 理想を求めてて、現実にそぐわないようにも感じます。そろそろ法王が変わりそうなので、それに合わせてカトリックとキリスト教全体も変化していって欲しいです。

\item
幼稚園がミッション系のところだったのでそのころに少し読みました。

\item
今、歌劇でやっているJesusの最後の1週間を扱う`The Last 7 Days'の練習で少し資料として。キリスト教って、そもそもの始まりは一部の熱狂的な支持者によって生まれたもので、ずいぶん後になって宗教として成立してきたものなんだなあと痛感すると共に、じゃあJesus Christって本当は何だったんだろう、という思いが強くなってきています。やっぱりただの人だったのかな。まあ、そうなんだろうけど。だとしたら彼の功績は民衆を扇動しただけだったりして。

\item
私は高校の授業でキリスト教を学んだ。イエスが十字架の上で最後に口にした言葉が、「父よ、なぜ私をお見捨てになたtのですか」だったというのが、とてもイエスが人間らしいと感じられるところだと思う。

\item
小学校のときカナダのシスターに教会で英語を習っていました。子供用の聖書でした。そこの教会ではいつも大きな声でしゃべると何かいけない気がしたものです。

\item
大学礼拝にいく時に牧師さんが読んでくれるところは読んでいます。キリスト教は好きです。映画、(洋)を見ていると海外とキリスト教のつながりの深さが知れておもしろい ex.天使にラブソングを、フェイス・オフ。サウンドオブミュージックでも反もある。時間がないのでかけない。

\item
インクリの授業もまだとっていないので、ほとんど読んだことがありません。幼稚園がキリスト教系だったのでお祈りはしていましたがもう忘れてしまいました。ICUに入ったからにはキリスト教にもう少しふれてみたいと思います。先生はどういうきかっけでキリスト教に関心をもたれたのですか?

\item
僕の両親は両方ともChristianなので、幼い時からキリスト教にふれあう機会が多かった。しかし両親は幼児洗礼に反対していたので、僕はchristianではない。それで、中高生のときは反抗期でキリスト教など大嫌いで、ミッションスクールに行くなんて考えられなかったので、今自分がICUにいるのがある意味不思議である。今夏アイルランドに行く時に、向こうはカトリックの国だからということで一様聖書(小型で日英訳つき)を持っていって読んだこともあった。個人的にはキリスト教にはいつも批判的な立場をとっているので、今後も洗礼をうけることはない。ただ、ヨーロッパの政治や文化を考える時、キリスト教は不可欠だとアイルランドでよくわかった。

\item
読んだことありません。でも、アメリカ映画の翻訳をやりたいので、聖書とマザーグースを知ってることは字幕翻訳家の基本らしくて今苦戦しています。なにしろ今までキリスト教の知識がゼロだったのにいきなり英語のインクリをとってしまったので…。でもやってみるとけっこうおもしろいですね。

\item
断片的になら。キリスト教を含めて私煮は宗教というものがどういうものかわかりません、というよりは純粋な宗教は手にとって、みることができあmせん、それができたらその宗教は"お説教"のようなものです。

\item
全然ダメ、でも、この前、かなり、(自分にも他人にもゴマかして生きてきた⇒自分に甘い人間ができた。)自分の人生を反省したので Finalはなんとかしたいなあ。

\item
今インクリをとっているので、聖書を読んでいると、はっきりとは覚えていませんが、昔同じような尾はなしを何かで聞いたことか、読んだことがあるなと尾もいます。私は、特定の宗教を信じているというわけではないので、種教を否定はしませんが、神とかキリスト教とか言われても実感がわかない、というのが今のところの感想です。

\item
あります。とりあえず、ICUをうけるにあたり、少しはキリスト教について知識をとりこもうと思ったため途中まで読みましたが途中で挫折し、阿刀田高の"旧約聖書をしってますか"と"新約聖書を知ってますか"と、新書を何冊か読みました。だから完全によんだということはできません。ちなみに、このシリーズはギリシャ・ローマ神話もあります。
