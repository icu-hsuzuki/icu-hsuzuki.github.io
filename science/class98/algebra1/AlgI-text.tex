%original format is imported from AlgIII-text.tex on April 12, 1997
%original : 「群の定義と例」1997.4.13 - 14
%original : 「部分群」1997. 4. 14
%original : 「剰余類」1997. 4. 24, 25, 26
%original : 「巡回群」1997. 4. 26, 27
%original : 「正規部分群と剰余群」1997.5.10, 11, 14
%original : 「同型と準同型」1997. 5. 19.
%original : 「同型定理」1997. 5. 23, 24, 30
%original : 「群の作用」1997. 5. 30
%original : 「アーベル群の基本定理」1997. 5. 31
%original : 「シローの定理」1997. 6. 2, 3

\documentstyle[12pt]{jarticle}
%A4 Size Setting
\topmargin = -0.5cm
\oddsidemargin = 0cm \evensidemargin = 0cm
\textheight = 23cm \textwidth = 15cm % default 16cm
%A4 size Setting End

%\pagestyle{empty}
\renewcommand{\thepage}{%
	\arabic{section}--\arabic{page}}
\newcommand{\mysection}[1]{%
	\section{#1}\setcounter{page}{1}}

\title{ALGEBRA I}
      
\author{Hiroshi SUZUKI\thanks{E-mail:hsuzuki@icu.ac.jp}\\ 
        Department of Mathematics \\ 
        International Christian University}

\newtheorem{thm}{定理}[section]
\newtheorem{prop}[thm]{命題}
\newtheorem{lemma}[thm]{補題}
\newtheorem{cor}[thm]{系}
\newtheorem{exercise}{練習問題}[section]
\newtheorem{example}{例}[section]
\newtheorem{problem}{問題}[section]
\newtheorem{defin}{定義}[section]
\newenvironment{definition}{\begin{defin} \rm}{\end{defin}}
\newenvironment{ex}{\begin{exercise} \rm}{\end{exercise}}
\newenvironment{eg}{\begin{example} \rm}{\end{example}}
\newenvironment{prob}{\begin{problem} \rm}{\end{problem}}
\newcommand{\remarks}{\vspace{2ex}\noindent{\bf Remarks.\quad}}
\newcommand{\note}{\vspace{2ex}\noindent{\gt 注\quad}}
\newcommand{\proof}{{\gt 証明\quad}}
\newcommand{\qed}{\hfill\hbox{\rule{6pt}{6pt}}}
\newcommand{\bZ}{\mbox{\boldmath $Z$}}
\newcommand{\bN}{\mbox{\boldmath $N$}}
\newcommand{\bR}{\mbox{\boldmath $R$}}
\newcommand{\bC}{\mbox{\boldmath $C$}}
\newcommand{\bQ}{\mbox{\boldmath $Q$}}
\newcommand{\mat}{\mbox{{\rm Mat}}}
\newcommand{\gl}{\mbox{{\rm GL}}}
\renewcommand{\ker}{\mbox{{\rm Ker}}}
\newcommand{\im}{\mbox{{\rm Im}}}
\newcommand{\ch}{\mbox{{\rm char}}}
\newcommand{\aut}{\mbox{{\rm Aut}}}
\newcommand{\mod}{\mbox{{\rm mod}}}
\newcommand{\syl}{\mbox{{\rm Syl}}}

\begin{document}
\setcounter{page}{0}
\maketitle

\newpage
\mysection{群の定義と例}
\begin{definition}
演算 $\circ$ が定義されている集合を $G$ とする。
\begin{itemize}
\item[$G1$] $G$  の任意の元 $a, b, c$ に対して、$(a\circ b)\circ c = a\circ(b\circ c)$ が成り立つ。(結合律)
\item[$G2$] $G$ のある元 $e$ に対して、$e\circ a = a\circ e = a$ が $G$ の任意の元 $a$ について、成り立つ。(単位元の存在)
\item[$G3$] $G$ の各元 $a$ に対し $a\circ b = b\circ a = e$ となる $G$ の元 $b$ が存在する。(逆元の存在)
\item[$G4$] $G$ の任意の元 $a, b$ に対して、$a\circ b = b\circ a$ が、成り立つ。(交換律)
\end{itemize}
$G1$ が成立するとき、半群 (semigroup)、$G1$、$G2$ が成立するとき、モノイド (monoid)  $G1$、$G2$、$G3$ が成立するとき、群 (group) という。さらに $G4$ が成立するものをそれぞれ、可換半群、可換モノイド、可換群という。可換群をアーベル群 (abelian group) ともいう。
\end{definition}

\note
ここで、集合 $G$ に演算 $\circ$ が定義されているとは、
$$f:G \times G \to G\;((a, b)\mapsto a\circ b)$$
が写像であることをいう。

\begin{eg}
$(\bZ, +)$、$(\bQ, +)$、$(\bR, +)$、$(\bC, +)$ は、可換群である。これに対して、$(\bZ, \cdot)$、$(\bQ, \cdot)$、$(\bR, \cdot)$、$(\bC, \cdot)$ は、モノイドである。後の3つは、$0$ 以外の元は、逆元を持つ。そこで、$\#$ をつけたときは、$0$ 以外の元を表すとする。このとき、$(\bQ^{\#}, \cdot)$、$(\bR^{\#}, \cdot)$、$(\bC^{\#}, \cdot)$ も可換群になる。
\end{eg}

\begin{eg}
$\mat(n,\bR)$ で、$n$ 次正方行列全体を表し、$\gl(n,\bR)$ で、$n$ 次正則行列全体を表すものとする。このとき、$(\mat(n,\bR), +)$ は、群、$(\mat(n,\bR), \cdot)$ は、モノイドであるが、$(\gl(n,\bR), \cdot)$ は、群となる。これは、可換群ではない。
\end{eg}

基本的性質
\begin{enumerate}
\item $(G,\circ)$ を半群とする。 $a_1, a_2, \ldots, a_n\in G$ に対して、
$$a_1\circ a_2\circ \cdots \circ a_n = (\cdots ((a_1\circ a_2)\circ a_3)\circ \cdots )\circ a_n$$
で、定義する。このとき、項の前後を入れ替えなければ、括弧の付け方によらず演算の結果は同じである。

例えば、4個の場合は5通りの括弧の付け方があり、それらは、以下のようになる。
\begin{eqnarray*}
((a_1\circ a_2)\circ a_3)\circ a_4 & = & (a_1\circ a_2)\circ (a_3\circ a_4) = a_1\circ(a_2\circ(a_3\circ a_4))\\
& = & a_1\circ((a_2\circ a_3)\circ a_4) = (a_1\circ(a_2\circ a_3))\circ a_4.
\end{eqnarray*}
一般の時はどうであろうか。

\item $(G, \circ)$ がモノイドの時は、単位元は、ただ一つ。

$e, e'\in G$ が $a\circ e = e\circ a = a$、$a\circ e' = e'\circ a = a$ を任意の元 $a\in G$ について満たすとする。
$$e = e\circ e' = e'.$$

\item $(G, \circ)$ モノイドにおいて、$u\in G$ に対して、$u\circ v = v\circ u = 1$ なる $v\in G$ が存在するとき、$u$ を正則元 $v$ を $u$ の逆元という。モノイドの正則元 $u$ の逆元はただ一つ。$v$ と $w$ を $u$ の逆元とする。
$$v = v\circ e = v\circ (u\circ w) = (v\circ u)\circ w = e\circ w = w.$$
\end{enumerate}
.
以後、$a\circ b$ を $ab$、$e$ を $1$、正則元 $u$  の逆元を $u^{-1}$ と、積表示する。また、$x$ の $n$ 個の積を、$x\cdot x\cdots x = x^n$、$x^{-1}$ の $n$ 個の積を、$x^{-1}\cdot x^{-1}\cdots x^{-1} = x^{-n}$ とかく。これにより、任意の整数 $n$ に関して、$x^n$ が定義された。

\begin{prop} \label{prop:x2=1}
群 $G$ の任意の元 $x$ に対して、$x^2 = 1$ が成り立てば $G$ は、可換群(アーベル群)である。
\end{prop}
\proof
$a, b\in G$ とする。このとき、
$$ab = ab(ba)^2 = abbaba = ba.$$
従って、$G$ は、交換律 $G4$ を満たす。
\qed

\medskip
\note
上の証明で、暗黙の内に、一般結合律を用いている。

\begin{prop} \label{prop:unitgroup}
モノイド $G$ の正則元全体 $U(G)$  は、$G$ の演算に関して、群になる。
\end{prop}
\proof
まず、演算が定義できること、すなわち、$G$ の演算に関して、$U(G)$ が閉じていることを示す。$a, b\in U(G)$ とする。
$abb^{-1}a^{-1} = 1$ より、$ab$ は、正則元、従って、$ab\in U(G)$。また、$1\cdot 1 = 1$ より、$1\in U(G)$。$a^{-1} a = aa^{-1} = 1$  より、$a^{-1}$ は、$a$ を逆元に持つから、$a^{-1}\in U(G)$。これらにより、$U(G)$  は、$G$ の演算に関して、群となる。
\qed

\begin{eg}
$\bZ$ を 乗法に関するモノイドとすると、$U(\bZ) = \{\pm 1\}$ となり、これは、積に関して、群となる。同様に、$U(\bQ) = \bQ^{\#}$、$U(\bR) = \bR^{\#}$、$U(\bC) = \bC^{\#}$ なども、これから、群になる。$U(\mat(n,\bR)) = \gl(n,\bR)$。この群を一般線形群という。
\end{eg}

\newpage
\mysection{部分群}
\begin{definition}
群 $G$  の部分集合 $H$ が $G$ の演算に関して群になるとき、$H$ を $G$ の部分群であるといい、$H \leq G$ と書く。
\end{definition}

\begin{prop} \label{prop:subgroup}
$G$ を群とする。このとき、次は、同値。
\begin{itemize}
\item[$(1)$] $H \leq G$。
\item[$(2)$] $(i)$ $\emptyset \neq H \subset G$、$(ii)$ $a, b\in H \to ab\in H$、$(iii)$ $a\in H\to a^{-1}\in H$。
\end{itemize}
\end{prop}
\proof
$(1)\Rightarrow (2)$ 明らか。

$(2)\Rightarrow (1)$ $(ii)$ より、演算に関して閉じている。$(i)$ より、$a\in H$ とすると、$(iii)$、$(ii)$ より、$a^{-1}\in H$、従って、$1 = aa^{-1}\in H$。結合律は、$G$ で成立しているから、$H$ でも成立。従って、$H$ は、$G$ の演算に関して、群となる。
\qed

\medskip
$A, B\subset G$ とする。このとき、
$$AB = \{ab\mid a\in A, \;b\in B\}, \; A^{-1} = \{a^{-1}\mid a\in A\}$$
と書く。特に、$B = \{b\}$ の時、$AB = Ab$、$BA = bA$ ともかく。

\medskip
この記法を用いると、
$$H \leq G \Leftrightarrow \emptyset \neq H \subset G,\; HH\subset H, \;H^{-1} \subset H.$$

\medskip
常に、$G\leq G$、$\{1\}\leq G$ である。これらを自明な部分群と呼び、特に、$\{1\}$ を $1$ とも書く。

\medskip
$S \subset G$ のとき、
$$<S> = \{a_1^{n_1}\cdot a_2^{n_2}\cdots a_r^{n_r}\mid a_1, a_2, \ldots, a_r\in S, \;n_1, n_2, \ldots, n_r\in \bZ\}$$
を $S$ で生成される部分群という。これは、実際、$G$ の部分群となる。

\smallskip
特に、一つの元 $a$ で生成される部分群、
$$<a> = \{ a^n \mid n\in \bZ\}$$
を $a$ で生成された巡回群といい、$a$ を生成元という。

\medskip
群 $G$ の元の数を 位数 (order) といい、$|G|$ と書き $o(a) = |<a>|$ を 元 $a$ の位数と呼ぶ。

\begin{prop} \label{prop:order_a}
巡回群 $<a>$ について、$S = \{n\in \bN\mid a^n = 1\}$ とする。
\begin{itemize}
\item[$(1)$] $S =\emptyset$ の時、$\min S = n$  とする。このとき、$o(a) = |<a>| = n$ で、次が成立。

$(i)$ $a^m = 1 \Leftrightarrow n\mid m$、$(ii)$ $<a> = \{1, a, a^2, \ldots, a^{n-1}\}$。
\item[$(2)$] $S \neq \emptyset$ のとき、$<a>$ は、無限巡回群で、$\ldots, a^{-2}, a^{-1}, 1, a, a^2, \ldots $  は、すべて異なる。
\end{itemize}
\end{prop}
\proof
$(1)$ $m = nq + r$、$q, r\in \bZ$、$0\leq r < n$ とする。$a^n = 1$ より、
$$a^m = a^{nq + r} = (a^n)^q\cdot a^r = a^r$$
で、$r<n$ だから、
$$a^m = 1 \Leftrightarrow r = 0 \Leftrightarrow n\mid m.$$
従って、$<a> \subset \{1,a, a^2, \ldots, a^{n-1}\} \subset <a>$。さらに、
$$a^i = a^j \to a^{i-j} = 1 \to n\mid i-j, \;0\leq i, j\leq n-1$$
より、$i = j$、従って、$1, a, \ldots, a^{n-1}$ は、すべて異なり、$o(a) = n$。

$(2)$ $a^i = a^j$、$i>j$  とすると、$a^{i-j} = 1$、$i-j\in \bN$ より、$S = \emptyset$ に矛盾。従って、$a^i$ は、すべて異なり、$<a>$ は、無限巡回群。
\qed

\begin{eg}
$(\bZ,+)$ 無限巡回群。$a^i$ は、この場合は、$ia$ のこと。

$(\bZ_n,+)$、$\bZ_n = \{\bar{0}, \bar{1}, \ldots, \overline{n-1}\}$ において、演算を
$$\bar{i} + \bar{j} = \overline{i + j \mbox{ の $n$  による剰余}}$$
とすると、$\bZ_n$ は、位数 $n$  の巡回群になる。
\end{eg}

\begin{eg}
$X$ を集合としたとき、$X^X$ で $X$ から $X$ 自身への写像全体の集合を表すものとする。$\sigma\in X^X$ のとき、$a\in X$ の $\sigma$ による像を、$a^{\sigma}$ と書くことにする。$\sigma, \tau\in X^X$ に対して、$\sigma\cdot\tau : a\mapsto (a^{\sigma})^{\tau}$ とすると、$\sigma\cdot\tau\in X^X$ となり、$X^X$ は、モノイドになる。この正則元の全体 $S^X = U(X^X)$ は、$X$ から、$X$ への全単射全体となるが、これを、$X$ 上の対称群という。$X = \{1, 2, \ldots, n\}$  のときは、$S^X$ を $S_n$ と書き、$n$ 次対称群という。$\sigma\in S_n$ のとき、
$$\sigma = {i\choose i^{\sigma}} = \left(\begin{array}{cccc}1 & 2 & \cdots & n\\ 1^{\sigma} & 2^{\sigma} & \cdots & n^{\sigma}\end{array}\right)$$
で表す。また、$(i_1, i_2, \ldots, i_r)$ で、$i^{\sigma}_j = i_{j+1}$、$(j = 1, \ldots, r-1)$、$i_r^{\sigma} = i_1$ それ以外の、$i$ については、$i^{\sigma} = i$ である置換を表し、$r$ 次の巡回置換という。2次の巡回置換を互換という。
\end{eg}

\newpage
\mysection{剰余類}
まず、同値関係、同値類、類別 について復習し、群に応用する。

\begin{definition}
集合 $A$ における関係 $\sim$ が、3つの条件

\smallskip
$(i)$ $a\sim a$\quad $(ii)$ $a\sim b \to b\sim a$\quad $(iii)$ $a\sim b, \;b\sim c\to a\sim c$。

\smallskip\noindent
を満たすとき、$\sim$ を同値関係(equiavelence relation)という。このとき、
$C_a = \{b\in A\mid a\sim b\}$ ($a$ と同値な元全体)を、$a$  を含む同値類という。
\end{definition}

\begin{lemma}
同値類について次が成立。
\begin{itemize}
\item[$(1)$] $a\in C_a$
\item[$(2)$] $b\in C_a \Leftrightarrow C_a = C_b$
\item[$(3)$] $C_a\neq C_b \Leftrightarrow C_a \cap C_b = \emptyset$
\end{itemize}
\end{lemma}
\proof
$(1)$ 同値関係の $(i)$ より明らか。

$(2)$ $(ii)$ によって、$b\in C_a \Leftrightarrow a\in C_b$。従って、$C_a\subset C_b$ を示せばよい。$c\in C_a$ とすると、$a\sim c, a\sim b$ より、$b\sim c$。これは、$c\in C_b$ を意味する。

$(3)$ $c\in C_a\cap C_b$ とすると、$(2)$ より、$C_a = C_c = C_b$ 従って、$(3)$ が成立する。
\qed

\medskip
$\{C_\lambda\mid \lambda\in \Lambda\}$ を異なる同値類としたとき、
$$A = \cup_{\lambda\in \Lambda} C_{\lambda}\; \mbox{ (disjoint union) }$$
を類別、$a_\lambda\in C_\lambda$ を代表元 $\{a_\lambda\mid \lambda\in \Lambda\}$ を完全代表系という。

\smallskip
数学においては、各所で、同値関係を定義し、それによって、類別するということをよく用いるが、群においては、様々な形で、同値関係が、自然に定義される。まずは、部分群が与えられたときのいくつかの同値関係の定義、それによる類別とその応用を述べる。

\medskip
$H\leq G$ のとき、右合同 $\equiv_r$ を、次のように定義する。
$$a, b\in G \mbox{ について、}\;a\equiv_r b\:(\mod H) \Leftrightarrow ab^{-1}\in H.$$
このとき、$\equiv_r$ は、同値関係になる。$a$ を含む同値類は $\{b\in G\mid a\equiv_r  b\:(\mod H)\} = Ha$  となる。これを右剰余類という。したがって、$Ha = Hb \Leftrightarrow ab^{-1}\in H$ である。$G$ における $H$ の異なる右剰余類の集合 $\{Ha_i\mid i\in I\}$ を、$H\backslash G$ と書き、$G$ の右剰余類への分解を、$G$ の $H$ による右分解という。
$$G = \sum_{i\in I}Ha_i = Ha_1 + Ha_2 + \cdots + Ha_n \;\mbox{ (有限の時) }$$
とも書く。$\{a_i \mid i\in I\}$ を $H\backslash G$ の完全代表系という。右剰余類の個数 $|H\backslash G|$ を $H$ の $G$ における指数(index) と呼び、$|G:H|$ と書く。

\medskip
左合同 $\equiv_l$ を、次のように定義する。
$$a, b\in G \mbox{ について、}a\equiv_l b\:(\mod H) \Leftrightarrow a^{-1}b\in H.$$
すると、右合同の時と同様に、$\equiv_l$ は、同値関係になる。$a$ を含む同値類は $\{b\in G\mid a\equiv_l \:(\mod H)\} = Ha$  となる。これを左剰余類といい。左分解なども同様に定義される。

\medskip
$$H\backslash G = \{Ha_i\mid i\in I\}\ni Ha \leftrightarrow a^{-1}H\in \{a_i^{-1}H\mid i\in I\} = G/H$$
なる対応により、右剰余類と、左剰余類は1対1に対応する。特に、
$$|H\backslash G| = |G/H| = |G:H|.$$

\begin{thm} {\rm (Lagrange 1736--1813)}
$G$ を有限群 $H$ を $G$ の部分群とすると、
$$|G| = |G:H||H|.$$
特に、$H$ の位数も、指数も共に $G$ の位数の約数である。
\end{thm}
\proof
まず 任意の $a\in G$ に対して $|H| = |Ha|$ である。このことは、例えば、
$$r_a : H \to Ha \;(h\mapsto ha)$$
が全単射であることから分かる。従って、
$$G = \sum_{i=1}^n Ha_i$$
を $H$ による $G$ の右分解とすると、$n = |G:H|$、$Ha_i\cap Ha_j = \emptyset \;(i\neq j)$。これより、
$$|G| = \left|\sum_{i=1}^n Ha_i\right| = \sum_{i = 1}^n|Ha_i| = \sum_{i=1}^n |H| = n|H| = |G:H||H|.$$
これより、定理の主張が得られる。
\qed

\begin{cor} \label{cor:euler}
$G$ を有限群、$a\in G$ とする。このとき、$o(a)$ は、$|G|$ の約数である。特に、$a^{|G|} = 1$。
\end{cor}
\proof
$o(a) = |<a>|$ だったから、$o(a)$ も部分群の位数であり、$|G|$ を割り切る。
\qed

\begin{cor} \label{cor:prime_order}
位数 $|G|$ が素数の群 $G$ は、巡回群である。
\end{cor}
\proof
$|G|\neq 1$ だから、$1\neq a\in G$ を取る。$o(a)\neq 1$ で、かつ、$o(a)$ は、$|G|$ の約数である。仮定から、$o(a) = |G|$ となり、$|<a>| = |G|$ だから、$G = <a>$ を得る。
\qed

\begin{eg}
$S_3 = \{1, (12), (13), (23), (123), (132)\}$ の6個の元の位数は、それぞれ、1, 2, 2, 2, 3, 3 である。$|S_3| = 6$ だから、$S_3$ の部分群の位数は、1, 2, 3, 6 のいずれかである。位数が、1 の場合は、単位元 $1$ だけからなり、6 の場合は、$S_3$ となることは明らかである。2 又は、3  のときは、共に素数だから、系~\ref{cor:prime_order} より、巡回群、すべて、それぞれ、位数が、2、3 の元で生成されることが分かる。従って、$<(123)> = <(132)> = \{1,(123), (132)\}$  であることに注意すると、位数が、2の部分群は、$<(12)>, <(13)>, <(23)>$ の3種類、位数が、3 のものは、$<(123)>$ の一つだけであることが分かる。この様に、$S_3$ には、全部で、6個の部分群がある。
\end{eg}

\medskip
$H$、$K$ を群 $G$ の部分群とする。$a, b\in G$ について、
$$a\equiv b\; (\mod (H,K)) \Leftrightarrow b = hak \;\mbox{ for some }\;h\in H, k\in K$$
とすると、これは同値関係になる。$a$ を含む同値類は、明らかに、$HaK$ となり、$(H,K)$ の $G$ における両側剰余類と呼ばれる。$G$ における $(H,K)$ の異なる両側剰余類の集合 $\{Ha_iK\mid i\in I\}$ を $H\backslash G/K$ と書き、
$$G = \sum_{i\in I}Ha_iK$$
を $G$ の $(H,K)$ による両側分解という。

\begin{prop} \label{prop:double-coset}
$H, K\leq G$、$a\in G$ とする。$K$ の $K\cap a^{-1}Ha$  による右分解を
$$K = \sum_{j\in J}(K\cap a^{-1}Ha)k_j$$
とすれば、$\{Hak_j\mid j\in J\}$ は、$HaK$ に含まれる $H$ の異なる右剰余類全体と一致する。特に、$G$ が有限群の時は、
$$|HaK| = |H||K:K\cap a^{-1}Ha|.$$
さらに、$G = \sum_{i\in I}Ha_iK$ (すなわち、$\{a_i\mid i\in I\}$ を両側剰余類の完全代表系)とすると、
$$|G:H| = \sum_{i\in I}|K:K\cap a^{-1}Ha_i.$$
\end{prop}
\proof
$k, k'\in K$ に対して、
\begin{eqnarray*}
Hak = Hak' & \Leftrightarrow & akk'^{-1}a^{-1}\in H\\
	&\Leftrightarrow& kk'^{-1}\in K\cap a^{-1}Ha\\
	&\Leftrightarrow& (K\cap a^{-1}Ha)k = (K\cap a^{-1}Ha)k'
\end{eqnarray*}
これよりすべての主張をうる。
\qed

\begin{cor}
$|HK| = |H||K:K\cap H| = |H||K|/|K\cap H|$.
\end{cor}
\proof
上の命題において、$a = 1$ と置く。
\qed

\newpage
\mysection{巡回群}
$G$ が巡回群 (cyclic group) であるとは、$G$ の元 $a$ で、
$$G = <a> = \{a^n\mid n\in bZ\}$$
となるものが存在する事であった。この、$a$ を巡回群 $G$ の生成元と言う。

\begin{thm} \label{thm:cyclic:subgroup}
$G = <a>$  を巡回群とする。このとき、以下が成り立つ。
\begin{itemize}
\item[$(1)$] $1 \neq H\leq G$  とし、$h = \min\{i\in \bN\mid a^i\in H\}$ とすると、$H = <a^h>$。特に巡回群の部分群は、巡回群。
\item[$(2)$] $|G| = n = ml$ とすると、$<a^l>$ は $G$ の位数 $m$ のただ一つの部分群である。
\end{itemize}
\end{thm}
\proof
$(1)$ $1\neq a^i\in H$ とする。$a^{-i} = (a^i)^{-1}\in H$ だから、$\{i\in\bN\mid a^i\in H\}\neq \emptyset$。そこで、$h = \min\{i\in \bN \mid a^i\in H\}$ とする。$a^h\in H$ だから
$$<a^h> = \{(a^h)^i\mid i\in\bZ\}\subset H.$$
一方、$a^i\in H$、$i = hq + r,\;q,r\in \bZ,\;0\leq r<h$ とすると、
$$a^r = a^{i-hq} = (a^i)(a^h)^{-q}\in HH\subset H.$$
$h$ の取り方から $r = 0$。従って、$a^i = a^{hq} \in <a^h>$。

$(2)$ $G = <a> = \{1, a, a^2, \ldots, a^{n-1}\}$、そして、命題~\ref{prop:order_a} によって、この $n$ 個の元はすべて異なる。$n = ml$ とする。このとき、$m$ が、$(a^l)^m = 1$ となる、最小の自然数だから、命題~\ref{prop:order_a} によって
$$<a^l> = \{1,a^l,a^{2l},\ldots,a^{(m-1)l}\},\;(a^{ml} = 1)$$
は、$G$ の位数 $m$ の部分群である。一方、$H\leq G$、$|H| = m$ とし、$h$ を $(1)$ の様に取ると、$H = <a^h>$。$a^n = 1\in H$ だから、$n$ は、$h$ の倍数である。従って、$(1)$ より、
$|H| = n/h = m$。これより、$h = l$ を得る。
\qed

\begin{prop} \label{prop:gcd}
$(\bZ,+)$ は、$1$ で生成される巡回群である。$a_1,a_2,\ldots,a_r\in\bZ$ の最大公約数を $d$ とすると、
$$<a_1, a_2, \ldots, a_r> = <d>.$$
従って、$a_1x_1 + a_2x_2 + \cdots + a_rx_r = d$ となる、整数 $x_1, x_2, \ldots, x_r$ となる整数、$x_1, x_2, \ldots, x_r$ が存在する。
\end{prop}
\proof
$H = <a_1,a_2, \ldots, a_r> = \{a_1x_1 + a_2x_2 + \cdots + a_rx_r \mid x_1, x_2, \ldots, x_r\in\bZ\}$ である。$\bZ$ は、巡回群だから、命題~\ref{thm:cyclic:subgroup} より、その部分群も巡回群。そこで、$H = <c> = \{cx\mid x\in \bZ\}$ とすると、$a_i\in H$ より、$c$ は、$a_i$ の約数、従って、$c$ は、$d$ の約数である。また、$c = a_1x_1 + a_2x_2 + \cdots + a_rx_r$  であることより、$d$ は、$c$ の約数である。従って、$c = \pm d$。従って、$<c> = <d>$。
\qed

\medskip
二つの整数 $m, n$  の最大公約数を $(m,n)$ で表す。

\begin{prop}
$G = <a>$ を位数 $n$ の有限巡回群とする。このとき、
$$<a^r> = <a^{(n,r)}>, \; o(a^r) = n/(n,r).$$
\end{prop}
\proof
$d = (n,r)$ とする。$d\mid r$ より、$<a^r>\subset <a^d>$。また、命題~\ref{prop:gcd} より、$d = nx + ry$ となる、$x,y\in\bZ$ が存在する。従って、
$$a^d = (a^n)^x(a^r)^y = (a^r)^y\in <a^r>.$$
すなわち、$<a^d>\subset <a^r>$ を得る。また、定理~\ref{thm:cyclic:subgroup} より、$o(a^d) = n/d$。
\qed

\begin{eg}
$(\bZ_n,+)$ を考える。$\bZ_n = \{\bar{0},\bar{1},\ldots,\overline{n-1}\}$ で、
$$\bar{a} + \bar{b} = \overline{a+b \mbox{ を $n$ で割った余り}}$$
であった。上で学んだことより、例えば、$\bZ_6$ の部分群は、以下の4つであることが分かる。
$$<\bar{0}> = \{0\},\;<\bar{1}> = <\bar{5}> = \bZ_6,\;<\bar{2}> = <\bar{4}> = \{\bar{0},\bar{2},\bar{4}\},\;<\bar{3}> = \{\bar{0},\bar{3}\}.$$
\end{eg}

\begin{eg}
$(\bZ_n,\cdot)$ は、モノイドであった。演算は、
$$\bar{a}\cdot\bar{b} = \overline{a\cdot b \mbox{ を $n$ で割った余り}}.$$
この、正則元全体 $\bZ^*_n = U(\bZ_n,\cdot)$ は、群となるが、これを既約剰余類群と呼ぶ。また、その位数 $|\bZ^*_n|$ を $\phi(n)$ と書く。$\phi$ は、オイラー関数(Euler function) と呼ばれる。
\begin{eqnarray*}
\bar{a}\in \bZ_n^* & \Leftrightarrow & \bar{a}\bar{b} = \bar{1} \mbox{ for some } \bar{b}\in \bZ^*_n\\
& \Leftrightarrow & ab + qn = 1 \mbox{ for some }b,q\in \bZ\\
& \Leftrightarrow & (a,n) = 1
\end{eqnarray*}
このことから、$\bZ^*_6 = \{\bar{1},\bar{5}\}$ は、位数 $\phi(6) = 2$ の巡回群であること、 $\bZ^*_5 = \{\bar{1},\bar{2},\bar{3},\bar{4}\}$ は、位数 $\phi(5) = 4$ の巡回群であることが分かる。$p$ を素数とすると、$\phi(p) = p-1$ であるが、$\bZ^*_p$  は、巡回群であることが分かる。しかし、一般的には、$\bZ^*_n$ は、巡回群であることも、そうでないこともある。 
また、系~\ref{cor:euler} より以下が成り立つことも分かる。
$(a,6) = 1 \to a^2 \equiv 1 \;(\mod 6)$、また、$(a,5) = 1 \to a^4 \equiv 1 \;(\mod 5)$。
より一般には、$(a,n) = 1\to a^{\phi(n)}\equiv 1 \;(\mod n)$。$n$ が素数の時は、フェルマーの小定理と呼ばれ、一般の場合は、オイラーの定理と呼ばれる。
\end{eg}

\newpage
\mysection{正規部分群と剰余群}
\begin{definition}
$G$ の部分群 $N$ が、$G$ のすべての元 $a\in G$ について $a^{-1}Na = N$ が成り立つとき、$N$ を $G$ の正規部分群であると言い、$N\lhd G$ と書く。この条件は、$Na = aN$ とも書くことが出来る。
\end{definition}

アーベル群の部分群は、すべて正規部分群である。条件は、$G$ の元 $a$ と、$N$ の元とが交換可能と言っているのではない。$aN$ と、$Na$ が、集合として同じだと言うことである。

\begin{lemma} \label{lemma:factorgp}
$G$ の部分群 $N$ について次は同値。
$$N \lhd G \Leftrightarrow (aN)(bN) = abN \;\mbox{ for all }a,b\in G.$$
\end{lemma}
\proof
$(\Rightarrow)\quad$
$aNbN = abNN = abN$。$H\leq G \Rightarrow HH = H$ であることに注意。

\smallskip
$(\Leftarrow)\quad$
$a^{-1}\to a$、$a\to b$、また、$a \to a$、$a^{-1}\to b$ として、適用することによって、
\begin{eqnarray*}
a^{-1}Na & \subset & a^{-1}NaN = a^{-1}aN = N \\
	& = & a^{-1}aNa^{-1}a \subset a^{-1}aNa^{-1}Na = a^{-1}Na
\end{eqnarray*}
\qed

\medskip
$N\lhd G$ とする。$aN, bN\in G/N$ とするとき、補題~\ref{lemma:factorgp} により、
$$aNbN = abN \in G/N.$$
従って、これにより $G/N$ に演算が定義できる。この演算により $G/N$ は群になる。これを $G$ の $N$ による剰余群 (factor group) と呼ぶ。ここで、
$$1_{G/N} = N = 1N,\;(aN)^{-1} = a^{-1}N.$$

\begin{eg}
$\bZ$ はアーベル群であるから、$n\bZ \lhd \bZ$。ここで、
$$a \equiv_r b\;(\mod\; n\bZ) \Leftrightarrow a\equiv b\;(\mod\; n).$$
$\bZ/n\bZ = \{n\bZ, 1 + n\bZ, \ldots, n-1+n\bZ\}$  で、演算は、
$$(a + n\bZ) + (b + b\bZ) = (a+b)+n\bZ  = \overline{a+b} + n\bZ$$
従って、$\bZ/n\bZ$ は、本質的に $\bZ_n$ と同じである。(このことを同型という。)
\end{eg}

\begin{lemma} \label{lemma:index2}
群 $G$ の指数 $2$ の部分群は、正規部分群である。
\end{lemma}
\proof
$a\in H$ とすると、$Ha = H = aH$。一方、$a\not\in H$ とすると、$aH \neq H \neq Ha$ だから、$|G:H| = 2$ だから、$G$ の $H$ による右分解も左分解も
$$G = H + Ha = H + aH$$
となる。従って、$aH = G - H = Ha$。これは、上で示したこととあわせると、$G$ のすべての元 $a$ について、$aH = Ha$ が成り立つ。すなわち $H\lhd G$。
\qed

\begin{eg} \label{eg:ansn}
$|S_n:A_n| = 2$ (Exercise 3.5 参照) だから、$A_n\lhd S_n$。さらに、
$S_n/A_n = \{A_n, (12)A_n\}$ は、位数 $2$ の巡回群である。これは、本質的に、$\bZ_2$、$(\{\pm 1\},\cdot)$  と同じ(同型)である。
\end{eg}

\begin{definition}
$G$  を群、$S\subset G$ とする。このとき、
\begin{enumerate}
\item $N_G(S) = \{x\in G\mid x^{-1}Sx = S\}$ を $S$ の正規化群 (normalizer) という。
\item $C_G(S) = \{x\in G\mid x^{-1}sx = s, \;\mbox{ for all }s\in S\}$ を $S$ の中心化群 (centralizer) という。
\item $Z(G) = C_G(G)$ を $G$ の中心という。
\item $S = \{a\}$ のときは、$N_G(S) = C_G(S)$ であり、これを、$C_G(a)$ とも書く。
\end{enumerate}
\end{definition}

簡単に定義から分かるように、$H\leq G$ とすると、
$$H\lhd G \Leftrightarrow N_G(H) = G.$$

\begin{eg}
$G = S_3$、$H = <(123)> = \{1,(123),(132)\} = A_3$、$C_G(H) = H$、$N_G(H) = G$。
$(12)^{-1}(123)(12) = (132)$ であることに注意。
\end{eg}

$S \subset G$、$x\in G$ のとき、$x^{-1}Sx = \{x^{-1}sx\mid s\in S\}$ を、$S^x$ とも書き、$S$ の $x$ による共役 (conjugate) と言う。$S, T\subset G$ について、$T = S^x$ となる$x\in G$ が存在するとき、$T$ と $S$ は、共役であると言い、$T\sim_G S$ と書く。$\sim_G$ は、${\cal P}(G) = 2^G$ 上の同値関係。また、$G$ の元は、$a\sim_G b$ (すなわち $\{a\}\sim_G \{b\}$) である時、$a$ と $b$ は、共役であるという。その同値類を共役類という。
すなわち 
$$a\sim_G b\Leftrightarrow b = g^{-1}ag\;\mbox{ for some }g\in G.$$

$N\lhd G$ とすると、$a\in N$ に対し $g^{-1}ag \in g^{-1}Ng = N$。よって、$N$ は、$G$ の共役類の和集合である。このことを用いると、正規部分群を決定することが楽になることが多い。

\begin{eg}
$S_3$ の共役類は、$\{1\}, \;\{(12),(13),(23)\}, \;\{(123),(132)\}$ で、部分群の位数は、群の位数の約数だから、$S_3$ の正規部分群は、$1,\;A_3, \;S_3$ のみである。

\smallskip
より一般に
$$\tau = {i \choose i^{\tau}},\;\sigma = {i\choose i^{\sigma}} \;\mbox{ とすると、}\tau^{-1} = {i^{\tau} \choose i}$$
これより、以下を得る。
$$\tau^{-1}\sigma\tau = {i^{\tau} \choose i}{i\choose i^{\sigma}}{i \choose i^{\tau}} = {i^{\tau}\choose i^{\sigma\tau}}.$$
例えば、
$$\tau = \left(\begin{array}{ccccc}1 & 2 & 3 & 4 & 5 \\ 5 & 1 & 2 & 4 & 3\end{array}\right), 
\sigma = (123)(45) \to \tau^{-1}\sigma\tau = (512)(43).$$
従って、巡回置換への分解の型は、保たれ、逆に、型が同じものは、共役であることも分かる。(Exersise 5.8 参照)
\end{eg}

\newpage
\mysection{同型と準同型}
群 $G$ から 群 $G'$ への全単射 $f:G\to G'$ があって、
$$f(ab) = f(a)f(b)\;(\mbox{for all }a,b\in G)$$
を満たすとき、$G$ と $G'$ は同型であると言い、$G\simeq G'$ と表す。このとき、$f$ を $G$ から $G'$ への同型写像という。

\smallskip
同型とは ある1対1対応のもとで、乗積表が同じと言うことである。

\begin{eg} $\omega = (-1+\sqrt{-3})/2$ とする。$\omega$ は、1 の原始 3 乗根である。
$$\begin{array}{|c||c|c|c|}\multicolumn{4}{c}{\bZ_3}\\
\hline + & \bar{0} & \bar{1} & \bar{2}\\
\hline\hline \bar{0} & \bar{0} & \bar{1} & \bar{2}\\
\hline\bar{1} & \bar{1} & \bar{2} & \bar{0}\\
\hline\bar{2} & \bar{2} & \bar{0} & \bar{1}\\
\hline
\end{array} 
\simeq
\begin{array}{|c||c|c|c|}\multicolumn{4}{c}{\bZ/3\bZ}\\
\hline + & 3\bZ  & 1+3\bZ  & 2+3\bZ \\
\hline\hline 3\bZ  & 3\bZ  & 1+3\bZ  & 2+3\bZ \\
\hline1+3\bZ  & 1+3\bZ  & 2+3\bZ  & 3\bZ \\
\hline2+3\bZ  & 2+3\bZ  & 3\bZ  & 1+3\bZ \\
\hline
\end{array}
\simeq
\begin{array}{|c||c|c|c|}\multicolumn{4}{c}{(\{1,\omega,\omega^2\},\cdot)}\\
\hline + & 1  & \omega  & \omega^2 \\
\hline\hline 1  & 1  & \omega  & \omega^2 \\
\hline\omega  & \omega  & \omega^2  & 1 \\
\hline\omega^2  & \omega^2  & 1  & \omega \\
\hline
\end{array}$$
\end{eg}

\begin{eg}
$G = <a> = \{a^n\mid n\in \bZ\}$ を無限巡回群とする。$\bZ$ で有理整数の加法群を表す。
$$f : \bZ \to <a> = \{a^n\mid n\in \bZ\}\;(n\mapsto a^n)$$
とすると、これは、同型写像である。実際、全単射は 命題~\ref{prop:order_a} より明らか。また、
$$f(i + j) = a^{i+j} = a^ia^j = f(i)f(j)$$
を満たす。$\bZ$ の演算を $+$ で、$G$ の演算を積で書いていることに注意。
\end{eg}

\begin{eg}
$G = <a> = \{1, a, a^2, \ldots, a^{n-1}\}$ を位数 $n$ の巡回群とする。
$$f:\bZ/n\bZ \to <a> = \{1, a, a^2, \ldots, a^{n-1}\} \;(i + n\bZ \mapsto a^i)$$
とする。まず、$f$ は写像となることを確認する。
$$i + n\bZ = j + n\bZ \Leftrightarrow i - j \in n\bZ \Leftrightarrow n\mid i-j \Leftrightarrow a^i = a^j.$$
これは、$i + n\bZ$ を他の表し方 $j + n\bZ$ と表しても剰余類として同じならば、$a^i = a^j$ である事を言っている。従って、$\bZ/n\bZ$ の元に対して、$<a>$ の元が一つ定まる。すなわち $f$ は、写像である。このことを $f$ を (写像として) well-defined であるという。また、この写像が全単射であることは、明らか。
$$f((i+n\bZ) + (j+n\bZ)) = f(i+j + n\bZ) = a^{i+j} = a^ia^j = f(i+n\bZ)f(j+n\bZ)$$
従って、$f$ は同型写像で、$G$(任意の位数 $n$ の巡回群)は、$\bZ/n\bZ$ と同型である。
(上で、$f(i+j + n\bZ) = a^{i+j}$ と出来たのは、一般に $i + n\bZ$ の対応先を $a^i$ とし、これが well-defined であることを示してあるからである。もし、$i + n\bZ$ の $i$ を例えば、$0,1,\ldots, n-1$  に限定すれば、写像の定義は問題ないが、$f(i+j + n\bZ) = a^{i+j}$ は、明らかではない。以下に現れる準同型定理の証明と比べよ。)
\end{eg}

\begin{eg}
$\bR$ で、実数全体からなる加法群を表し、$\bR^+$ で、正の実数からなる乗法群を表すものとする。これらは、以下の対応によって同型である。
$$f : \bR \to \bR^+\;(a\mapsto e^a).$$
実際 $f(a+b) = e^{a+b} = e^ae^b = f(a)f(b)$ を満たす。$f$ が全単射であることを示すのは、直接にも出来るが、$g : \bR^+ \to \bR\;(a\mapsto \log a)$ が、$fg = id_{\bR^+}$、$gf = id_{\bR}$ を満たすことからも言える。

\smallskip
一般に、写像 $f:X\to Y$、$g:Y\to X$ が与えられたとき、$gf = id_X$ ならば、$f$ が単射、$g$ が全射を満たす。
\end{eg}

\begin{definition}
$G$、$G'$ を群とする。写像 $f:G\to G'$ が
$$f(ab) = f(a)f(b)\;(\mbox{for all }a,b\in G)$$
を満たすとき、$f$ を 準同型(写像)(homomorphism) と言う。全単射準同型を、同型写像と言う。
\end{definition}

\begin{prop} \label{prop:ker+im}
$f:G\to G'$ を 準同型とする。このとき、次が成立する。
\begin{itemize}
\item[$(1)$] $f(1_G) = 1_{G'}$。
\item[$(2)$] $G$ の任意の元 $a$ について、$f(a^{-1}) = f(a)^{-1}$。
\item[$(3)$] $\im f = \{f(a)\mid \in G\}\leq G'$、即ち、$f$ の像は $G'$ の部分群である。
\item[$(4)$] $\ker f = f^{-1}(1_{G'}) = \{a\in G\mid f(a) = 1_{G'}\}\lhd G$、即ち、$f$ の核は $G$ の正規部分群である。
\end{itemize}
\end{prop}
\proof
$(1)$ $f(1) = f(1\cdot 1) = f(1)f(1)$ である。この両辺に $f(1)^{-1}$ をかけると、$1 = f(1)$ を得る。

$(2)$ $1 = f(1) = f(aa^{-1}) = f(a)f(a^{-1})$ である。この両辺に $f(a)^{-1}$ を左からかけると、$f(a)^{-1} = f(a^{-1})$ を得る。

$(3)$ $f(a)f(b) = f(ab)\in \im f$、$f(a)^{-1} = f(a^{-1})\in \im f$ であるから、$\im\leq G'$ である。

$(4)$ $a, b\in\ker f$ とすると、$f(a) = f(b) = 1$。$f(ab) = f(a)f(b) = 1$、$f(a^{-1}) = f(a)^{-1} = 1$ であるから、$ab\in \ker f$、$a^{-1}\in \ker f$。従って $\ker f\leq G$。さらに、$x\in G$ とすると、
$$f(x^{-1}ax) = f(x^{-1})f(a)f(x) = f(x^{-1})f(x) = f(1) = 1$$
より、$x^{-1}ax\in \ker f$ を得る。従って $\ker f\lhd G$ である。
\qed

\begin{eg}
$N$ を群 $G$ の正規部分群とする。
$$f: G\to G/N\;(a\mapsto aN)$$
を自然な準同型 (cannonical homomorphism) という。実際、
$$f(ab) = abN = aNbN = f(a)f(b)$$
より、$f$ は準同型写像となる。
\end{eg}

\newpage
\mysection{同型定理}
\begin{thm} {\rm (準同型定理)} \label{thm:homo_thm}
$G$、$G'$ を群とし、$f:G\to G'$ を準同型とする。このとき、
$$G/\ker f \simeq \im f.$$
\end{thm}
\proof
$K = \ker f$ とすると、命題~\ref{prop:ker+im} によって $K\lhd G$。$a,b\in G$ に対して、
$$f(a) = f(b) \Leftrightarrow f(a)f(b)^{-1} = f(ab^{-1}) = 1 \Leftrightarrow ab^{-1}\in K \Leftrightarrow Ka = Kb$$
ここで、$\bar{f}(Ka) = f(a)$ とすると以下のことが分かる。
\begin{itemize}
\item $\bar{f}$ は、well-defined(剰余類 $Ka$ の代表元 $a$ の取り方によらず一定)。
\item $\bar{f}$ は、単射($\bar{f}(Ka) = \bar{f}(Kb) \to Ka = Kb$)。
\item $\bar{f}$ は、準同型。なぜなら、
$$\bar{f}(KaKb) = \bar{f}(Kab) = f(ab) = f(a)f(b) = \bar{f}(Ka)\bar{f}(Kb).$$
\item $\im \bar{f} = \im f$。
\end{itemize}
従って、$\bar{f}:G/K \to \im f$ は、同型写像。これより $G/K\simeq \im f$。
\qed

\begin{thm} {\rm (同型定理)} \label{thm:isom_thm}
\begin{itemize}
\item[$(1)$] $H\leq G, \;N\lhd G \Rightarrow NH/N \simeq H/H\cap N$.
\item[$(2)$] $f:G\to G'$ を全射準同型 $H'\lhd G'$、$H = f^{-1}(H')$ とすると、$H\lhd G$ かつ、$G/H \simeq G'/H'$。
\end{itemize}
\end{thm}
\proof
$(1)$ $f:H \to G/N\;(h\mapsto Nh)$  とすると、$f$ は準同型でかつ
$$\im f = f(H) = NH/N,\;\ker f = H\cap N$$
従って、定理~\ref{thm:homo_thm} によって $H/H\cap N\simeq NH/N$。

\smallskip
$(2)$ $g: G \stackrel{f}{\to} G' \to G'/H,\;a\to f(a) \to H'f(a)$ は全射準同型である。
$$\ker g = \{a\in G\mid H'f(a) = H'\} = \{a\in G\mid f(a)\in H'\} = f^{-1}(H')$$
従って、定理~\ref{thm:homo_thm} によって $G/H\simeq G'/H'$ である。
\qed

\begin{cor} $H,\;N\lhd G$、$N\subset H$ ならば $G/H \simeq (G/N)\mbox{\large$/$}(H/N)$。
\end{cor}
\proof
$f:G\to G/N$ を自然な準同型とする。$f^{-1}(H/N) = H$ であるから、定理~\ref{thm:isom_thm} $(2)$ により主張が得られる。
\qed

\medskip
以下に、準同型定理、同型定理の応用例をあげる。

\begin{eg}
$G = <a>$ を巡回群とする。$f:\bZ \to G = <a> = \{a^n\mid n\in\bZ\}\;(i\mapsto a^i)$ は、全射準同型である。実際
$$f(i+j) = a^{i+j} = a^ia^j = f(i)f(j).$$
従って、定理~\ref{thm:homo_thm} によって $\bZ/\ker f\simeq <a>$。$\ker f$ は、巡回群 $\bZ$ の部分群だから 定理~\ref{thm:cyclic:subgroup} によって、$\ker f$ も巡回部分群で $\ker f \simeq n\bZ$ と書ける。$n = 0$ の時、$G$  は、無限巡回群であり、$<a>\simeq \bZ/\{0\}\simeq \bZ$。また、$n\neq 0$ の時は、$G$  は、位数 $n$ の巡回群であり、そのような群は、$\bZ/n\bZ$ と同型である。
\end{eg}

\begin{eg}
$f:\bR \to \bC\;(a\mapsto e^{2\pi ai})$ とすると、$\im f = \{z\in\bC\mid |z| = 1\}$、$\ker f = \{a\in\bR\mid e^{2\pi ai} = 1\} = \bZ$。従って準同型定理により $\bR/\bZ \simeq S^1$。
\end{eg}

\begin{eg} $\gl(n,\bR)$ で $n$ 次正則行列全体からなる $n$ 次一般線形群を表すものとする。このとき、
$$f:G = \left\{\left.\left(\begin{array}{cc} A & O\\ B & C\end{array}\right)\right| A\in \gl(r,\bR),\:C\in \gl(s,\bR)\right\} \to \gl(r,\bR)\;\left(\left(\begin{array}{cc} A & O\\ B & C\end{array}\right)\mapsto A\right).$$
とすると、これは、全射でかつ、
$$\ker f = \left\{\left.\left(\begin{array}{cc} I & O \\ B & C \end{array}\right)\right| C\in\gl(s,\bR)\right\}\lhd G.$$
ここで、$I$ は、$r$ 次の単位行列を表すものとする。定理~\ref{thm:homo_thm} によって $G/\ker f \simeq \gl(r,\bR)$ である。
\end{eg}

$G\rhd N$、$G\geq H\geq N$ とする。$f:G \to G/N\;(a\mapsto aN)$ を自然な準同型とする。このとき、
$$f(H) = \{hN\mid h\in N\} = HN/N\leq G/N$$
逆に、$\bar{H}\leq G/N$ とすると、
$$H = f^{-1}(\bar{H}) = \{h\in G\mid hN \in \bar{H}\}$$
とすると、$N\lhd H\lhd G$ かつ、$f(H) = H/N = \bar{H}$ である。従って、
$${\cal S}(G,N) = \{H\lhd G\mid N\subset H\}\;\mbox{:$G$ の部分群で $N$ を含むもの全体}$$
$${\cal S}(G/N, 1) = \{\bar{H}\mid \bar{H}\leq G/N\}\;\mbox{:$G/N$ の部分群全体}$$
このとき、${\cal S}(G,N)$ の元、$H$ に対し、$f(H) = H/N$ は、${\cal S}(G/N, 1)$ の元であり、${\cal S}(G/N, 1)$ の元 $\bar{H}$ に対して、$f^{-1}(\bar{H})$ は、${\cal S}(G,N)$ の元になっている。この対応は、1対1対応である。

\begin{eg}
$\bZ \to \bZ/12\bZ$ を自然な準同型。このとき、
\begin{eqnarray*}
{\cal S}(\bZ,12\bZ) & = & \{n\bZ\mid 12\bZ \subset n\bZ,\;n\mid 12\}\\
& = &\{12\bZ,6\bZ,4\bZ,3\bZ,2\bZ,\bZ\}\\
{\cal S}(\bZ/12\bZ,0) & = & \{m\bZ/12\bZ \mid n\mid 12\}\\
& = &\{0,6\bZ/12\bZ,4\bZ/12\bZ,3\bZ/12\bZ,2\bZ/12\bZ,\bZ/12\bZ\}
\end{eqnarray*}
\end{eg}

\newpage
\mysection{群の作用}
\begin{definition}
$G$ を群、$X$ を集合とする。
$$f:X\times G \to X,\; ((\alpha,a)\in X\times G \mapsto f(\alpha,a) = \alpha^a)$$
が、次の2条件
$$\alpha^1 = \alpha,\; \alpha^{ab} = (\alpha^a)^b$$
を満たすとき、群 $G$ は、集合 $X$ に作用しているといい、$X$ は、$G$-集合であるという。
\end{definition}

$X$ を $G$-集合、$\alpha,\beta\in X$ のとき、
$$\alpha\sim_G \beta \Leftrightarrow \alpha^a = \beta\;\mbox{ for some } a\in G$$
とすると、$\sim_G$ は、同値関係になる。この同値類を $G$-軌道 ($G$-orbit) という。このとき、$\alpha$  を含む $G$-軌道 $\{\alpha^a\mid a\in G\}$ を、$\alpha^G$ ともかく。また、$|\alpha^G|$ を 軌道の長さという。

$G$-軌道がただ一つであるとき、すなわち $X = \alpha^G$ であるとき、可移(可遷とも言う。transitive)という。

$G_\alpha = \{a\in G\mid \alpha^a = \alpha\}$ とすると、$G_\alpha\leq G$ である。$\beta = \alpha^a$ ならば、$x\in G_{\alpha}$ としたとき、$G_\beta = a^{-1}G_{\alpha}a$ である。実際、
$$\beta^{a^{-1}xa} = \alpha^{aa^{-1}xa} = \alpha^{xa} = \alpha^a = \beta$$
だから、$G_\beta \subset a^{-1}G_{\alpha}a$、また、$\alpha = \beta^{a^{-1}}$ だから、$G_\alpha \subset aG_{\beta}a^{-1}$ である。これは、$a^{-1}G_{\alpha}a\subset G_{\beta}$ を意味する。

$G_{\alpha}$ を $\alpha$ の $G$ における安定部分群と呼ぶ。

\begin{thm} \label{thm:orbit_length}
$|\alpha^G| = |G:G_{\alpha}|$
\end{thm}
\proof
$a,b\in G$ に対して、
$$\alpha^a = \alpha^b\Leftrightarrow \alpha^{ab^{-1}} = \alpha \Leftrightarrow ab^{-1}\in G_\alpha \Leftrightarrow G_{\alpha}a = G_{\alpha}b$$
従って、$\phi : G_{\alpha}\backslash G \to \alpha^G\;(G_{\alpha}a\mapsto \alpha^a)$ は、全単射である。
\qed

\medskip
$X$ を $G$-集合とする。$a\in G$ に対して、$\sigma(a) : X\to X\;(\alpha\mapsto \alpha^a)$ とすると、$\sigma(a)$ は、$X$ 上の全単射である。さらに、
$$\sigma : G \to S^X\;(a\mapsto \sigma(a))$$
は、準同型になる。この $\sigma$ の核($\ker \sigma$)を $\ker(X,G)$ と書き $G$ の $X$ 上の作用の核という。(確認せよ。)

逆に $\sigma: G\to S^X$ を準同型とすると、$X\times G\to X\;((\alpha,a)\mapsto \alpha^{\sigma(a)})$ により、$X$ は、$G$-集合となる。この準同型を $G$ の置換表現という。

\note
群 $G$ から、一般線形群 $\gl(n,\bC)$ への準同型を線形表現という。$G$ を有限群とすると、必ず線形表現があることが知られているが、以下に見るように、置換表現はいつでも存在し、さらに、核が 1 となる(これを忠実な表現という)置換表現がある。

\begin{eg}
$H\leq S^X$ とすると、$\sigma : H \to S^X$ を埋め込みとすると、明らかに置換表現となるから、作用が定義できる。実際、$X\times H\to X,\;((\alpha,h)\mapsto \alpha^h)$ により、$X$ は、$H$ 集合になる。特に、$S_n$ の部分群は、$\{1,2,\ldots, n\}$ に自然に作用する。
($H\leq S^X$ を $X$ 上の置換群、$H\leq S_n$ を $n$ 次置換群という。)
\end{eg}

\begin{eg}
$H\leq G$ とする。
$$f : H\backslash G \times G \to H\backslash G,\;((Hx,a)\mapsto Hxa)$$
により、$G$ は、$H\backslash G$ に作用する。実際、
$$(Hx)^1 = Hx1 = Hx,\;(Hx)^{ab} = Hxab = (Hxa)^b = (Hxa)^b = ((Hx)^a)^b$$
この作用は明らかに可移である。
\begin{eqnarray*}
\ker(H\backslash G,G) & = & \{a\in G\mid Hxa = Hx\;\mbox{ for all }x\in G\}\\
& = & \bigcap_{x\in G}x^{-x}Hx \lhd G
\end{eqnarray*}
特に、$H = 1$ のとき、右正則表現という。$\ker(G,G) = 1$。$\sigma:G \to S^G$ は、単射である。従って、$G$ は、$S^G$ の部分群と同型である。
\end{eg}

\begin{eg}
$X = G$ とし、
$$G\times G \to G \;((x,a)\mapsto a^{-1}xa = x^a)$$
とすると、この作用により、$G$ は、$G$-集合になる。この場合は、
$$G_x = \{a\in G\mid x^a = x = \{a\in G\mid a^{-1}xa = x\} = C_G(x)$$
となる、$x^G$ は、$x$ の共役類となり、定理~\ref{thm:orbit_length} から、
$$|x^G| = |G:C_G(x)|$$
である。特に、共役類の元の数は、$|G|$ の約数である。
$$\ker(G,G) = \{a\in G\mid x^a = x,\;x\in G\mbox{ for all }x\in G\} = Z(G)\lhd G$$
また、
$$|x^G| = 1 \Leftrightarrow |G:C_G(a)| = 1 \Leftrightarrow x\in Z(G).$$
\end{eg}



\newpage
\mysection{アーベル群の基本定理}
以下で述べる有限生成アーベル群の基本定理は、数学の様々なところで用いられる基本的な定理である。巡回群は、群の中で構造が一番わかりやすいものであるが、巡回群の直積がアーベル群であることは簡単に分かる。基本定理は、生成元の数が有限個のアーベル群は、巡回群の直積であることを主張している。また、この定理によって、有限生成アーベル群の同型類を完全に記述することが出来る。有限生成でないアーベル群については、未解決な問題も多い。

\begin{thm} \label{thm:fin_gen_abelian}
有限生成のアーベル群 $G$ は、巡回群の直積である。さらに、
$$G = E_1\times E_2 \times \cdots \times E_m \times I_1 \times I_2 \times \cdots \times I_r,$$
$E_i\simeq \bZ_{e_i}$、$e_i>1$、$i = 1, 2, \ldots, m$、$e_i\mid e_{i+1}$、$i = 1, 2, \ldots, m-1$、$I_j\simeq \bZ$、$j = 1, 2, \ldots, r-1$ となる。この様な直積分解において、$(e_1,e_2,\ldots, e_m;r)$ は、 $G$ により一意的に定まる。(これを不変系という。)
\end{thm}

\begin{lemma} \label{lemma:reduction}
有限生成のアーベル群 $G = <x_1,x_2,\ldots, x_n>$ の元、
$y_1 = x^{a_1}x^{a_2}\cdots x^{a_n}$、$a_i\in \bZ$、$(i = 1,2,\ldots, n)$、$(a_1,a_2,\ldots, a_n) = 1$ とすると、$G = <y_1,y_2,\ldots, y_n>$ となる $y_2, \ldots, y_n$ が存在する。
\end{lemma}
\proof
$m = |a_1| + |a_2| + \cdots + |a_n|$ による帰納法で示す。

$m = 1$ とする。これは、ある $i$ について $a_i = \pm 1$ を意味するから $y_1 = x_i^{\pm 1}$ これより、この場合は明らか。

$m > 1$ とする。$(a_1,a_2,\ldots, a_n) = 1$ より $|a_i| \geq |a_j| > 0$ となる $a_i, a_j$ が存在する。このとき、$|a_i - \epsilon a_j| < |a_i|$ となるように、$\epsilon = \pm 1$ を取ることが出来る。すると、
$$y_1 = x_1^{a_1}\cdots x_i^{a_i-\epsilon a_j}\cdots (x_jx_i^{\epsilon})^{a_j}\cdots x_n^{a_n}$$
だから、ここで、$b_k = a_k$ $(k\neq i)$、$b_i = a_i-\epsilon a_j$、$z_l = x_l$、$(l\neq j)$、$z_j = x_jx_i^{\epsilon}$ とおけば、
$$(b_1,b_2,\ldots, b_n) = (a_1,\ldots, a_{i-1},a_i-\epsilon a_j,a_{i+1},\ldots, a_n) = (a_1,\ldots, a_n) = 1$$ 
が成り立ち、かつ、
$$G = <x_1,\ldots, x_n> = <x_1,\ldots, x_{j-1},x_jx_i^{\epsilon},x_{j+1},\ldots, x_n> =<z_1,z_2,\ldots, z_n>.$$
ここで、$y_1 = z_1^{b_1}\cdots z_n^{b_n}$、しかし、$|b_1| + \cdots + |b_n| < m$  だから、帰納法によって $y_2, \ldots, y_n$ で、$G = <y_1,\ldots, y_n>$ となるものが存在する。
\qed

\medskip
{\em Proof of Theorem~\ref{thm:fin_gen_abelian}\quad}
$G$ の元 $x_1,\ldots, x_n$ を次の条件を満たすように取る。
\begin{enumerate}
\item $G = <x_1,\ldots, x_n>$ で $n$ は、最小。
\item $G = <x_1,\ldots, x_{i-1},y_i,\ldots, y_n>$ ならば、$x_i$ の位数は、高々 $y_i$ の位数に等しい。
\end{enumerate}
($x_1$ を 1 を満たす、すべて可能な組の中で最小位数に取り、$x_{i-1}$ まで取ったとき、$x_i$ を $x_1,\ldots, x_{i-1}$ を固定したとき、すべて可能な $y_i$ の中で最小位数に取ればよい。)

$<x_i> = E_i$、$|E_i| = e_i$ と置く。($\infty$ も含める。)
このとき、$G = E_1\times \cdots \times E_n$ かつ、$e_i\mid e_{i+1}$ を示す。
$$\phi : E_1 \times \cdots \times E_n \to G,\;((x_1^{a_1},\ldots,x_n^{a_n})\mapsto xx_1^{a_1}\cdots x_n^{a_n})$$
は、準同型で、かつ条件1より、全射である。

$x_i^{a_i}\cdots x_j^{a_j} = 1$ とし、添字は順に大きくなるように取る。ここに現れるものについて、$0 < |a_k| \leq e_k$ としてよい。$d = (a_i,\ldots, a_j)$ とし、$a_k = db_k$ と書く。すると、$(b_i,\ldots, b_j) = 1$ である。従って、補題~\ref{lemma:reduction} により、
$y_i = x_i^{b_i}\cdots x_j^{b_j}, \;y_{i+1},\ldots, y_n$ があって、
$<x_i,\ldots,x_n> = <y_i,\ldots, y_n>$。$y_i$ の定義により、
$$y_i^d = x_i^{db_i}\cdots x_j^{db_j} = x_i^{a_i}\cdots x_j^{a_j} = 1.$$
従って、$y_i$ の位数は、高々 $d$。一方、$x_i$ の取り方から、
$$e_i\leq o(y_i) \leq d \leq |a_i| \leq e_i.$$
従って、$e_i = d = |a_i|$。すなわち $x_i^{a_i} = 1$。従って、$\phi$ は、同型である事がわかる。さらに、$x_i^{e_i}x_{i+1}^{e_{i+1}} = 1$ だから、$d = (e_i,e_{i+1})$ とすると、上で示したことより、$e_i = d$ である。従って、$e_i\mid e_{i+1}$。$e_1,\ldots, e_m$ が、有限で $e_{m+1}$ からさきが、無限なら、
$$G = E_1 \times \cdots \times E_m \times I_1 \times \cdots \times I_r$$
$E_i \simeq \bZ_{e_i}$、$I_j \simeq \bZ$、$e_i\mid e_{i+1}$、$1 < e_i$ が得られる。

\smallskip
(一意性)\quad
\begin{eqnarray*}
G & = & E_1 \times \cdots \times E_m\times I_1 \times \cdots \times I_r\\
& = & E'_1 \times \cdots \times E'_{m'} \times I'_1 \times \cdots \times I'_{r'}
\end{eqnarray*}
$E_i = <x_i>$、$I_j = <z_j>$、$I'_j = <w_j>$、$E'_i = <y_i>$、$e_i = |E_i|$、$e'_i = |E'_i|$ とする。$T(G)$ を $G$ の位数が有限なもの全体とする。すると、
$$T(G) = E_1 \times \cdots \times E_m = E'_1 \times \cdots \times E'_m$$
$p$ を $e_1$ の一つの素因子とし、$p$ は、$e'_{i+1}$ 以降を割り切るとする。すると、
$$T(G)_p = \{x\in T(G)\mid x^p = 1\} = <x_1^{e_1/p},\ldots, x_m^{e_m/p}> = <y_{i+1}^{e'_{i+1}/p},\ldots,y_{m'}^{e'_{m'}/p}>$$
だから、位数を考えると、$|T(G)_p| = p^m = p^{m'-i}$ となり、$m\leq m'$ を得る。同様にして、$m'\leq m$ を得るから、$m = m'$、$p\mid e'_1$ を得る。ここで、
$$T(G)^p = \{x^p\mid x\in T(G)\} = <x_1^{p},\ldots, x_m^{p}> = <y_{1}^{p},\ldots,y_{m}^{p}>$$
だから、それぞれの、直積因子の位数は、$e_1/p,\ldots, e_m/p$ および、$e'_1/p,\ldots, e'_m/p$ となり、数学的帰納法により、これらは、等しくなる。従って、$e_i = e'_i$ を得る。

ここでさらに、$e = e_m$ として、
$$G^e = \{x^e\mid x\in G\} = <z_1^{e},\ldots, z_r^{e}> = <w_{1}^{e},\ldots,w_{r'}^{e}>$$
$$G^{2e} = \{x^{2e}\mid x\in G\} = <z_1^{2e},\ldots, z_r^{2e}> = <w_{1}^{2e},\ldots,w_{r'}^{2e}>$$
だから、$G^e/G^{2e}$ の位数を考えると、$2^r = 2^{r'}$ を得るから、$r = r'$ となる。
\qed

\begin{eg}
$G$ を 位数 12 のアーベル群とする。$12 = e_1e_2\cdots e_r$、$e_1\mid e_2,\;e_2\mid e_3 \ldots e_{r-1}\mid e_r$ だから、$e^{r}\mid 12$。これらは、
$(12), (2,6)$ のいずれかしかないことが分かる。
すなわち、$G$ は、$\bZ_{12}$(位数 12 の巡回群)又は、$\bZ_2\times \bZ_6$ のいずれかに同型であることが分かる。
\end{eg} 

\newpage
\mysection{シローの定理}
この節では、$G$ は、有限群を表すものとする。

\begin{definition}
$p$ を素数とするとき、$G$ が $p$--群であるとは、$|G|$ が $p$--べき $(|G| = p^r)$ であることである。$|G| = p^ng',\;(p,g') = 1$ としたとき($p^n$ を $|G|$ を割り切る最大べき)$P$ が $G$ のシロー $p$-部分群 (Sylow-$p$-group) であるとは、$P\leq G$  かつ、$|P| = p^n$。
\end{definition}

ここでまず問題となるのは、以下の事である。
\begin{enumerate}
\item 任意の有限群と任意の素数 $p$ について、シロー $p$--部分群は存在するか。
\item シロー $p$--部分群について何が言えるか。複数個の シロー $p$--部分群があったとき、それらは、同型だろうか、共役だろうか、いくつあるだろうか。
\end{enumerate}

二つの補題を準備する。
\begin{lemma} \label{lemma:p-ele_in_abelian}
$p$ を素数とし、$G$ は、その位数 $|G|$ が $p$ で割り切れるアーベル群とする。このとき、$G$ に位数 $p$ の元が存在する。
\end{lemma}
\proof
定理~\ref{thm:fin_gen_abelian} より明らか。
\qed

\begin{lemma} \label{lemma:nontrivial_center}
$G\neq 1$ とし、$G$ の任意の真部分群に対して、$p\mid |G:H|$ が成り立つとする。このとき、$p\mid |Z(G)|$ である。特に、$Z(G)\neq 1$。
\end{lemma}
\proof
Exercise 8.2 参照。
\qed

\begin{thm} \label{thm:sylow:existence}
ある素数 $p$ に対して、$p^r \mid |G|$ ならば $G$ に位数 $p^r$ の部分群が存在する。特に、$G$ のシロー $p$--部分群が存在する。
\end{thm}
\proof
$G$ の位数に関する帰納法で証明する。明らかに、$G\neq 1$ としてよい。

{\em Case 1.\quad} $G$ の真部分群 $H$ で、$(|G:H|,p) = 1$ となるものが存在するとき。

このときは、$p^r\mid |G| = |G:H||H|$ だから、$p^r\mid |H|$ である。従って、帰納法の仮定より、$H$ の部分群 $P$ で、位数が $p^r$ のものが存在する。$P\leq G$ だから、定理の主張が成り立つ。

\smallskip
{\em Case 2. \quad} $G$ の任意の真部分群に対して $p\mid |G:H|$ が成立するとき。

このときは、補題~\ref{lemma:nontrivial_center} により $p\mid |Z(G)|$ である。$Z(G)$ は、アーベル群だから、補題~\ref{lemma:p-ele_in_abelian} より、$Z(G)$ に位数 $p$ の元 $x$ があることが分かる。$A = <x>$ とする。$|A| = |<x>| = o(x) = p$。$A\leq Z(G)$ だから、$A\lhd G$。剰余群 $G/A$ を考えると、
$$p^r \mid |G| = |G:A||A| = p|G/A|$$
だから、$p^{r-1}\mid |G/A|$ となる。従って、帰納法の仮定により、$G/A$ は、位数が、$p^{r-1}$ の部分群 $P_1$ を持つ。$\pi:G \to G/A$ を自然な準同型とし、$P = \pi^{-1}(P_1)$ とすると、$A \leq P$ だから、$P/A\simeq P_1$、従って、$|P| = p^r$ を得る。
\qed

\medskip
$\syl_p(G)$ によって、$G$ のシロー $p$--部分群全体を表すものとする。
\begin{thm} \label{thm:sylow:conjugate}
$G$ を有限群、$p$ を素数とする。
\begin{itemize}
\item[$(1)$] $H$ を $G$ の $p$--部分群とすると、$G$ の $p$--シロー部分群 $P$ で $H$ を含むものが存在する。
\item[$(2)$] $P, Q$ を $G$ のシロー $p$--部分群とすると、$Q = g^{-1}Pg$ となる $G$ の元 $g$ が存在する。
\item[$(3)$] $|\syl_p(G)|\equiv 1\;(\mod\:p)$、すなわち、シロー $p$--部分群の個数は、$kp + 1$、$(k\in\bZ)$ と書ける。
\end{itemize}
\end{thm}
\proof
$(1)$ $P\in \syl_p(G)$、$H$ を $G$ の $p$--部分群とする。$(P,H)$ による $G$ の両側分解を
$$G = Pa_1H + Pa_2H + \cdots + Pa_rH$$
とすると、命題~\ref{prop:double-coset} によって、
$$|G:P| = \sum_{i=1}^r |H:H\cap a^{-1}_iPa_i|$$
が成り立つ。$P$ は、$G$ のシロー $p$--部分群だから、左辺は $p$ と素である。従って、右辺のある項は、$p$ と互いに素であることが分かる。$(|H:H\cap a^{-1}_iPa_i|, p) = 1$ とする。$H$ は、$p$ 群だから、
$$|H| = |H:H\cap a^{-1}_iPa_i||H\cap a^{-1}_iPa_i|$$
を考えると、$|H:H\cap a^{-1}_iPa_i| = 1$ すなわち $H = H\cap a^{-1}_iPa_i$ である。従って、$H\subset a^{-1}_iPa_i$ である。$a^{-1}_iPa_i$ は、$P$ と同じ位数の $G$ の部分群だから $a^{-1}_iPa_i$ は、$G$  の シロー $p$--部分群、従って、$H$ を含む $G$ のシロー $p$--部分群が存在する。

$(2)$ $Q = H$ と置くと、$(1)$ の証明より $G$ の元 $g = a_i$ によって、$Q\subset g^{-1}Pg$ となる。$|Q| = |P| = |g^{-1}Pg|$ だから、$Q = g^{-1}Pg$ が成り立つ。

$(3)$ $\syl_p(G)\times G \to \syl_p(G),\;((P,a)\mapsto a^{-1}Pa)$ とすると($a^{-1}Pa\in \syl_p(G)$ に注意)、$\syl_p(G)$ は、$G$-集合となり、$(2)$ により、この作用は可移である。この作用に関する $P\in\syl_p(G)$ の安定部分群を、$N$ とすると、
$$N = \{a\in G\mid P^a = P\} = \{a\in G\mid a^{-1}Pa = P\} = N_G(P)$$
である。定理~\ref{thm:orbit_length} により、
$$|\syl_p(G) = P^G = |G:N| = |G:N_G(P)|$$
となる。ここで、$G$ の $(N,P)$ による両側分解を
$$G = Nb_1P + Nb_2P + \cdots + Nb_rP$$
ただし、$b_1 = 1$ とする。命題~\ref{prop:double-coset} によって
$$|G:N| = \sum_{i=1}^r|P:P\cap b_i^{-1}Nb_i|$$
となる。$P\subset N_G(P) = N$ だから、$|G:N|\mid |G:P|$ で、$|G:N|$ は、$p$ と互いに素である。従って、右辺にも $p$ と互いに素な項があることになるが、
\begin{eqnarray*}
|P:P\cap b_i^{-1}Nb_i| = 1 & \Leftrightarrow & P = P\cap b_i^{-1}Nb_i \;\Leftrightarrow \; P \subset b_i^{-1}Nb_i\\
&\Leftrightarrow & b_iPb_i^{-1} \subset N \;\mbox{ すなわち } b_iPb_i^{-1}\in \syl_p(N)\\
& \Leftrightarrow & b_iPb_i^{-1} = n^{-1}Pn = P\;\mbox{ となる }n\in N \;\mbox{ が存在する。}\\
& \Leftrightarrow & b_i\in N = N_G(P) \\
& \Leftrightarrow & Nb_iP = N1P (= N) \\
& \Leftrightarrow & b_i = b_1
\end{eqnarray*}
すなわち、この様な $b_i$ は、ただ一つであり、他の $|P:P\cap b_i^{-1}Nb_i|$ は、すべて $p$ で割り切れる。$|P:P\cap b_1^{-1}Nb_1| = |P:P\cap N| = |P:P| = 1$ だから、
$$|\syl_p(G)| = |G:N_G(P)| = kp+1$$
と書ける。
\qed

\end{document} 